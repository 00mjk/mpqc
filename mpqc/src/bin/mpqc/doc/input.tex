
\chapter{MPQC Input}
\index{input}

MPQC is an object-oriented program that directly allows the user to specify
objects that MPQC then manipulates to obtain energies, properties, etc.
Thus, the input may seem a bit complex to the beginner; however, many
calculations will quite similar to the examples given later in this
chapter.  The best way to get started is to use one of the example input
files and modify it to meet your needs.

MPQC starts off by creating a
\hyperref{\clsnm{ParsedKeyVal}}
         {\clsnm{ParsedKeyVal} (see Section~}
         {)}
         {ParsedKeyVal}
object that parses the input file specified on the command line.
It is the \clsnm{ParsedKeyVal} object that dictates the format
of the input file.  It is basically a free format input that
associates keywords and logical groupings of keywords with
values.  The values can be scalars, arrays, or objects.

The keywords recognized by \exenm{mpqc} begin with either an
{\ttfamily mpqc}
or {\ttfamily default} prefix.  That is, they
must be nested between an {\ttfamily mpqc:(} and a {\ttfamily )}.
Alternately, each keyword can be prefixed by {\ttfamily mpqc:}.
The primary keywords are given below.  Some of the keywords
specify objects, in which case the object will require more
\clsnmref{KeyVal} input.  These objects are created from the
input by using their \clsnmref{KeyVal} constructors, which
are documented in
\begin{latexonly}
Chapters~\ref{firstclasschapt} through~\ref{lastclasschapt}.
\end{latexonly}
\begin{htmlonly}
several chapters that follow.
\end{htmlonly}

\begin{description}
\item[{\ttfamily mole}] This is the most important keyword for
        \exenm{mpqc}.  It specifies the \clsnmref{MolecularEnergy} object.
        This is an object that knows how to compute the energy of a
        molecule.
\item[{\ttfamily opt}] This keyword must be specified for optimizations.
        It specifies an \clsnmref{Optimizer} object.
\item[{\ttfamily freq}] This keyword must be specified to compute frequencies.
        It specifies an \clsnmref{MolecularFrequency} object.
\item[{\ttfamily checkpoint}] The value of this keyword is boolean.  If
        true the calculation will be checkpointed.  The default is to
        checkpoint.
\item[{\ttfamily restart}] The value of this keyword is boolean.  If true
        \exenm{mpqc} will attempt to restart the calculation.  If the
        checkpoint file is not found, the calculation will continue as if
        the value were false. The default is true.
\item[{\ttfamily do\_energy}] The value of this keyword is boolean.  If
        true a single point energy calculation will be done for the
        \clsnmref{MolecularEnergy} object given with the \ttfamily{mole}
        keyword.  The default is true.
\item[{\ttfamily do\_gradient}] The value of this keyword is boolean.  If
        true a single point gradient calculation will be done for the
        \clsnmref{MolecularEnergy} object given with the \ttfamily{mole}
        keyword.  The default is false.
\item[{\ttfamily optimize}] The value of this keyword is boolean.  If
        true and the \ttfamily{opt} keyword was set to a valid value,
        then an optimization will be performed.  The default is true.
\item[{\ttfamily write\_pdb}] The value of this keyword is boolean.  If
        true a PDB file with the molecular coordinates will be written.
\item[{\ttfamily filename}] The value of this keyword is a string that
        gives a name from which checkpoint and other filenames are
        constructed.  The default is the basename of the input file.
\end{description}

There also some utility keywords that tell mpqc some technical
details about how to do the calculation:
\begin{description}
\item[{\ttfamily message}]
        This optional keyword gives a \clsnmref{MessageGrp} object
        that sets up MPQC for parallel processing.  The default
        is a \clsnmref{ProcMessageGrp} that runs MPQC on a single
        processor.  If another \clsnmref{MessageGrp} is used because
        of command line options or due to the environment in which MPQC
        is run, then this keyword may be ignored.
        See \hyperref{the chapter on running mpqc}
                     {Chapter~}
                     {}
                     {running mpqc}
        for other ways to specify the \clsnmref{MessageGrp} object.
\item[{\ttfamily debug}]
        This optional keyword gives a \clsnmref{Debugger} object
        which can used to help find the problem if MPQC encounters
        a catastrophic error.
\item[{\ttfamily matrixkit}]
        This optional keyword gives a \clsnmref{SCMatrixKit} specialization
        which is used to produce matrices of the desired type.
        The default is a \clsnmref{ReplSCMatrixKit} which replicates
        matrices on all of the nodes.
\end{description}


\section{The \clsnm{ParsedKeyVal} Input Format}
\label{ParsedKeyVal}
\label{KeyVal}
\index{ParsedKeyVal}
\index{KeyVal}

The \clsnm{KeyVal} class provides a means for users to associate keywords
with values.  \clsnm{ParsedKeyVal} is a specialization of \clsnm{KeyVal}
that permits keyword/value associations in text such as an input file or a
command line string.

The package is flexible enough to allow complex structures and arrays as
well as objects to be read from an input file.

\subsection{Assignment}

As an example of the use of \clsnmref{ParsedKeyVal}, consider the following
input:
\begin{verbatim}
x_coordinate = 1.0
y_coordinate = 2.0
x_coordinate = 3.0
\end{verbatim}
Two assignements will be made.  The keyword \verb|x_coordinate| will be
associated with the value \verb|1.0| and the keyword \verb|y_coordinate|
will be assigned to \verb|2.0|.  The third line in the above input
will have no effect since \verb|x_coordinate| was assigned previously.

\subsection{Keyword Grouping}

Lets imagine that we have a program which needs to read in the
characteristics of animals.  There are lots of animals so it might be
nice to catagorize them by their family.  Here is a sample format for
such an input file:
\begin{verbatim}
reptile: (
  alligator: (
    legs = 4
    extinct = no
    )
  python: (
    legs = 0
    extinct = no
    )
  )
bird: (
  owl: (
    flys = yes
    extinct = no
    )
  )
\end{verbatim}

This sample illustrates the use of \vrbl{keyword} \verb|=| \vrbl{value}
assignments and the keyword grouping operators \verb|(| and \verb|)|.
The keywords in this example are
\begin{verbatim}
reptile:alligator:legs
reptile:alligator:extinct
reptile:alligator:legs
reptile:python:size
reptile:python:extinct
bird:owl:flys
bird:owl:extinct
\end{verbatim}

The \verb|:|'s occuring in these keywords break the keywords into
smaller logical units called keyword segments.  The sole purpose of this
is to allow persons writing input files to group the input into easy to
read sections.  In the above example there are two main sections, the
reptile section and the bird section.  The reptile section takes the
form \verb|reptile| \verb|:| \verb|(| \vrbl{keyword} \verb|=| \vrbl{value}
assignments \verb|)|.  Each of the keywords found between the
parentheses has the \verb|reptile:| prefix attached to it.  Within each
of these sections further keyword groupings can be used, as many and as
deeply nested as the user wants.

Keyword grouping is also useful when you need many different programs to
read from the same input file.  Each program can be assigned its own
unique section.

\subsection{Array Construction}

A method of specifying arrays of data would be useful.  One way to
do this would be as follows:
\begin{verbatim}
array: (
  0 = 5.4
  1 = 8.9
  2 = 3.7
  )
\end{verbatim}
The numbers \verb|0|, \verb|1|, and \verb|2| in this example are keyword
segments which serve as indices of \verb|array|.  However, this syntax
is somewhat awkward and array construction operators have been provided
to simplify the input for this case.  The following input is equivalent
to the above input:
\begin{verbatim}
array = [ 5.4 8.9 3.7 ]
\end{verbatim}

More complex arrays than this can be imagined.  Suppose an array of
complex numbers is needed.  For example the input
\begin{verbatim}
carray: (
  0: ( r = 1.0  i = 0.0 )
  1: ( r = 0.0  i = 1.0 )
  )
\end{verbatim}
could be written as
\begin{verbatim}
carray: [
  (r = 1.0 i = 0.0)
  (r = 0.0 i = 1.0)
  ]
\end{verbatim}
which looks a bit nicer than the example without array construction
operators.

Furthermore, the array construction operators can be nested in about
every imaginable way.  This allows multidimensional arrays of
complicated data to be represented.

It would be nice to just extend an array that is already defined.  This
feature has not been implemented; however, user demand for it might
result in its implementation.  The operators reserved for this purpose
are \verb|+[| and \verb|]|.

\subsection{Table Construction}

Although the array contstruction operators will suit most requirements
for enumerated lists of data, in some cases the input can still look
ugly.  This can, in some cases, be fixed with the table construction
operators, \verb|{| and \verb|}|.

Suppose a few long vectors of the same length are needed and the data in
the \verb|i|th element of each array is related or somehow belong
together.  If the arrays are so long that the width of a page is
exceeded, then data that should be seen next to each other are no longer
adjacent.  The way this problem can be fixed is to arrange the data
vertically side by side rather than horizontally.  The table
construction operators allows the user to achieve this in a very simple
manner.
\begin{verbatim}
balls: (
  color    = [  red      blue     red   ]
  diameter = [   12       14       11   ]
  material = [  rubber  vinyl   plastic ]
  bouces   = [  yes      no       no    ]
  coordinate = [[ 0.0  0.0  0.0]
                [ 1.0  2.0 -1.0]
                [ 1.0 -1.0  1.0]]
  )
\end{verbatim}
can be written
\begin{verbatim}
balls: (
  { color diameter material bounces     coordinate}
                  
  {  red     12    rubber    yes     [ 0.0  0.0  0.0]
     blue    14    vinyl     no      [ 1.0  2.0 -1.0]
     red     11    plastic   no      [ 1.0 -1.0  1.0] }
  )
\end{verbatim}
The length and width of the table can be anything the user desires.

\subsection{Value Substitution}
\label{ParsedKeyVal:valsub}

Occasionally, a user may need to repeat some value several times in an
input file.  If the value must be changed, it would be nice to only
change the value in one place.  The value substitution feature of
\clsnmref{ParsedKeyVal} always the user to do this.  Any place a value can
occur the user can place a \verb|$|.  Following this a keyword must be
given.  This keyword must have been assigned before the attempt is made
to use its value in a value substitution.

Here is an example illustrating most of the variable substition
features:
\begin{verbatim}
default:linewidth = 130
testsub: (
  ke: (
    ke_1 = 1
    ke_2 = 2
    ke_3: (
      ke_31 = 31
      ke_32 = 32
      )
    )
  kx = $ke
  r1 = 3.0
  r2 = $r1
  linewidth = $:default:linewidth
  )
\end{verbatim}
is the same as specifying
\begin{verbatim}
testsub: (
  ke: (
    ke_1 = 1
    ke_3: (
      ke_31 = 31
      ke_32 = 32
      )
    ke_2 = 2
    )
  linewidth = 130
  r2 = 3.0
  r1 = 3.0
  kx: (
    ke_1 = 1
    ke_2 = 2
    ke_3: (
      ke_31 = 31
      ke_32 = 32
      )
    )
  )
\end{verbatim}
It can be seen from this that value substitution can result in entire
keyword segment hierarchies being copied, as well as simple
substitutions.


\subsection{Expression Evaluation}

It is nice to be able to multiply numbers together directly in the input
file.  For example, you may need to change the units on a number and
want to see where exactly the number is coming from.  This motivated the
development of an expression evaluation facility in the
\clsnmref{ParsedKeyVal} package.  This facility is still under development,
but some primitive expressions can be evaluated in the current release.

Suppose your program requires several parameters \verb|x1|, \verb|x2|,
and \verb|x3|.  Furthermore, suppose that their ratios remain fixed for
all the runs of the program that you desire.  It would be best to
specify some scale factor in the input that would be the only thing that
has to be changed from run to run.  If you don't want to or cannot
modify the program, then this can be done directly with
\clsnmref{ParsedKeyVal} as follows
\begin{verbatim}
scale = 1.234
x1 = ( $:scale *  1.2 )
x2 = ( $:scale *  9.2 )
x3 = ( $:scale * -2.0 )
\end{verbatim}
So we see that to the right of the ``\verb|=|'' the characters
``\verb|(|'' and ``\verb|)|'' are the expression construction operators.
This is in contrast to their function when they are to the left of the
``\verb|=|'', where they are the keyword grouping operators.

Currently, the expression must be binary and the data is all converted
to double.  If you use the expression construction operators to produce
data that the program expects to be integer, you will certainly get the
wrong answers (unless the desired value happens to be zero).

\subsection{Objects}

An instance of an object can be can be specified by surrounding it's
classname with the ``\verb|<|'' and ``\verb|>|'' operators immediately
after the keyword naming the data.

A pointer to a single object can be associated with multiple keywords by
using
\hyperref{value substitution}
         {value substitution (see Section~}
         {)}
         {ParsedKeyVal:valsub}.
This is accomplished by holding references to all objects once they are
read in.

Consider the following input for a linked list, where \clsnm{A}
\begin{verbatim}
a1<A>: (
    next<A>: (
        next<B>: (
            bdata = 4
            next<A>:()
            )
        )
    )
a2 = $:a
\end{verbatim}

The \verb|a1| list would contain two \verb|A| objects followed by a
\verb|B| object followed by another \verb|A| object.  The \verb|a2| list
refers to exactly the same object as \verb|a1| (not a copy of
\verb|a1|).



\section{Sample Input Files}

The easiest way to get started with \exenm{mpqc} is to start with
one of sample inputs that most nearly matches your problem.  All
of the samples inputs shown here can be found in the directory
\filnm{src/bin/mpqc/samples}.

\subsection{SCF Energy}
\index{SCF!example}
\index{examples!SCF}

The following input will compute the SCF energy of water.

\input{scf.in.tex}

\subsection{MP2 Energy}
\index{MP2!example}
\index{examples!MP2}

The following input will compute the MP2 energy of water.

\input{mp2.in.tex}

\subsection{SCF Optimization}
\index{optimization!example}
\index{examples!optimization}

The following input will optimize the geometry of water using
the quasi-newton method.

\input{scfopt.in.tex}

\subsection{SCF Frequencies}
\index{frequencies!example}
\index{examples!frequencies}

The following input will compute SCF frequencies by finite
displacements.  A thermodynamic analysis will also be
performed.  If optimization input is also provided, then the
optimization will be run first, then the frequencies.

\input{scffreq.in.tex}

\subsection{Giving Coordinates and a Guess Hessian}
\index{fixed coordinates!example}
\index{examples!fixed coordinates}
\index{guess hessian!example}
\index{examples!guess hessian}
\index{coordinates!example}
\index{examples!coordinates}

The following example shows several features that are really independent.
The variable coordinates are explicitly given, rather than generated
automatically.  This is especially useful when a guess hessian is to be
provided, as it is here.  This hessian, as given by the user, is not
complete and the \clsnmref{QNewtonOpt} object will fill in the missing
values using a guess the hessian provided by the \clsnmref{MolecularEnergy}
object.  Also, fixed coordinates are given in this sample input.

\input{mancoor.in.tex}

\subsection{Optimizations with Hydrogen Bonds}
\index{hydrogen bonds!example}
\index{examples!hydrogen bonds}

The automatic internal coordinate generator will fail if it cannot find
enough redundant internal coordinates.  In this case, the internal
coordinate generator must be explicitly created in the input and given
extra connectivity information, as is shown below.

\input{hbondopt.in.tex}
