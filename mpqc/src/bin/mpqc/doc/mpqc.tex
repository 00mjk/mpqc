
\documentclass[letterpaper%,twoside
              ]{report}

\tolerance=1500
\hbadness=1500

\usepackage{html}
\usepackage{verbatim}
\usepackage{alltt}
\usepackage{makeidx}
\usepackage{scdoc}
\usepackage{longtable}
\setlongtables
\pagestyle{headings}
%\markboth{\chapter}{\section}

\setcounter{tocdepth}{1}

\makeindex

\begin{document}

\title{The MPQC User Manual \\
       Version \input{version.tex}}

\author{Curtis L. Janssen \and Edward T. Seidl \and Ida M. B. Nielsen
        \and Michael E. Colvin}

\date{\today}

\maketitle

\begin{abstract}
The Massively Parallel Quantum Chemistry program (MPQC) computes
the properties of molecules, {\it ab initio}, on a wide variety
of computer architectures.

\begin{sloppypar}
MPQC can compute closed shell and general restricted open shell
Hartree-Fock energies and gradients, second order open shell
perturbation theory (OPT2[2]) and Z-averaged perturbation theory
(ZAPT2) energies, and second order closed shell M\o{}ller-Plesset
perturbation theory energies and gradients.  It also includes a
robust internal coordinate geometry optimizer that efficiently
optimizes molecules with many degrees of freedom.
\end{sloppypar}

MPQC runs on Unix compatible workstations (Intel/Linux, SGI R8000,
IBM RS/6000), symmetric multi-processors (SGI Power Challenge), and
massively parallel computers (IBM SP2, Intel Paragon).
\end{abstract}

\tableofcontents
\listoftables

\chapter{Introduction}
The Massively Parallel Quantum Chemistry program (MPQC) computes
the properties of molecules, {\it ab initio}, on a wide variety
of computer architectures.

MPQC can compute closed shell and general restricted open shell
Hartree-Fock energies and gradients, second order open shell
perturbation theory (OPT2[2]) and Z-averaged perturbation theory
(ZAPT2) energies, and second order closed shell M\o{}ller-Plesset
perturbation theory energies and gradients.  It also includes a
robust internal coordinate geometry optimizer that efficiently
optimizes molecules with many degrees of freedom.

MPQC runs on Unix compatible workstations (Intel/Linux, SGI R8000,
IBM RS/6000), symmetric multi-processors (SGI Power Challenge),
and massively parallel computers (IBM SP2, Intel Paragon).

MPQC is designed using object-orient programming techniques and
implemented in the C++ programming language.  This design has
propagated even to the input of MPQC which is object-oriented
as well.  The user specifies a group of objects that the program
creates when it starts.  MPQC manipulates the objects to produce
energies, gradients, geometries, and properties.

The object-oriented nature of the input makes this user manual
read like a bit like a programmers manual.  Object-orient oriented
terminology is thus a natural way to describe the input and a
short \htmlref{glossary}{glossary} is provided in the back for
those in need.


\chapter{Compiling MPQC}
\index{compiling}
\index{installing}


These instructions are for a machine running Unix or a clone. The codes
work under Linux, and AIX, at least.  The code has also be built under MS
Windows NT using the Cygnus development tools.

\section{Prerequisites}
 Make sure that you have the following programs available. Most can be
found at any GNU software FTP repository.
  \begin{itemize}

    \item Compilers for the C and C++ languages are needed. I currently use
          GNU gcc 2.7.2.1 on Linux Redhat 4.1 with a recent kernel and a
          few other updates, most important of which is a recent libc (I
          use glibc 2.0). I also use xlC (3.1.3) on an IBM SP2.  I highly
          recommend the GNU compiler which can be installed on many
          architectures.  It can be obtained by FTP from prep.ai.mit.edu.

    \item Parts of GNU libg++ are needed.  It isn't necessary to link with
          libg++, but you need to have the genclass command in your path
          and it must be able to locate the prototype classes.  If you
          don't have GNU libg++, here is how to get the necessary parts
          without installing the whole thing:

       \begin{itemize}
          \item Get libg++-2.7.2.tar.gz by FTP from prep.ai.mit.edu and
                untar it.
          \item Move the libg++/src/gen directory to wherever you want it.
          \item Copy the libg++/genclass/genclass.sh shell script to
                somewhere in your path and name it ``genclass''.  Make
                it executable.
          \item Modify genclass so that PROTODIR points to your gen
                directory.
       \end{itemize}

    \item GNU flex (version 2.5.2 or greater): This is a lexical analyzer
          generator used to generate code to read input files. Make sure
          that FlexLexer.h from flex is in your include path. You may need
          to give the path to it to configure with an argument that looks
          something like: \type{--with-include=-I/usr/local/include}

    \item GNU bison (version 1.24 or greater): This is a parser generator used
          to generate code to read input files.

    \item GNU gmake (version 3.70 or greater): GNU specific extensions to make
          are used extensively.

    \item perl: This is used to convert template classes to macros,
          generate documentation, generate and check the validation suite
          etc.  To compile the code, either perl 4 or perl 5 will work.  To
          generate the validation inputs and automatically check the
          outputs, perl 5.003 is needed.

  \end{itemize}

\section{Configuration}

 You can optionally make a sister directory to \filnm{SC} which will be
used to hold all of the files generated by the compilation.  This directory
is usually named to indicate the architecture (e.g. \filnm{SC.i686-linux})
and will be referred to as the target directory below.

 In the target directory execute the "configure" command which is located
in the SC source directory. This command should build a hierarchy of
target directories and the necessary makefiles. Do a \type{configure
--help} to see a list of options.  Useful options to configure
include:

\begin{description}
\item[\type{--enable-debug}] Options for debugging will be given to
the compiler.
\item[\type{--disable-parallel}] Do not try to find communications
libraries.
\item[\type{--enable-ref-debug}] Check for overwrites and overflows
for reference counts.  Implied by ``--enable-debug''.
\item[\type{--disable-ref-macros}] Use template classes for reference
counting.  This doesn't work with any known compiler.  The default is to
use a CPP macro to generate a class definition.
\item[\type{--enable-cross-compile}] If this option is set then the
configure script will take care to not execute any compiled test programs.
\item[\type{--enable-shared-libs}] This will generate shared objects and
link with them instead of standard ``.a'' libraries.  This works on a
Linux-ELF system.
\item[\type{--with-cc}] Gives the name of the C compiler.
\item[\type{--with-cxx}] Gives the name of the C++ compiler.
\item[\type{--with-ranlib}] Gives the name of the archive indexing utility.
\item[\type{--with-ar}] Gives the name of the program than makes libraries.
\item[\type{--with-ld}] Gives the name of the object linker.
\item[\type{--with-include}] Gives directories in which include files
should be sought.  For example, \type{--with-include=-I/u/local/inc -I/u/cljanss/include}
\item[\type{--with-libs}] Specifies libraries that executables should be
linked with.
\item[\type{--with-libdirs}] Gives the directories in which libraries
should be sought.
\end{description}

  If you would like to further customize your target directory,
you can edit \filnm{src/lib/config.h} and \filnm{lib/LocalMakefile} to
suit your needs.

\section{Compiling MPQC}

 First you need to generate several include files to satisfy
cross-dependencies between libraries.  Do these by changing directory into
your target directory and type \type{make interface}.  If you get error
messages in this step do a \type{make distclean} after correcting the
problem since there might be some incorrectly generated include or source
files remaining.

 Now you are ready to build the libraries and executable.  Do this
by typing \type{make} in your target directory.  All the libraries and
a single executable, \exenm{src/bin/mpqc/mpqc}, should be built.
If you are running on a symmetric multi-processor, you can use GNU
make to parallelize the compilation.  To compile four files at a time,
type \type{make JOBS=-j4}.

\section{Validating MPQC}

 After you compile MPQC, you should run it through the validation suite.
You should also run the validation suite if you upgrade your operating
system software, since this could change shared libraries that are linking
with MPQC and could affect the results.

 The validation suite is in \filnm{src/bin/mpqc/validate}.  The
input files are divided into several categories:
\begin{description}
  \item[\filnm{h2o}] These are simple tests that exercise many of MPQC's
        features.

  \item[\filnm{h2omp2}] Tests that further exercise MP2.

  \item[\filnm{mbpt}] These tests exercise MP2 as well as the open-shell
        perturbation theory methods.  The various available
        algorithms are tested as well.

  \item[\filnm{ckpt}] Tests the checkpoint and restart capabilities.

  \item[\filnm{symm1}] Tests of point group symmetry.

  \item[\filnm{symm2}] More point group symmetry tests.  These use basis
        sets with higher angular momentum than \filnm{symm1}.

  \item[\filnm{basis1}] A variety of basis sets are tested for first row
        atoms along with hydrogen and helium.

  \item[\filnm{basis2}] Basis sets test for second row atoms.

\end{description}

  To generate the input files change into the \filnm{src/bin/mpqc/validate}
subdirectory of your object directory (where you compiled MPQC) and type
\type{make inputs}.  This will create a \filnm{run} subdirectory containing
MPQC input files ending with the \filnm{.in} suffix.  Files ending with a
\filnm{.qci} suffix will also be placed in the \filnm{run} directory.
These contain a description of the calculation that is used by the utility
program that checks the results of the validation suite.

  Next you need to run the calculations.  You might want to start with the
\filnm{h2o} input files first since they shouldn't take too long to run.
For the \filnm{ckpt} calculations you should run the calculations
alphabetically by input file name.  This ensures that the checkpoint files
will be created before they are needed.

  While the test calculations are running you can begin monitoring the
results by typing \type{make checkrun} in the \filnm{src/bin/mpqc/validate}
directory.  This will first do some consistency checks between pairs of
files selected from the \filnm{ckpt}, \filnm{mbpt}, \filnm{symm1}, and
\filnm{symm2} groups of calculations (see below for a discussion of the
output for comparison of two files).  Then each file is individually
checked.  An \type{ok} is printed next the test name, if it looks like the
calculation made it to the end.  A \type{missing} means the output file
could not be found.  A \type{failed} means that the output file has
problems (or the calculation may still be running).

  Next you will want to see if your compiled MPQC produces the same answer
as ours.  Note that we don't say ``the correct answer'' here, because our
reference validation suite has not been verified relative to a independent
code, except for a few spot checks.  If you find that MPQC doesn't produce
the same answer as another quantum chemistry program that you trust, then
please promptly notify us and send all the details.  The reference
validation suite is distributed separately from MPQC.  Obtain it (hopefully
it is available where you got the source code) and untar it in the
\filnm{src/bin/mpqc/validate} subdirectory of your MPQC object code
directory.  This will create the \filnm{ref} subdirectory.  Now you can
type \type{make check} and outputs in the \filnm{ref} and \filnm{run}
directories will be pairwise compared.

  When files are pairwise compared first the status (\type{ok},
\type{missing}, or \type{failing}) for each file is printed.  If both
statuses are \type{ok} then an \type{E:} is printed followed by the number
of digits to which the results agree.  If they agree to all digits
\type{99} is printed.  If a gradient was computed, then \type{Grad:} is
printed followed by the number of digits to which the gradients in least
agreement agree.  Likewise if frequencies were computed.

  Finally, you can do a detailed comparison of the contents of the
\type{ref} and \type{run} subdirectories by typing \type{make diff}.




\chapter{MPQC Input}
\index{input}

MPQC is an object-oriented program that directly allows the user to specify
objects that MPQC then manipulates to obtain energies, properties, etc.
Thus, the input may seem a bit complex to the beginner; however, most
calculations should be quite similar to the one of the examples given later
in this chapter.  The best way to get started is to use one of the example
input files and modify it to meet your needs.

MPQC starts off by creating a
\hyperref{\clsnm{ParsedKeyVal}}
         {\clsnm{ParsedKeyVal} (see Section~}
         {)}
         {ParsedKeyVal}
object that parses the input file specified on the command line.
It is the \clsnm{ParsedKeyVal} object that dictates the format
of the input file.  It is basically a free format input that
associates keywords and logical groupings of keywords with
values.  The values can be scalars, arrays, or objects.

The keywords recognized by MPQC begin with the \keywd{mpqc} prefix.
That is, they must be nested between an \keywd{mpqc:(} and a \keywd{)}.
Alternately, each keyword can be individually prefixed by \keywd{mpqc:}.
If they are not found with the \keywd{mpqc} prefix, the prefix
\keywd{default} will also be considered.  The primary keywords are given
below.  Some of the keywords specify \htmlref{objects}{pkvobject}, in which
case the object will require more \clsnmref{KeyVal} input.  These objects
are created from the input by using their \clsnmref{KeyVal} constructors,
which are documented in
\begin{latexonly}
Chapters~\ref{firstclasschapt} through~\ref{lastclasschapt}.
\end{latexonly}
\begin{htmlonly}
several chapters that follow.
\end{htmlonly}

\begin{description}
\item[\keywd{mole}] This is the most important keyword for \exenm{mpqc}.
        It specifies the \clsnmref{MolecularEnergy}
        \htmlref{object}{pkvobject}.  This is an object that knows how to
        compute the energy of a molecule.  The specializations of
        \clsnmref{MolecularEnergy} that are most commonly used here are
        \clsnmref{CLSCF}, \clsnmref{HSOSSCF}, and \clsnmref{MBPT2}.
\item[\keywd{opt}] This keyword must be specified for optimizations.  It
        specifies an \clsnmref{Optimize} \htmlref{object}{pkvobject}.
        Usually, \clsnmref{QNewtonOpt} is best for finding minima and
        \clsnmref{EFCOpt} is best for transition states.
\item[\keywd{freq}] This keyword must be specified to compute frequencies.
        It specifies an \clsnmref{MolecularFrequencies}
        \htmlref{object}{pkvobject}.
\item[\keywd{thread}] This specifies an \htmlref{object}{pkvobject} of type
        \clsnmref{ThreadGrp} that can be used to advantage on shared-memory
        multiprocessor machines for certain types of calculations.  This
        keyword can be overridden by giving the \clsnmref{ThreadGrp} in the
        environment or command line.  See
        \hyperref{the section on running MPQC}{Section~}{}{running mpqc}
        for more information.
        This feature is not fully implemented in the current release.
\item[\keywd{checkpoint}] The value of this keyword is boolean.  If true,
        then optimizations will be checkpointed after each iteraction.  The
        default is to checkpoint.
\item[\keywd{savestate}] The value of this keyword is boolean.  If true,
        then the states of the optimizer and wavefunction objects will be
        saved after the calculation completes.  The default is to save
        state.
\item[\keywd{restart}] The value of this keyword is boolean.  If true
        \exenm{mpqc} will attempt to restart the calculation.  If the
        checkpoint file is not found, the calculation will continue as if
        the value were false. The default is true.
\item[\keywd{restart\_file}] This gives the name of a file from which
        restart information is read.  If the file name ends in
        \filnm{.wfn} the \keywd{mole} object will be restored.  Otherwise,
        the \keywd{opt} object will be restored.  The default file name
        is formed by appending \filnm{.ckpt} to the input file name
        with the extension removed.
\item[\keywd{do\_energy}] The value of this keyword is boolean.  If
        true a single point energy calculation will be done for the
        \clsnmref{MolecularEnergy} object given with the \keywd{mole}
        keyword.  The default is true.
\item[\keywd{do\_gradient}] The value of this keyword is boolean.  If
        true a single point gradient calculation will be done for the
        \clsnmref{MolecularEnergy} object given with the \keywd{mole}
        keyword.  The default is false.
\item[\keywd{optimize}] The value of this keyword is boolean.  If
        true and the \keywd{opt} keyword was set to a valid value,
        then an optimization will be performed.  The default is true.
\item[\keywd{write\_pdb}] The value of this keyword is boolean.  If
        true a PDB file with the molecular coordinates will be written.
\item[\keywd{filename}] The value of this keyword is a string that
        gives a name from which checkpoint and other filenames are
        constructed.  The default is the basename of the input file.
\item[\keywd{print\_timings}] If this is true, timing information
        is printed at the end of the run.  The default is true.
\end{description}

There also some utility keywords that tell mpqc some technical
details about how to do the calculation:
\begin{description}
\item[\keywd{debug}] This optional keyword gives a \clsnmref{Debugger}
        \htmlref{object}{pkvojbect} which can used to help find the problem
        if MPQC encounters a catastrophic error.
\item[\keywd{matrixkit}]
        This optional keyword gives a \clsnmref{SCMatrixKit} specialization
        which is used to produce matrices of the desired type.
        The default is a \clsnmref{ReplSCMatrixKit} which replicates
        matrices on all of the nodes.
\end{description}


\section{The \clsnm{ParsedKeyVal} Input Format}
\label{ParsedKeyVal}
\label{KeyVal}
\index{ParsedKeyVal}
\index{KeyVal}

The \clsnm{KeyVal} class provides a means for users to associate keywords
with values.  \clsnm{ParsedKeyVal} is a specialization of \clsnm{KeyVal}
that permits keyword/value associations in text such as an input file or a
command line string.

The package is flexible enough to allow complex structures and arrays as
well as objects to be read from an input file.

\subsection{Assignment}

As an example of the use of \clsnmref{ParsedKeyVal}, consider the following
input:
\begin{verbatim}
x_coordinate = 1.0
y_coordinate = 2.0
x_coordinate = 3.0
\end{verbatim}
Two assignements will be made.  The keyword \verb|x_coordinate| will be
associated with the value \verb|1.0| and the keyword \verb|y_coordinate|
will be assigned to \verb|2.0|.  The third line in the above input
will have no effect since \verb|x_coordinate| was assigned previously.

\subsection{Keyword Grouping}

Lets imagine that we have a program which needs to read in the
characteristics of animals.  There are lots of animals so it might be
nice to catagorize them by their family.  Here is a sample format for
such an input file:
\begin{verbatim}
reptile: (
  alligator: (
    legs = 4
    extinct = no
    )
  python: (
    legs = 0
    extinct = no
    )
  )
bird: (
  owl: (
    flys = yes
    extinct = no
    )
  )
\end{verbatim}

This sample illustrates the use of \vrbl{keyword} \verb|=| \vrbl{value}
assignments and the keyword grouping operators \verb|(| and \verb|)|.
The keywords in this example are
\begin{verbatim}
reptile:alligator:legs
reptile:alligator:extinct
reptile:alligator:legs
reptile:python:size
reptile:python:extinct
bird:owl:flys
bird:owl:extinct
\end{verbatim}

The \verb|:|'s occuring in these keywords break the keywords into
smaller logical units called keyword segments.  The sole purpose of this
is to allow persons writing input files to group the input into easy to
read sections.  In the above example there are two main sections, the
reptile section and the bird section.  The reptile section takes the
form \verb|reptile| \verb|:| \verb|(| \vrbl{keyword} \verb|=| \vrbl{value}
assignments \verb|)|.  Each of the keywords found between the
parentheses has the \verb|reptile:| prefix attached to it.  Within each
of these sections further keyword groupings can be used, as many and as
deeply nested as the user wants.

Keyword grouping is also useful when you need many different programs to
read from the same input file.  Each program can be assigned its own
unique section.

\subsection{Array Construction}

A method of specifying arrays of data would be useful.  One way to
do this would be as follows:
\begin{verbatim}
array: (
  0 = 5.4
  1 = 8.9
  2 = 3.7
  )
\end{verbatim}
The numbers \verb|0|, \verb|1|, and \verb|2| in this example are keyword
segments which serve as indices of \verb|array|.  However, this syntax
is somewhat awkward and array construction operators have been provided
to simplify the input for this case.  The following input is equivalent
to the above input:
\begin{verbatim}
array = [ 5.4 8.9 3.7 ]
\end{verbatim}

More complex arrays than this can be imagined.  Suppose an array of
complex numbers is needed.  For example the input
\begin{verbatim}
carray: (
  0: ( r = 1.0  i = 0.0 )
  1: ( r = 0.0  i = 1.0 )
  )
\end{verbatim}
could be written as
\begin{verbatim}
carray: [
  (r = 1.0 i = 0.0)
  (r = 0.0 i = 1.0)
  ]
\end{verbatim}
which looks a bit nicer than the example without array construction
operators.

Furthermore, the array construction operators can be nested in about
every imaginable way.  This allows multidimensional arrays of
complicated data to be represented.

It would be nice to just extend an array that is already defined.  This
feature has not been implemented; however, user demand for it might
result in its implementation.  The operators reserved for this purpose
are \verb|+[| and \verb|]|.

\subsection{Table Construction}

Although the array contstruction operators will suit most requirements
for enumerated lists of data, in some cases the input can still look
ugly.  This can, in some cases, be fixed with the table construction
operators, \verb|{| and \verb|}|.

Suppose a few long vectors of the same length are needed and the data in
the \verb|i|th element of each array is related or somehow belong
together.  If the arrays are so long that the width of a page is
exceeded, then data that should be seen next to each other are no longer
adjacent.  The way this problem can be fixed is to arrange the data
vertically side by side rather than horizontally.  The table
construction operators allows the user to achieve this in a very simple
manner.
\begin{verbatim}
balls: (
  color    = [  red      blue     red   ]
  diameter = [   12       14       11   ]
  material = [  rubber  vinyl   plastic ]
  bouces   = [  yes      no       no    ]
  coordinate = [[ 0.0  0.0  0.0]
                [ 1.0  2.0 -1.0]
                [ 1.0 -1.0  1.0]]
  )
\end{verbatim}
can be written
\begin{verbatim}
balls: (
  { color diameter material bounces     coordinate}
                  
  {  red     12    rubber    yes     [ 0.0  0.0  0.0]
     blue    14    vinyl     no      [ 1.0  2.0 -1.0]
     red     11    plastic   no      [ 1.0 -1.0  1.0] }
  )
\end{verbatim}
The length and width of the table can be anything the user desires.

\subsection{Value Substitution}
\label{ParsedKeyVal:valsub}

Occasionally, a user may need to repeat some value several times in an
input file.  If the value must be changed, it would be nice to only
change the value in one place.  The value substitution feature of
\clsnmref{ParsedKeyVal} always the user to do this.  Any place a value can
occur the user can place a \verb|$|.  Following this a keyword must be
given.  This keyword must have been assigned before the attempt is made
to use its value in a value substitution.

Here is an example illustrating most of the variable substition
features:
\begin{verbatim}
default:linewidth = 130
testsub: (
  ke: (
    ke_1 = 1
    ke_2 = 2
    ke_3: (
      ke_31 = 31
      ke_32 = 32
      )
    )
  kx = $ke
  r1 = 3.0
  r2 = $r1
  linewidth = $:default:linewidth
  )
\end{verbatim}
is the same as specifying
\begin{verbatim}
testsub: (
  ke: (
    ke_1 = 1
    ke_3: (
      ke_31 = 31
      ke_32 = 32
      )
    ke_2 = 2
    )
  linewidth = 130
  r2 = 3.0
  r1 = 3.0
  kx: (
    ke_1 = 1
    ke_2 = 2
    ke_3: (
      ke_31 = 31
      ke_32 = 32
      )
    )
  )
\end{verbatim}
It can be seen from this that value substitution can result in entire
keyword segment hierarchies being copied, as well as simple
substitutions.


\subsection{Expression Evaluation}

It is nice to be able to multiply numbers together directly in the input
file.  For example, you may need to change the units on a number and
want to see where exactly the number is coming from.  This motivated the
development of an expression evaluation facility in the
\clsnmref{ParsedKeyVal} package.  This facility is still under development,
but some primitive expressions can be evaluated in the current release.

Suppose your program requires several parameters \verb|x1|, \verb|x2|,
and \verb|x3|.  Furthermore, suppose that their ratios remain fixed for
all the runs of the program that you desire.  It would be best to
specify some scale factor in the input that would be the only thing that
has to be changed from run to run.  If you don't want to or cannot
modify the program, then this can be done directly with
\clsnmref{ParsedKeyVal} as follows
\begin{verbatim}
scale = 1.234
x1 = ( $:scale *  1.2 )
x2 = ( $:scale *  9.2 )
x3 = ( $:scale * -2.0 )
\end{verbatim}
So we see that to the right of the ``\verb|=|'' the characters
``\verb|(|'' and ``\verb|)|'' are the expression construction operators.
This is in contrast to their function when they are to the left of the
``\verb|=|'', where they are the keyword grouping operators.

Currently, the expression must be binary and the data is all converted
to double.  If you use the expression construction operators to produce
data that the program expects to be integer, you will certainly get the
wrong answers (unless the desired value happens to be zero).

\subsection{Objects}

An instance of an object can be can be specified by surrounding it's
classname with the ``\verb|<|'' and ``\verb|>|'' operators immediately
after the keyword naming the data.

A pointer to a single object can be associated with multiple keywords by
using
\hyperref{value substitution}
         {value substitution (see Section~}
         {)}
         {ParsedKeyVal:valsub}.
This is accomplished by holding references to all objects once they are
read in.

Consider the following input for a linked list, where \clsnm{A}
\begin{verbatim}
a1<A>: (
    next<A>: (
        next<B>: (
            bdata = 4
            next<A>:()
            )
        )
    )
a2 = $:a
\end{verbatim}

The \verb|a1| list would contain two \verb|A| objects followed by a
\verb|B| object followed by another \verb|A| object.  The \verb|a2| list
refers to exactly the same object as \verb|a1| (not a copy of
\verb|a1|).



\section{Sample Input Files}

The easiest way to get started with \exenm{mpqc} is to start with
one of sample inputs that most nearly matches your problem.  All
of the samples inputs shown here can be found in the directory
\filnm{src/bin/mpqc/samples}.

\subsection{SCF Energy}
\index{SCF!example}
\index{examples!SCF}

The following input will compute the SCF energy of water.

\input{scf.in.tex}

\subsection{MP2 Energy}
\index{MP2!example}
\index{examples!MP2}

The following input will compute the MP2 energy of water.

\input{mp2.in.tex}

\subsection{SCF Optimization}
\index{optimization!example}
\index{examples!optimization}

The following input will optimize the geometry of water using
the quasi-newton method.

\input{scfopt.in.tex}

\subsection{Optimization with a Computed Guess Hessian}
\index{optimization!example}
\index{examples!optimization}

The following input will optimize the geometry of water using
the quasi-newton method.  The guess hessian will be computed
at a lower level of theory.

\input{scfoptguesshess.in.tex}

\subsection{Optimization Using Newton's Method}
\index{optimization!example}
\index{examples!optimization}

The following input will optimize the geometry of water using the Newton's
method.  The hessian will be computed at each step in the optimization.
However, hessian recomputation is usually not worth the cost; try using the
computed hessian as a guess hessian for a quasi-Newton method before
resorting to a Newton optimization.

\input{newton.in.tex}

\subsection{SCF Frequencies}
\index{frequencies!example}
\index{examples!frequencies}

The following input will compute SCF frequencies by finite
displacements.  A thermodynamic analysis will also be
performed.  If optimization input is also provided, then the
optimization will be run first, then the frequencies.

\input{scffreq.in.tex}

\subsection{Giving Coordinates and a Guess Hessian}
\index{fixed coordinates!example}
\index{examples!fixed coordinates}
\index{guess hessian!example}
\index{examples!guess hessian}
\index{coordinates!example}
\index{examples!coordinates}

The following example shows several features that are really independent.
The variable coordinates are explicitly given, rather than generated
automatically.  This is especially useful when a guess hessian is to be
provided, as it is here.  This hessian, as given by the user, is not
complete and the \clsnmref{QNewtonOpt} object will fill in the missing
values using a guess the hessian provided by the \clsnmref{MolecularEnergy}
object.  Also, fixed coordinates are given in this sample input.

\input{mancoor.in.tex}

\subsection{Optimization with a Hydrogen Bond}
\index{hydrogen bonds!example}
\index{examples!hydrogen bonds}

The automatic internal coordinate generator will fail if it cannot find
enough redundant internal coordinates.  In this case, the internal
coordinate generator must be explicitly created in the input and given
extra connectivity information, as is shown below.

\input{hbondopt.in.tex}

\subsection{Fixed Coordinate Optimization}
\label{fixedexample}
\index{fixed coordinates!example}
\index{frozen coordinates!example}
\index{examples!fixed coordinates}

 This example shows how to selectively fix internal coordinates in an
optimization.  Any number of linearly independent coordinates can be given.
These coordinates must remain linearly independent throughout the
optimization, a condition that might not hold since the coordinates can be
nonlinear.

 By default, the initial fixed coordinates' values are taken from the
cartesian geometry given by the \clsnmref{Molecule} object; however, the
molecule will be displaced to the internal coordinate values given with the
fixed internal coordinates if \keywd{have\_fixed\_values} keyword is set to
true, as shown in this example.  In this case, the initial cartesian
geometry should be reasonably close to the desired initial geometry and all
of the variable coordinates will be frozen to their original values during
the initial displacement.

\input{fixed.in.tex}

\subsection{Transition State Optimization}
\label{tsexample}
\index{transition state!example}
\index{examples!transition state}

This example shows a transition state optimization of the N-inversion in
$\mathrm{CH}_3\mathrm{NH}_2$ using mode following.  The initial geometry
was obtained by doing a few fixed coordinate optimizations along the
inversion coordinate.

\input{ts.in.tex}

\subsection{Transition State Optimization with a Computed Guess Hessian}
\index{transition state!example}
\index{examples!transition state}

This example shows a transition state optimization of the N-inversion in
$\mathrm{CH}_3\mathrm{NH}_2$ using mode following.  The initial geometry
was obtained by doing a few fixed coordinate optimizations along the
inversion coordinate.  An approximate guess hessian will be computed, which
makes the optimiziation converge much faster in this case.

\input{tsguesshess.in.tex}



\chapter{Running MPQC}
\index{running MPQC}

MPQC runs on Linux workstations, Silicon Graphics computers,
and more.

\section{Command Line Options}

\section{Environmental Variables}

By default, MPQC tries to find basis set library files in
the source code distribution.  If the executable or source
code is moved, MPQC can be notified with the environmental
variable \verb|SCLIBDIR|.  In the source code distribution
this directory is \verb|SC/lib|.

For example if you need to run MPQC on a machine that doesn't
have the source code distribution in the same place as the
machine on which MPQC is compiled you must do the following
on the machine with the source code:

\begin{alltt}
cd SC/lib
tar cvf ../sclib.tar .
\end{alltt}

Then transfer \verb|sclib.tar| to the machine that you want to run
MPQC on and do something like

\begin{alltt}
mkdir ~/sclib
cd ~/sclib
tar xvf ../sclib.tar
setenv SCLIBDIR ~/sclib
\end{alltt}

The setenv command is specific to the C-shell.  You will need to
do what is appropriate for your shell.

Now MPQC should be ready to run.

\section{Running MPQC on a Shared Memory Multiprocessor}
\index{computers, shared memory}
\index{shared memory}

By default, MPQC will run on only one CPU.  To specify more, associate
a \clsnmref{ShmMessageGrp} object with the \verb|message|
keyword.  For example, putting the following at the top of your input file
would run MPQC on four processors:

\begin{alltt}
message<\clsnmref{ShmMessageGrp}>: n = 4
\end{alltt}


To run MPQC, type:
\begin{alltt}
{\itshape mpqc_executable} {\itshape input_file}
\end{alltt}

If MPQC should unexpectedly die, shared memory segments and
semaphores will be left on the machine.  These should be promptly
cleaned up or other jobs may be prevented from running.  To
see if you have any of these resources allocated, use the
\verb|ipcs| command.  The output will look something
like:

\begin{alltt}
IPC status from /dev/kmem as of Wed Mar 13 14:42:18 1996
T     ID     KEY        MODE       OWNER    GROUP
Message Queues:
Shared Memory:
m 288800 0x00000000 --rw-------  cljanss     user
Semaphores:
s    390 0x00000000 --ra-------  cljanss     user
s    391 0x00000000 --ra-------  cljanss     user
\end{alltt}

To remove the IPC resources used by \verb|cljanss| in
the above example, type:

\begin{alltt}
ipcrm -m 288800
ipcrm -s 390
ipcrm -s 391
\end{alltt}

\section{Running MPQC on the Intel Paragon Running OSF}
\index{computers, Intel Paragon}
\index{Paragon}

To run interactively type:
\begin{alltt}
{\itshape mpqc_executable} -sz {\itshape n_node} {\itshape input_file}
\end{alltt}

\section{Running MPQC on the IBM SP2}
\index{computers, IBM SP2}
\index{SP2}

The code is currently working on the
\htmladdnormallink{Maui SP2}{http://www.mhpcc.edu/}.
They have \htmladdnormallink{extensive information}{http://www.mhpcc.edu/}
on how to run on their SP2.

\subsection{Running Interactively}

The job is controlled by a group of environment variables.
Here are my current settings:
\begin{verbatim}
MP_CSS_INTERRUPT=YES
MP_HOSTFILE=NULL
MP_EUILIB=us
MP_EUIDEVICE=css0
MP_RMPOOL=0
MP_RESD=YES
MP_LABELIO=yes
MP_INFOLEVEL=1
MP_PGMMODEL=spmd
MP_PROCS=4
\end{verbatim}
The last variable, \verb|MP_NPROCS|, sets the number
of processors that will be used.

The program is run with the command:
\begin{alltt}
poe {\itshape executable} {\itshape inputfile} -messagegrp MPIMessageGrp
\end{alltt}

\subsection{Using Load Leveler}

A load leveler script is needed to submit a command to the batch queue.
This specifies the queue (called class) and constraints on memory the
number of processors, etc.  The following template is a good start for
MPQC:

\begin{alltt}
#!/bin/sh
#@ environment = MessageGrp=MPIMessageGrp;MP_LANG=En_US;MP_LABELIO=YES;MP_INFOLEVEL=1;MP_PGMMODEL=spmd;MP_RESD=YES;MP_CSS_INTERRUPT=YES;MP_EUILIB=us
#@ input = /dev/null
#@ output = mpqc.out.$(Cluster).$(Process)
#@ error = mpqc.err.$(Cluster).$(Process)
#@ class = {\itshape CLASS}
#@ job_type = parallel
#@ min_processors = {\itshape MINPROC}
#@ max_processors = {\itshape MAXPROC}
#@ requirements =  (Adapter == "hps_user" && Memory >= {\itshape MEMORY})
#@ notification = complete
#@ notify_user = {\itshape EMAIL}
#@ shell = /bin/sh
#@ cpu_limit = {\itshape CPULIMIT}
#@ queue
/usr/lpp/poe/bin/poe {\itshape EXECUTABLE} {\itshape INPUTFILE}
\end{alltt}

To make the above useful, the italicized variables must be replaced.
Table~\ref{running:spvariables} explains the meaning of the variables:

\begin{table}
\caption{Variables Meanings for Sample IBM SP2 Loadleveler Script.}
\begin{center}
\begin{tabular}{lp{2.5in}c}
  \multicolumn{1}{c}{Variable}
     & \multicolumn{1}{c}{Meaning}
     & \multicolumn{1}{c}{Example} \\
  CLASS & queue name & small\_short \\
  MINPROC & minimum number of nodes & 4 \\
  MAXPROC & maximum number of nodes & 8 \\
  EMAIL & your email address & cljanss@ca.sandia.gov \\
  CPULIMIT & maximum CPU time (hour:min:sec) & 1:0:0 \\
  MEMORY & maximum amount of memory (MBytes) & 64 \\
  EXECUTABLE & path to the executable & mpqcic \\
  INPUTFILE & the input file name & mpqc.in \\
\end{tabular}
\end{center}
\label{running:spvariables}
\end{table}

The modified script is submitted with the following command:

\begin{alltt}
llsubmit {\itshape scriptname}
\end{alltt}



\chapter{Utility Classes}
\label{firstclasschapt}


\clssection{Debugger}
\index{Debugger}

The \clsnm{Debugger} class describes what should be done when
a catastrophic error causes unexpected program termination.
It can try things such as run start a debugger running where
the program died or it can attempt to produce a stack traceback
showing roughly where the program died.  These attempts could
just complicate matters more if the program has gotten into
a peculiar enough state.

Table~\ref{debugger:keyval} shows the \clsnm{KeyVal} input parameters
for \clsnm{Debugger}.

\begin{table}
\caption{\clsnmref{KeyVal} Input for \clsnm{Debugger}}
\begin{center}
\begin{tabular}{lp{2.5in}p{1in}}
  \multicolumn{1}{c}{Keyword}
     & \multicolumn{1}{c}{Meaning}
     & \multicolumn{1}{c}{Default} \\
 \verb|debug|
        & Try to start a debugger when an error occurs.  Doesn't
        work on all machines.
        & \verb|yes|, if possible \\
 \verb|traceback|
        & Try to print out a traceback extracting return addresses
        from the call stack.  Doesn't work on most machines.
        & \verb|yes|, if possible \\
 \verb|exit|
        & Exit on errors.
        & \verb|yes| \\
 \verb|wait_for_debugger|
        & When starting a debugger go into an infinite loop
        to give the debugger a change to attach to the process.
        & \verb|yes| \\
 \verb|handle_defaults|
        & Handle a standard set of signals such as SIGBUS, SIGSEGV, etc.
        & \verb|yes| \\
 \verb|prefix|
        & Gives a string that is printed before each line
        that is printed by \clsnm{Debugger}.
        & empty \\
 \verb|cmd|
        & Gives a command to be executed to start the debugger.
        & varies with machine
\end{tabular}
\end{center}
\label{debugger:keyval}
\end{table}


%%%%%%%%%%%%%%%%%%%%%%%%%%%%%%%%%%%%%%%%%%%%%%%%%%%%%%%%%%%%%%%%%%%%%%%%%%

\section{The \clsnm{MessageGrp} Class}
\label{MessageGrp}\index{MessageGrp}

The \clsnm{MessageGrp} abstract class provides
a mechanism for moving data and objects between
nodes in a parallel machine.

%%%%%%%%%%%%%%%%%%%%%%%%%%%%%%%%%%%%%%%%%%%%%%%%%%%%%%%%%%%%%%%%%%%%%%%%%%

\section{The \clsnm{ProcMessageGrp} Class}
\label{ProcMessageGrp}\index{ProcMessageGrp}
\index{MessageGrp!ProcMessageGrp}

The \clsnm{ProcMessageGrp} class provides
a simple single processor implementation of
the \clsnmref{MessageGrp} class.

%%%%%%%%%%%%%%%%%%%%%%%%%%%%%%%%%%%%%%%%%%%%%%%%%%%%%%%%%%%%%%%%%%%%%%%%%%

\section{The \clsnm{ShmMessageGrp} Class}
\label{ShmMessageGrp}\index{ShmMessageGrp}
\index{MessageGrp!ShmMessageGrp}

The \clsnm{ShmMessageGrp} class is an implementation of
\clsnmref{MessageGrp} that allows multiple process to be
started that communication with shared memory.  This
only provides improved performance if you have multiple
CPU's in a symmetric multiprocessor configuration.  Nonetheless,
it is quite useful on a single CPU for tracking down bugs.

The \clsnm{ShmMessageGrp} \clsnmref{KeyVal} constructor takes
a single argument that specifies the number of processors.
Here is an example of a \clsnmref{ParsedKeyVal} input that
creates a \clsnm{ShmMessageGrp} that runs on four processors.
\begin{alltt}
message<\clsnm{ShmMessageGrp}>: n = 4
\end{alltt}

%%%%%%%%%%%%%%%%%%%%%%%%%%%%%%%%%%%%%%%%%%%%%%%%%%%%%%%%%%%%%%%%%%%%%%%%%%

\section{The \clsnm{MemoryGrp} Class}
\label{MemoryGrp}\index{MemoryGrp}


\chapter{Math Classes}


%%%%%%%%%%%%%%%%%%%%%%%%%%%%%%%%%%%%%%%%%%%%%%%%%%%%%%%%%%%%%%%%%%%%%%%%%%

\section{The \clsnm{SCMatrixKit} Class}
\label{SCMatrixKit}\index{SCMatrixKit}

The \clsnm{SCMatrixKit} abstract class acts as a factory for producing
matrices.  By using one of these, the program makes sure that all of the
matrices are consistent.

%%%%%%%%%%%%%%%%%%%%%%%%%%%%%%%%%%%%%%%%%%%%%%%%%%%%%%%%%%%%%%%%%%%%%%%%%%

\section{The \clsnm{LocalSCMatrixKit} Class}
\label{LocalSCMatrixKit}\index{LocalSCMatrixKit}

The \clsnm{LocalSCMatrixKit} produces matrices that work in a
single processor environment.

%%%%%%%%%%%%%%%%%%%%%%%%%%%%%%%%%%%%%%%%%%%%%%%%%%%%%%%%%%%%%%%%%%%%%%%%%%

\section{The \clsnm{ReplSCMatrixKit} Class}
\label{ReplSCMatrixKit}\index{ReplSCMatrixKit}

The \clsnm{ReplSCMatrixKit} produces matrices that work in a
many processor environment.  A copy of the entire matrix is stored
on each node.

%%%%%%%%%%%%%%%%%%%%%%%%%%%%%%%%%%%%%%%%%%%%%%%%%%%%%%%%%%%%%%%%%%%%%%%%%%

\section{The \clsnm{DistSCMatrixKit} Class}
\label{DistSCMatrixKit}\index{DistSCMatrixKit}

The \clsnm{DistSCMatrixKit} produces matrices that work in a many processor
environment.  The matrix is distributed across all nodes.  Some debugging
may be required with this specialization.

%%%%%%%%%%%%%%%%%%%%%%%%%%%%%%%%%%%%%%%%%%%%%%%%%%%%%%%%%%%%%%%%%%%%%%%%%%

\section{The \clsnm{SCDimension} Class}
\label{SCDimension}\index{SCDimension}

\begin{description}
  \item[\keywd{n}] This gives size of the dimension.  One of \keywd{n} or
        \keywd{blocks} is required.

  \item[\keywd{blocks}] The block information for the dimension can be
        given as a \clsnmref{SCBlockInfo} object.  One of \keywd{n} or
        \keywd{blocks} is required.

\end{description}

%%%%%%%%%%%%%%%%%%%%%%%%%%%%%%%%%%%%%%%%%%%%%%%%%%%%%%%%%%%%%%%%%%%%%%%%%%

\section{The \clsnm{SCBlockInfo} Class}
\label{SCBlockInfo}\index{SCBlockInfo}

\begin{description}
  \item[\keywd{sizes}]  This is a vector giving the size of each
        subblock.  There is no default.

  \item[\keywd{subdims}] If this vector is given there is must be entry for
        each entry in the \keywd{sizes} vector.  Each entry is an
        \clsnmref{SCDimension} object.  The default is to not store
        subdimension information.

\end{description}


%%%%%%%%%%%%%%%%%%%%%%%%%%%%%%%%%%%%%%%%%%%%%%%%%%%%%%%%%%%%%%%%%%%%%%%%%%

\section{The \clsnm{Function} Class}\label{Function}\index{Function}

The \clsnm{Function} abstract class computes the value and
derivatives of a function for a given set of input
parameters.  Its \clsnmref{KeyVal} constructor reads the
following information:

\begin{description}
  \item[\keywd{matrixkit}] Gives a \clsnmref{SCMatrixKit} object.
    If it is not specified, a default \clsnmref{SCMatrixKit}
    is selected.

  \item[\keywd{value\_accuracy}] Sets the accuracy to which values are
    computed.  The default is the machine accuracy.

  \item[\keywd{gradient\_accuracy}] Sets the accuracy to which gradients
    are computed.  The default is the machine accuracy.

  \item[\keywd{hessian\_accuracy}] Sets the accuracy to which hessians are
    computed.  The default is the machine accuracy.

\end{description}

%%%%%%%%%%%%%%%%%%%%%%%%%%%%%%%%%%%%%%%%%%%%%%%%%%%%%%%%%%%%%%%%%%%%%%%%%%

\section{The \clsnm{Optimize} Class}\label{Optimize}\index{Optimize}

The \clsnm{Optimize} abstract class find stationary points for the values
of \clsnmref{Function} objects.  Its \clsnmref{KeyVal} constructor reads
the following information:

\begin{description}
  \item[\keywd{checkpoint}] If true, the optimization will be checkpointed.
     The default is false.

  \item[\keywd{checkpoint\_file}] The name of the checkpoint file.
     The default is \filnm{opt\_ckpt.dat}.

  \item[\keywd{max\_iterations}] The maximum number of interations.
     The default is 10.

  \item[\keywd{max\_stepsize}] The maximum stepsize.  The default is 0.6.

  \item[\keywd{function}] A \clsnmref{Function} object.  There is
     no default.

  \item[\keywd{convergence}] This can be either a floating point number
     or a \clsnmref{Convergence} object.  If it is a floating point
     number then it is the convergence criterion.  See the description
     \clsnmref{Convergence} class for the default.

\end{description}

%%%%%%%%%%%%%%%%%%%%%%%%%%%%%%%%%%%%%%%%%%%%%%%%%%%%%%%%%%%%%%%%%%%%%%%%%%

\section{The \clsnm{Convergence} Class}\label{Convergence}\index{Convergence}

The \clsnm{Convergence} class is used by the optimizer to determine
when an optimization is converged.  The \clsnmref{KeyVal} input for
\clsnm{Convergence} is given below.  Giving none of these keywords 
is the same as giving the following input:
\begin{alltt}
  conv<\clsnm{Convergence}>: (
    max_disp = 1.0e-6
    max_grad = 1.0e-6
    graddisp = 1.0e-6
  )
\end{alltt}

\begin{description}
  \item[\keywd{max\_disp}] The value of the maximum displacement must be
     less then the value of this keyword for the calculation to be
     converged.  The default is to not check this parameter.  However, if
     no other keyword are given, default convergence parameters are chosen
     as described above.

  \item[\keywd{max\_grad}] The value of the maximum gradient must be less
     then the value of this keyword for the calculation to be converged.
     The default is to not check this parameter.  However, if no other
     keyword are given, default convergence parameters are chosen as
     described above.

  \item[\keywd{rms\_disp}] The value of the RMS of the displacements must
     be less then the value of this keyword for the calculation to be
     converged.  The default is to not check this parameter.  However, if
     no other keyword are given, default convergence parameters are chosen
     as described above.

  \item[\keywd{rms\_grad}] The value of the RMS of the gradients must be
     less then the value of this keyword for the calculation to be
     converged.  The default is to not check this parameter.  However, if
     no other keyword are given, default convergence parameters are chosen
     as described above.

  \item[\keywd{graddisp}] The value of the scalar product of the gradient
     vector with the displacement vector must be less then the value of
     this keyword for the calculation to be converged.  The default is to
     not check this parameter.  However, if no other keyword are given,
     default convergence parameters are chosen as described above.

\end{description}

%%%%%%%%%%%%%%%%%%%%%%%%%%%%%%%%%%%%%%%%%%%%%%%%%%%%%%%%%%%%%%%%%%%%%%%%%%

\section{The \clsnm{QNewtonOpt} Class}\label{QNewtonOpt}\index{QNewtonOpt}

The \clsnm{QNewtonOpt} class derives from \clsnmref{Optimize}.  It
implements a quasi-Newton optimization scheme.

\begin{description}
  \item[\keywd{update}] This gives a \clsnmref{HessianUpdate} object.
     The default is to not update the hessian.

  \item[\keywd{hessian}] By default, the guess hessian is obtained from the
     \clsnmref{Function} object.  This keyword specifies an lower triangle
     array (the second index must be less than or equal to than the first)
     that replaces the guess hessian.  If some of the elements are not
     given, elements from the guess hessian will be used.

  \item[\keywd{lineopt}] This gives a \clsnmref{LineOpt} object for doing
     line optimizations in the Newton direction.  The default is to skip
     the line optimizations.

  \item[\keywd{accuracy}] The accuracy with which the first gradient will
     be computed.  If this is two large, it may be necessary to evaluate
     the first gradient point twice.  If it is two small, it may take
     longer to evaluate the first point. The default is 0.0001.

  \item[\keywd{print\_x}] If true, print the coordinates each iteration.
     The default is false.

  \item[\keywd{print\_gradient}] If true, print the gradient each
    iteration. The default is false.

  \item[\keywd{print\_hessian}] If true, print the approximate hessian each
    iteration. The default is false.

\end{description}

%%%%%%%%%%%%%%%%%%%%%%%%%%%%%%%%%%%%%%%%%%%%%%%%%%%%%%%%%%%%%%%%%%%%%%%%%%

\section{The \clsnm{EFCOpt} Class}\label{EFCOpt}\index{EFCOpt}

The \clsnm{EFCOpt} class derives from \clsnmref{Optimize}.  It implements
eigenvector following as described by Baker in J. Comput. Chem., Vol 7, No
4, 385-395, 1986.

\begin{description}
  \item[\keywd{update}] This gives an \clsnmref{HessianUpdate} object.
     The default is to not update the hessian.

  \item[\keywd{transition\_state}] If this is true than a transition
     state search will be performed. The default is false.

  \item[\keywd{mode\_following}] The default is false.

  \item[\keywd{hessian}] By default, the guess hessian is obtained from the
     \clsnmref{Function} object.  This keyword specifies an lower triangle
     array (the second index must be less than or equal to than the first)
     that replaces the guess hessian.  If some of the elements are not
     given, elements from the guess hessian will be used.

  \item[\keywd{accuracy}] The accuracy with which the first gradient will
     be computed.  If this is two large, it may be necessary to evaluate
     the first gradient point twice.  If it is two small, it may take
     longer to evaluate the first point. The default is 0.0001.

\end{description}

%%%%%%%%%%%%%%%%%%%%%%%%%%%%%%%%%%%%%%%%%%%%%%%%%%%%%%%%%%%%%%%%%%%%%%%%%%

\section{The \clsnm{HessianUpdate} Class}\label{HessianUpdate}\index{HessianUpdate}

The \clsnm{HessianUpdate} abstract class is used to specify a
hessian update scheme.  It is used, for example, by
\clsnmref{QNewtonOpt} objects.

%%%%%%%%%%%%%%%%%%%%%%%%%%%%%%%%%%%%%%%%%%%%%%%%%%%%%%%%%%%%%%%%%%%%%%%%%%

\section{The \clsnm{DFPUpdate} Class}\label{DFPUpdate}\index{DFPUpdate}

The \clsnm{DFPUpdate} class is derived from \clsnmref{HessianUpdate} and
used to specify a Davidson, Fletcher, and Powell hessian update scheme.

\begin{description}
  \item[\keywd{xprev}] The previous coordinates can be given (but is not
    recommended).  The default is none.

  \item[\keywd{gprev}] The previous gradient can be given (but is not
    recommended).  The default is none.

\end{description}

%%%%%%%%%%%%%%%%%%%%%%%%%%%%%%%%%%%%%%%%%%%%%%%%%%%%%%%%%%%%%%%%%%%%%%%%%%

\section{The \clsnm{BFGSUpdate} Class}\label{BFGSUpdate}\index{BFGSUpdate}

The \clsnm{DFPUpdate} class is derived from \clsnmref{DFPUpdate} and used
to specify a Broyden, Fletcher, Goldfarb, and Shanno hessian update scheme.
This hessian update method is the recommended method for use with
\clsnmref{QNewtonOpt} objects.

%%%%%%%%%%%%%%%%%%%%%%%%%%%%%%%%%%%%%%%%%%%%%%%%%%%%%%%%%%%%%%%%%%%%%%%%%%

\section{The \clsnm{PowellUpdate} Class}\label{PowellUpdate}\index{PowellUpdate}

The \clsnm{PowellUpdate} class is derived from \clsnmref{HessianUpdate}
used to specify a Powell hessian update.

%%%%%%%%%%%%%%%%%%%%%%%%%%%%%%%%%%%%%%%%%%%%%%%%%%%%%%%%%%%%%%%%%%%%%%%%%%

\section{The \clsnm{LineOpt} Class}\label{LineOpt}\index{LineOpt}

The \clsnm{LineOpt} abstract class derives from \clsnmref{Optimize} and is
used to perform one dimensional optimizations.  However, there are
currently no implementations.


\section{The \clsnm{PointGroup} Class}\label{PointGroup}\index{PointGroup}

The \clsnm{PointGroup} \clsnmref{KeyVal} constructor looks
for three keywords: \srccd{symmetry},
\srccd{symmetry\_frame}, and
\srccd{origin}. \srccd{symmetry} is a string containing the
Schoenflies symbol of the point group.  \srccd{origin} is an
array of doubles which gives the x, y, and z coordinates of
the origin of the symmetry frame.  \srccd{symmetry\_frame}
is a 3 by 3 array of arrays of doubles which specify the
principal axes for the transformation matrices as a unitary
rotation.

For example, a simple input which will use the default
\srccd{origin} and \srccd{symmetry\_frame} ((0,0,0) and the
unit matrix, respectively), might look like this:

\begin{alltt}
pointgrp<\clsnmref{PointGroup}>: (
  symmetry = "c2v"
)
\end{alltt}

By default, the principal rotation axis is taken to be the z
axis.  If you already have a set of coordinates which assume
that the rotation axis is the x axis, then you'll have to
rotate your frame of reference with \srccd{symmetry\_frame}:

\begin{alltt}
pointgrp<\clsnmref{PointGroup}>: (
  symmetry = "c2v"
  symmetry_frame = [
    [ 0 0 1 ]
    [ 0 1 0 ]
    [ 1 0 0 ]
  ]
)
\end{alltt}


\chapter{Chemistry Classes}
\label{lastclasschapt}


%%%%%%%%%%%%%%%%%%%%%%%%%%%%%%%%%%%%%%%%%%%%%%%%%%%%%%%%%%%%%%%%%%%%%%%%%%

\section{The \clsnm{Molecule} Class}\label{Molecule}\index{Molecule}

The \clsnm{Molecule} class contains information about molecules.  It has a
\clsnmref{KeyVal} constructor that can create a new molecule from either a
PDB file or from a list of Cartesian coordinates.

The following \clsnmref{ParsedKeyVal} input reads from the PDB
file \verb|h2o.pdb|:
\begin{alltt}
molecule<\clsnmref{Molecule}>: (
   pdb_file = "h2o.pdb"
 )
\end{alltt}

The following input explicitly gives the atom coordinates, using the
\clsnmref{ParsedKeyVal} \htmlref{table}{pkvtable} notation:
\begin{alltt}
molecule<\clsnmref{Molecule}>: (
    unit=angstrom
    \{ atom_labels atoms           geometry            \} = \{
          O1         O   [ 0.000000000 0  0.369372944 ]
          H1         H   [ 0.783975899 0 -0.184686472 ]
          H2         H   [-0.783975899 0 -0.184686472 ]
     \}
    )
  )
\end{alltt}
The default units are Bohr with can be overridden with
\verb|unit=angstrom|.  The \verb|atom_labels| array can be
omitted.  The \verb|atoms| and \verb|geometry| arrays
are required.

The \clsnmref{Molecule} class has a \clsnmref{PointGroup}
member object, which also has a \clsnmref{KeyVal} constructor
that is called when a \clsnmref{Molecule} is made.  The
following example constructs a molecule with $C_{2v}$ symmetry:
\begin{alltt}
molecule<\clsnmref{Molecule}>: (
    symmetry=c2v
    unit=angstrom
    \{ atoms         geometry            \} = \{
        O   [0.000000000 0  0.369372944 ]
        H   [0.783975899 0 -0.184686472 ]
     \}
    )
  )
\end{alltt}
Only the symmetry unique atoms need can be specified.  Nonunique
atoms can be given too, however, numerical errors in the
geometry specification can result in the generation of extra
atoms so be careful.

%%%%%%%%%%%%%%%%%%%%%%%%%%%%%%%%%%%%%%%%%%%%%%%%%%%%%%%%%%%%%%%%%%%%%%%%%%

\section{The \clsnm{MolecularEnergy} Class}\label{MolecularEnergy}\index{MolecularEnergy}

The \clsnm{MolecularEnergy} abstract class inherits from the
\clsnmref{Function} class.  It computes the energy of the
molecule as a function of the geometry.  The coordinate system
used can be either internal or cartesian.

\begin{description}
  \item[\keywd{molecule}] A \clsnmref{Molecule}
    \htmlref{object}{pkvobject}.  There is no default.

  \item[\keywd{coor}] A \clsnmref{MolecularCoor}
    \htmlref{object}{pkvobject} that describes the coordinates.  If this is
    not given cartesian coordinates will be used.  For convenience, two
    keywords needed by the \clsnmref{MolecularCoor} object are
    automatically provided: \keywd{natom3} and \keywd{matrixkit}.

  \item[\keywd{value\_accuracy}] Sets the accuracy to which values are
    computed.  The default is 1.0e-6 atomic units.

  \item[\keywd{gradient\_accuracy}] Sets the accuracy to which gradients
    are computed.  The default is 1.0e-6 atomic units.

  \item[\keywd{hessian\_accuracy}] Sets the accuracy to which hessians are
    computed.  The default is 1.0e-4 atomic units.

  \item[\keywd{print\_molecule\_when\_changed}] If true, then whenever the
    molecule's coordinates are updated they will be printed.  The default
    is true.
\end{description}

%%%%%%%%%%%%%%%%%%%%%%%%%%%%%%%%%%%%%%%%%%%%%%%%%%%%%%%%%%%%%%%%%%%%%%%%%%

\section{The \clsnm{MolecularCoor} Class}\label{MolecularCoor}\index{MolecularCoor}

The \clsnm{MolecularCoor} abstract class describes the coordinate system
used to describe a molecule.  It is used to convert a molecule's cartesian
coordinates to and from this coordinate system.

\begin{description}
  \item[\keywd{molecule}] A \clsnmref{Molecule}
    \htmlref{object}{pkvobject}.  There is no default.

  \item[\keywd{debug}] An integer which, if nonzero, will cause extra
    output.

  \item[\keywd{matrixkit}] A \clsnmref{SCMatrixKit}
    \htmlref{object}{pkvobject}.  It is usually unnecessary to give this
    keyword.

  \item[\keywd{natom3}] An \clsnmref{SCDimension}
    \htmlref{object}{pkvobject} for the dimension of the vector of
    cartesian coordinates.  It is usually unnecessary to give this keyword.
\end{description}

%%%%%%%%%%%%%%%%%%%%%%%%%%%%%%%%%%%%%%%%%%%%%%%%%%%%%%%%%%%%%%%%%%%%%%%%%%

\section{The \clsnm{IntMolecularCoor} Class}\label{IntMolecularCoor}\index{IntMolecularCoor}

The \clsnm{IntMolecularCoor} abstract class inherits from the
\clsnmref{MolecularCoor} class.  It describes a molecule's coordinates in
terms of internal coordinates.

\begin{description}
  \item[\keywd{variable}] Gives a \clsnmref{SetIntCoor}
    \htmlref{object}{pkvobject} that specifies the internal coordinates
    that can be varied. If this is not given, the variable coordinates will
    be generated.

  \item[\keywd{followed}] Gives a \clsnmref{IntCoor}
    \htmlref{object}{pkvobject} that specifies a coordinate to used as the
    first coordinate in the variable coordinate list.  The remaining
    coordinates will be automatically generated.  The default is no
    followed coordinate.  This option is usually used to set the initial
    search direction for a transition state optimization, where it is used
    in conjunction with the \keywd{mode\_following} keyword read by the
    \clsnmref{EFCOpt} class.

  \item[\keywd{fixed}] Gives a \clsnmref{SetIntCoor}
    \htmlref{object}{pkvobject} that specifies the internal coordinates
    that will be fixed.  The default is no fixed coordinates.

  \item[\keywd{have\_fixed\_values}] If true, then values for the fixed
    coordinates must be given in \keywd{fixed} and an attempt will be made
    to displace the initial geometry to the given fixed values. The default
    is false.

  \item[\keywd{extra\_bonds}] This is only read if the \keywd{generator}
     keyword is not given.  It is a \htmlref{vector}{pkvarray} of atom
     numbers, where elements $(i-1)\times 2 + 1$ and $i\times 2$ specify
     the atoms which are bound in extra bond $i$.  The \keywd{extra\_bonds}
     keyword should only be needed for weakly interacting fragments,
     otherwise all the needed bonds will be found.

  \item[\keywd{generator}] Specifies an \clsnmref{IntCoorGen}
    \htmlref{object}{pkvobject} that creates simple, redundant internal
    coordinates. If this keyword is not given, then a vector giving extra
    bonds to be added is read from \keywd{extra\_bonds} and this is used to
    create an \clsnmref{IntCoorGen} object.

  \item[\keywd{decouple\_bonds}] Automatically generated internal
    coordinates are linear combinations of possibly any mix of simple
    internal coordinates.  If \keywd{decouple\_bonds} is true, an attempt
    will be made to form some of the internal coordinates from just stretch
    simple coordinates.  The default is false.

  \item[\keywd{decouple\_bends}] This is like \keywd{decouple\_bonds}
    except it applies to the bend-like coordinates.  The default is false.

  \item[\keywd{max\_update\_disp}] The maximum displacement to be used in
    the displacement to fixed internal coordinates values.  Larger
    displacements will be broken into several smaller displacements and new
    coordinates will be formed for each of these displacments. This is only
    used when \keywd{fixed} and \keywd{have\_fixed\_values} are given.  The
    default is 0.5.

  \item[\keywd{max\_update\_steps}] The maximum number of steps permitted
    to convert internal coordinate displacements to cartesian coordinate
    displacements.  The default is 100.
               
  \item[\keywd{update\_bmat}] Displacements in internal coordinates are
     converted to a cartesian displacements interatively.  If there are
     large changes in the cartesian coordinates during conversion, then
     recompute the $B$ matrix, which is using to do the conversion.  The
     default is false.

  \item[\keywd{only\_totally\_symmetric}] If a simple test reveals that an
     internal coordinate is not totally symmetric, then it will not be
     added to the internal coordinate list.  The default is true.
               
  \item[\keywd{simple\_tolerance}] The internal coordinates are formed as
     linear combinations of simple, redundant internal coordinates.
     Coordinates with coefficients smaller then \keywd{simple\_tolerance}
     will be omitted. The default is 1.0e-3.

  \item[\keywd{cartesian\_tolerance}] The tolerance for conversion of
     internal coordinate displacements to cartesian displacements.  The
     default is 1.0e-12.
               
  \item[\keywd{form:print\_simple}] Print the simple internal coordinates.
     The default is false.
               
  \item[\keywd{form:print\_variable}] Print the variable internal
     coordinates.  The default is false.
               
  \item[\keywd{form:print\_constant}] Print the constant internal
     coordinates.  The default is false.
               
  \item[\keywd{form:print\_molecule}] Print the molecule when forming
     coordinates.  The default is false.
               
  \item[\keywd{scale\_bonds}] Obsolete.  The default value is 1.0.

  \item[\keywd{scale\_bends}] Obsolete.  The default value is 1.0.
               
  \item[\keywd{scale\_tors}] Obsolete.  The default value is 1.0.

  \item[\keywd{scale\_outs}] Obsolete.  The default value is 1.0.

  \item[\keywd{symmetry\_tolerance}] Obsolete.  The default is 1.0e-5.

  \item[\keywd{coordinate\_tolerance}] Obsolete.  The default is 1.0e-7.

\end{description}


%%%%%%%%%%%%%%%%%%%%%%%%%%%%%%%%%%%%%%%%%%%%%%%%%%%%%%%%%%%%%%%%%%%%%%%%%%

\section{The \clsnm{SymmMolecularCoor} Class}\label{SymmMolecularCoor}\index{SymmMolecularCoor}

The \clsnm{SymmMolecularCoor} class derives from
\clsnmref{IntMolecularCoor}.  It provides a unique set of totally symmetric
internal coordinates.  Giving an \clsnmref{MolecularEnergy} object a
\keywd{coor} is usually the best way to optimize a molecular structure.
However, for some classes of molecules \clsnm{SymmMolecularCoor} doesn't
work very well.  For example, enediyne can cause problems.  In these cases,
cartesian coordinates (obtained by not giving the
\clsnmref{MolecularEnergy} object the \keywd{coor} keyword) might be better
or you can manually specify the coordinates that the
\clsnm{SymmMolecularCoor} object uses with the \keywd{variable} keyword
(see the \clsnmref{IntMolecularCoor} class description).

\begin{description}
  \item[\keywd{change\_coordinates}] If true, the quality of the internal
    coordinates will be checked periodically and if they are beginning to
    become linearly dependent a new set of internal coordinates will be
    computed.  The default is false.

  \item[\keywd{max\_kappa2}] A measure of the quality of the internal
    coordinates.  Values of the 2-norm condition, $\kappa_2$, larger than
    \keywd{max\_kappa2} are considered linearly dependent.  The default is
    10.0.

  \item[\keywd{transform\_hessian}] If true, the hessian will be transformed
    every time the internal coordinates are formed.  The default is true.

\end{description}

%%%%%%%%%%%%%%%%%%%%%%%%%%%%%%%%%%%%%%%%%%%%%%%%%%%%%%%%%%%%%%%%%%%%%%%%%%

\section{The \clsnm{RedundMolecularCoor} Class}\label{RedundMolecularCoor}\index{RedundMolecularCoor}

The \clsnm{RedundMolecularCoor} class derives from
\clsnmref{IntMolecularCoor}.  It provides a redundant set of simple
internal coordinates.

%%%%%%%%%%%%%%%%%%%%%%%%%%%%%%%%%%%%%%%%%%%%%%%%%%%%%%%%%%%%%%%%%%%%%%%%%%

\section{The \clsnm{IntCoorGen} Class}\label{IntCoorGen}\index{IntCoorGen}

The \clsnm{IntCoorGen} class is used to construct a set of internal
coordinates for a molecule.

\begin{description}
  \item[\keywd{molecule}] A \clsnmref{Molecule}
    \htmlref{object}{pkvobject}.  There is no default.

  \item[\keywd{radius\_scale\_factor}] If the distance between two atoms is
      less than the radius scale factor times the sum of the atoms' atomic
      radii, then a bond is placed between the two atoms for the purpose of
      finding internal coordinates.  The default is 1.1.

  \item[\keywd{linear\_bend\_threshold}] A bend angle in degress greater
      than 180 minus this keyword's floating point value is considered a
      linear bend. The default is 5.0.

  \item[\keywd{linear\_tors\_threshold}] The angles formed by atoms a-b-c
      and b-c-d are checked for near linearity.  If an angle in degrees is
      greater than 180 minus this keyword's floating point value, then the
      torsion is classified as a linear torsion. The default is 5.0.

  \item[\keywd{linear\_bend}] Generate \clsnmref{BendSimpleCo} objects
      to describe linear bends.  The default is false.

  \item[\keywd{linear\_lbend}] Generate pairs of \clsnmref{LinIPSimpleCo}
      and \clsnmref{LinIPSimpleCo} objects to describe linear bends.  The
      default is true.

  \item[\keywd{linear\_tors}] Generate \clsnmref{TorsSimpleCo} objects
      to described linear torsions.  The default is false.

  \item[\keywd{linear\_stors}] Generate \clsnmref{ScaledTorsSimpleCo}
      objects to described linear torsions.  The default is true.


  \item[\keywd{extra\_bonds}] This is a \htmlref{vector}{pkvarray} of atom
     numbers, where elements $2 (i-1) + 1$ and $2 i$ specify the atoms
     which are bound in extra bond $i$.  The \keywd{extra\_bonds} keyword
     should only be needed for weakly interacting fragments, otherwise all
     the needed bonds will be found.

\end{description}

%%%%%%%%%%%%%%%%%%%%%%%%%%%%%%%%%%%%%%%%%%%%%%%%%%%%%%%%%%%%%%%%%%%%%%%%%%

\section{The \clsnm{IntCoor} Class}\label{IntCoor}\index{IntCoor}

The \clsnm{IntCoor} abstract class describes an internal coordinate of a
molecule.  The following keywords are recognized in the input:

\begin{description}
  \item[\keywd{label}] A label for the coordinate using only to
     identify the coordinate to the user in printouts.  The default
     is no label.

  \item[\keywd{value}] A value for the coordinate.  In the way that
     coordinates are usually used, the default is to compute a value
     from the cartesian coordinates in a \clsnmref{Molecule} object.

  \item[\keywd{unit}] The unit in which the value is given.  This can be
     \keywd{bohr}, \keywd{anstrom}, \keywd{radian}, and \keywd{degree}.
     The default is \keywd{bohr} for lengths and \keywd{radian} for angles.

\end{description}

%%%%%%%%%%%%%%%%%%%%%%%%%%%%%%%%%%%%%%%%%%%%%%%%%%%%%%%%%%%%%%%%%%%%%%%%%%

\section{The \clsnm{SimpleCo} Class}\label{SimpleCo}\index{SimpleCo}

The \clsnm{SimpleCo} abstract class describes a simple internal coordinate
of a molecule.  The number atoms involved can be 2, 3 or 4 and is
determined by the specialization of \clsnm{SimpleCo}.

There are three ways to specify the atoms involved in the internal
coordinate.  The first way is a shorthand notation, just a vector of a
label followed by the atom numbers (starting at 1) is given.  For example,
a stretch between two atoms, 1 and 2, is given, in the
\clsnmref{ParsedKeyVal} format, as
\begin{alltt}
  stretch<\clsnmref{StreSimpleCo}>: [ R12 1 2 ]
\end{alltt}

The other two ways to specify the atoms are more general.  With them, it is
possible to give parameters for the \clsnm{IntCoor} base class (and thus
give the value of the coordinate).  In the first of these input formats, a
vector associated with the keyword \keywd{atoms} gives the atom numbers.
The following specification for \keywd{stretch} is equivalent to that
above:
\begin{alltt}
  stretch<\clsnmref{StreSimpleCo}>:( label = R12 atoms = [ 1 2 ] )
\end{alltt}

In the second, a vector, \keywd{atom\_labels}, is given along with a
\clsnmref{Molecule} object.  The atom labels are looked up in the
\clsnmref{Molecule} object to find the atom numbers.
The following specification for \keywd{stretch} is equivalent to those
above:
\begin{alltt}
  molecule<\clsnmref{Molecule}>: (
    \{ atom_labels atoms   geometry      \} = \{
          H1         H   [ 1.0 0.0 0.0 ]
          H2         H   [-1.0 0.0 0.0 ] \} )
  stretch<\clsnmref{StreSimpleCo}>:( label = R12
                          atom_labels = [ H1 H2 ]
                          molecule = $molecule )
\end{alltt}

%%%%%%%%%%%%%%%%%%%%%%%%%%%%%%%%%%%%%%%%%%%%%%%%%%%%%%%%%%%%%%%%%%%%%%%%%%

\section{The \clsnm{StreSimpleCo} Class}\label{StreSimpleCo}\index{StreSimpleCo}

The \clsnm{StreSimpleCo} class describes an stretch internal coordinate of a
molecule.  The input is described in the documentation of its parent
class \clsnmref{SimpleCo}.

Designating the two atoms as $a$ and $b$ and their cartesian positions as
$\bar{r}_a$ and $\bar{r}_b$, the value of the coordinate, $r$, is
\[ r = \| \bar{r}_a - \bar{r}_b \| \]

%%%%%%%%%%%%%%%%%%%%%%%%%%%%%%%%%%%%%%%%%%%%%%%%%%%%%%%%%%%%%%%%%%%%%%%%%%

\section{The \clsnm{BendSimpleCo} Class}\label{BendSimpleCo}\index{BendSimpleCo}

The \clsnm{BendSimpleCo} class describes an bend internal coordinate of a
molecule.  The input is described in the documentation of its parent
class \clsnmref{SimpleCo}.

Designating the three atoms as $a$, $b$, and $c$ and their cartesian
positions as $\bar{r}_a$, $\bar{r}_b$, and $\bar{r}_c$, the value of the
coordinate, $\theta$, is given by
\begin{eqnarray*}
 \bar{u}_{ab} &=& \frac{\bar{r}_a - \bar{r}_b}{\| \bar{r}_a - \bar{r}_b \|}\\
 \bar{u}_{cb} &=& \frac{\bar{r}_c - \bar{r}_b}{\| \bar{r}_c - \bar{r}_b \|}\\
 \theta       &=& \arccos ( \bar{u}_{ab} \cdot \bar{u}_{cb} )
\end{eqnarray*}

%%%%%%%%%%%%%%%%%%%%%%%%%%%%%%%%%%%%%%%%%%%%%%%%%%%%%%%%%%%%%%%%%%%%%%%%%%

\section{The \clsnm{TorsSimpleCo} Class}\label{TorsSimpleCo}\index{TorsSimpleCo}

The \clsnm{TorsSimpleCo} class describes an torsion internal coordinate of a
molecule.  The input is described in the documentation of its parent
class \clsnmref{SimpleCo}.

Designating the four atoms as $a$, $b$, $c$, and $d$ and their cartesian
positions as $\bar{r}_a$, $\bar{r}_b$, $\bar{r}_c$, and $\bar{r}_d$, the
value of the coordinate, $\tau$, is given by
\begin{eqnarray*}
 \bar{u}_{ab} &=& \frac{\bar{r}_a - \bar{r}_b}{\| \bar{r}_a - \bar{r}_b \|}\\
 \bar{u}_{cb} &=& \frac{\bar{r}_c - \bar{r}_b}{\| \bar{r}_c - \bar{r}_b \|}\\
 \bar{u}_{cd} &=& \frac{\bar{r}_c - \bar{r}_d}{\| \bar{r}_c - \bar{r}_b \|}\\
 \bar{n}_{abc}&=& \frac{\bar{u}_{ab} \times \bar{u}_{cb}}
                       {\| \bar{u}_{ab} \times \bar{u}_{cb} \|} \\
 \bar{n}_{bcd}&=& \frac{\bar{u}_{cd} \times \bar{u}_{bc}}
                       {\| \bar{u}_{cd} \times \bar{u}_{bc} \|} \\
 s            &=& \cases{1, &if $(\bar{n}_{abc}\times\bar{n}_{bcd})
                                  \cdot \bar{u}_{cb} > 0$;\cr
                         -1, &otherwise.\cr}\\
 \tau       &=& s \arccos ( - \bar{n}_{abc} \cdot \bar{n}_{bcd} )
\end{eqnarray*}

%%%%%%%%%%%%%%%%%%%%%%%%%%%%%%%%%%%%%%%%%%%%%%%%%%%%%%%%%%%%%%%%%%%%%%%%%%

\section{The \clsnm{OutSimpleCo} Class}\label{OutSimpleCo}\index{OutSimpleCo}

The \clsnm{OutSimpleCo} class describes an out-of-plane internal coordinate
of a molecule.  The input is described in the documentation of its parent
class \clsnmref{SimpleCo}.

Designating the four atoms as $a$, $b$, $c$, and $d$ and their cartesian
positions as $\bar{r}_a$, $\bar{r}_b$, $\bar{r}_c$, and $\bar{r}_d$, the
value of the coordinate, $\tau$, is given by
\begin{eqnarray*}
 \bar{u}_{ab} &=& \frac{\bar{r}_a - \bar{r}_b}{\| \bar{r}_a - \bar{r}_b \|}\\
 \bar{u}_{cb} &=& \frac{\bar{r}_b - \bar{r}_c}{\| \bar{r}_c - \bar{r}_b \|}\\
 \bar{u}_{db} &=& \frac{\bar{r}_c - \bar{r}_d}{\| \bar{r}_c - \bar{r}_b \|}\\
 \bar{n}_{bcd}&=& \frac{\bar{u}_{cb} \times \bar{u}_{db}}
                       {\| \bar{u}_{cb} \times \bar{u}_{db} \|} \\
 \phi       &=& \arcsin ( \bar{u}_{ab} \cdot \bar{n}_{bcd} )
\end{eqnarray*}

%%%%%%%%%%%%%%%%%%%%%%%%%%%%%%%%%%%%%%%%%%%%%%%%%%%%%%%%%%%%%%%%%%%%%%%%%%

\section{The \clsnm{LinIPSimpleCo} Class}\label{LinIPSimpleCo}\index{LinIPSimpleCo}

The \clsnm{LinIPSimpleCo} class describes an in-plane component of a linear
bend internal coordinate of a molecule.  The input is described in the
documentation of its parent class \clsnmref{SimpleCo}.  A vector,
$\bar{u}$, given as the keyword \keywd{u}, that is not colinear with either
$\bar{r}_a - \bar{r}_b$ or $\bar{r}_b - \bar{r}_c$ must be provided, where
$\bar{r}_a$, $\bar{r}_b$, and $\bar{r}_c$ are the positions of the first,
second, and third atoms, respectively.

  Usually, \clsnmref{LinIPSimpleCo} is used with a corresponding
\clsnmref{LinOPSimpleCo}, which is given exactly the same \keywd{u}.

Designating the three atoms as $a$, $b$, and $c$ and their cartesian
positions as $\bar{r}_a$, $\bar{r}_b$, and $\bar{r}_c$, the value of the
coordinate, $\theta_i$, is given by
\begin{eqnarray*}
 \bar{u}_{ab} &=& \frac{\bar{r}_a - \bar{r}_b}{\| \bar{r}_a - \bar{r}_b \|}\\
 \bar{u}_{cb} &=& \frac{\bar{r}_b - \bar{r}_c}{\| \bar{r}_c - \bar{r}_b \|}\\
 \theta_i     &=& \pi - \arccos ( \bar{u}_{ab} \cdot \bar{u} )
                      - \arccos ( \bar{u}_{cb} \cdot \bar{u} )
\end{eqnarray*}

%%%%%%%%%%%%%%%%%%%%%%%%%%%%%%%%%%%%%%%%%%%%%%%%%%%%%%%%%%%%%%%%%%%%%%%%%%

\section{The \clsnm{LinOPSimpleCo} Class}\label{LinOPSimpleCo}\index{LinOPSimpleCo}

The \clsnm{LinOPSimpleCo} class describes an out-of-plane component of a
linear bend internal coordinate of a molecule.  The input is described in
the documentation of its parent class \clsnmref{SimpleCo}.  A vector,
$\bar{u}$, given as the keyword \keywd{u}, that is not colinear with either
$\bar{r}_a - \bar{r}_b$ or $\bar{r}_b - \bar{r}_c$ must be provided, where
$\bar{r}_a$, $\bar{r}_b$, and $\bar{r}_c$ are the positions of the first,
second, and third atoms, respectively.

  Usually, \clsnmref{LinOPSimpleCo} is used with a corresponding
\clsnmref{LinIPSimpleCo}, which is given exactly the same \keywd{u}.

Designating the three atoms as $a$, $b$, and $c$ and their cartesian
positions as $\bar{r}_a$, $\bar{r}_b$, and $\bar{r}_c$, the value of the
coordinate, $\theta_o$, is given by
\begin{eqnarray*}
 \bar{u}_{ab} &=& \frac{\bar{r}_a - \bar{r}_b}{\| \bar{r}_a - \bar{r}_b \|}\\
 \bar{u}_{cb} &=& \frac{\bar{r}_b - \bar{r}_c}{\| \bar{r}_c - \bar{r}_b \|}\\
 \bar{n}      &=& \frac{\bar{u} \times \bar{u}_{ab}}
                       {\| \bar{u} \times \bar{u}_{ab} \|}\\
 \theta_o     &=& \pi - \arccos ( \bar{u}_{ab} \cdot \bar{n} )
                      - \arccos ( \bar{u}_{cb} \cdot \bar{n} )
\end{eqnarray*}

%%%%%%%%%%%%%%%%%%%%%%%%%%%%%%%%%%%%%%%%%%%%%%%%%%%%%%%%%%%%%%%%%%%%%%%%%%

\section{The \clsnm{ScaledTorsSimpleCo} Class}\label{ScaledTorsSimpleCo}\index{ScaledTorsSimpleCo}

The \clsnm{ScaledTorsSimpleCo} class describes an scaled torsion internal
coordinate of a molecule.  The scaled torsion is more stable that ordinary
torsions (see the \clsnmref{TorsSimpleCo} class) in describing situations
where one of the torsions plane's is given by three near linear atoms.

Designating the four atoms as $a$, $b$, $c$, and $d$ and their cartesian
positions as $\bar{r}_a$, $\bar{r}_b$, $\bar{r}_c$, and $\bar{r}_d$, the
value of the coordinate, $\tau_s$, is given by
\begin{eqnarray*}
 \bar{u}_{ab} &=& \frac{\bar{r}_a - \bar{r}_b}{\| \bar{r}_a - \bar{r}_b \|}\\
 \bar{u}_{cb} &=& \frac{\bar{r}_c - \bar{r}_b}{\| \bar{r}_c - \bar{r}_b \|}\\
 \bar{u}_{cd} &=& \frac{\bar{r}_c - \bar{r}_d}{\| \bar{r}_c - \bar{r}_b \|}\\
 \bar{n}_{abc}&=& \frac{\bar{u}_{ab} \times \bar{u}_{cb}}
                       {\| \bar{u}_{ab} \times \bar{u}_{cb} \|} \\
 \bar{n}_{bcd}&=& \frac{\bar{u}_{cd} \times \bar{u}_{cb}}
                       {\| \bar{u}_{cd} \times \bar{u}_{cb} \|} \\
 s            &=& \cases{-1, &if $(\bar{n}_{abc}\times\bar{n}_{bcd})
                                  \cdot \bar{u}_{cb} > 0$;\cr
                         1, &otherwise.\cr}\\
 \tau_s       &=& s \sqrt{\left(1-(\bar{u}_{ab} \cdot \bar{u}_{cb}\right)^2)
                        \left(1-(\bar{u}_{cb} \cdot \bar{u}_{cd}\right)^2)}
                  \arccos ( - \bar{n}_{abc} \cdot \bar{n}_{bcd} )
\end{eqnarray*}

%%%%%%%%%%%%%%%%%%%%%%%%%%%%%%%%%%%%%%%%%%%%%%%%%%%%%%%%%%%%%%%%%%%%%%%%%%

\section{The \clsnm{SumIntCoor} Class}\label{SumIntCoor}\index{SumIntCoor}

The \clsnm{SumIntCoor} class specifies a linear combination of
\clsnmref{IntCoor} objects.  The following keywords are recognized:

\begin{description}
  \item[\keywd{coor}] A \htmlref{vector}{pkvarray} of \clsnmref{IntCoor}
    \htmlref{objects}{pkvobject} that define the summed coordinates.

  \item[\keywd{coef}] A \htmlref{vector}{pkvarray} of floating point
    numbers that gives the coefficients of the summed coordinates.

\end{description}

The following is a sample \clsnmref{ParsedKeyVal} input for
a \clsnmref{SumIntCoor} object.
\begin{alltt}
  sumintcoor<\clsnmref{SumIntCoor}>: (
    coor: [
      <\clsnmref{StreSimpleCo}>:( atoms = [ 1 2 ] )
      <\clsnmref{StreSimpleCo}>:( atoms = [ 2 3 ] )
      ]
    coef = [ 1.0 1.0 ]
    )
\end{alltt}

%%%%%%%%%%%%%%%%%%%%%%%%%%%%%%%%%%%%%%%%%%%%%%%%%%%%%%%%%%%%%%%%%%%%%%%%%%

\section{The \clsnm{SetIntCoor} Class}\label{SetIntCoor}\index{SetIntCoor}

The \clsnm{SetIntCoor} class describes a set of internal coordinates.  It
can automatically generate these coordinates using a integral coordinate
generator object (see the \clsnmref{IntCoorGen} class) or the internal
coordinates can be explicity given.  Internal coordinate classes include:
\clsnmref{SumIntCoor}, \clsnmref{StreSimpleCo}, \clsnmref{BendSimpleCo},
\clsnmref{LinIPSimpleCo}, \clsnmref{LinOPSimpleCo},
\clsnmref{TorsSimpleCo}, \clsnmref{ScaledTorsSimpleCo}, and
\clsnmref{OutSimpleCo}.

\begin{description}
  \item[\keywd{generator}] A \clsnmref{IntCoorGen}
    \htmlref{object}{pkvobject} that will be used to generate the internal
    coordinates.

  \item[\vrbl{i}] A sequence of integer keywords, all $i$ for $0 \leq i <
    n$, can be assigned to \clsnmref{IntCoor} objects.

\end{description}

The following is a sample \clsnmref{ParsedKeyVal} input for
a \clsnmref{SetIntCoor} object.
\begin{alltt}
  setintcoor<\clsnmref{SetIntCoor}>: [
    <\clsnmref{SumIntCoor}>: (
      coor: [
        <\clsnmref{StreSimpleCo}>:( atoms = [ 1 2 ] )
        <\clsnmref{StreSimpleCo}>:( atoms = [ 2 3 ] )
        ]
      coef = [ 1.0 1.0 ]
      )
    <\clsnmref{BendSimpleCo}>:( atoms = [ 1 2 3 ] )
  ]
\end{alltt}

%%%%%%%%%%%%%%%%%%%%%%%%%%%%%%%%%%%%%%%%%%%%%%%%%%%%%%%%%%%%%%%%%%%%%%%%%%

\section{The \clsnm{MolecularFrequencies} Class}\label{MolecularFrequencies}\index{MolecularFrequencies}

The \clsnm{MolecularFrequencies} class is used to generate vibrational
frequencies and compute thermodynamic information for a molecule.

\begin{description}
  \item[\keywd{mole}] A \clsnmref{MolecularEnergy}
    \htmlref{object}{pkvobject}.  If this is not given then
    \keywd{molecule} must be given.

  \item[\keywd{molecule}] A \clsnmref{Molecule}
    \htmlref{object}{pkvobject}.  If this is not given then \keywd{mole}
    must be given.

  \item[\keywd{point\_group}] A \clsnmref{PointGroup}
    \htmlref{object}{pkvobject}.  This is the point group used to compute
    the finite displacements.  Since some \clsnm{MolecularEnergy} objects
    cannot handle changes in the molecule's point group, the molecule must
    be given $C_1$ symmetry for frequency calculations.  In this case, the
    \keywd{point\_group} keyword can be given to reduce number of the
    displacements needed to compute the frequencies.  If this is not given
    then the point group of the molecule is used.

  \item[\keywd{debug}] An integer which, if nonzero, will cause extra
    output.

  \item[\keywd{displacement}] The amount that coordinates will be
    displaced.  The default is 0.001.

\end{description}

%%%%%%%%%%%%%%%%%%%%%%%%%%%%%%%%%%%%%%%%%%%%%%%%%%%%%%%%%%%%%%%%%%%%%%%%%%

\section{The \clsnm{MolEnergyConvergence} Class}\label{MolEnergyConvergence}\index{MolEnergyConvergence}

The \clsnm{MolEnergyConvergence} class derives from the
\clsnmref{Convergence} class.  The \clsnm{MolEnergyConvergence} class
allows the user to request that cartesian coordinates be used in evaluating
the convergence criteria.  This is useful, since the internal coordinates
can be somewhat arbitary.  If the optimization is constrained, then the
fixed internal coordinates will be projected out of the cartesian
gradients.  The input is similar to that for \clsnmref{Convergence} class
with the exception that giving none of the convergence criteria keywords is
the same as providing the following input:
\begin{alltt}
  conv<\clsnm{MolEnergyConvergence}>: (
    max_disp = 1.0e-4
    max_grad = 1.0e-4
    graddisp = 1.0e-4
  )
\end{alltt}

For \clsnm{MolEnergyConverence} to work, the \clsnmref{Function} object
given to the \clsnmref{Optimizer} object must derive from
\clsnmref{MolecularEnergy}.

The other input parameter is listed below:
\begin{description}
  \item[\keywd{cartesian}] If true, cartesian displacements and gradients
    will be compared to the convergence criteria.  The default is true.

\end{description}


%%%%%%%%%%%%%%%%%%%%%%%%%%%%%%%%%%%%%%%%%%%%%%%%%%%%%%%%%%%%%%%%%%%%%%%%%%

\section{The \clsnm{Wavefunction} Class}
\label{Wavefunction}\index{Wavefunction}

The \clsnm{Wavefunction} abstract class derives from
\clsnmref{MolecularEnergy}.  \clsnm{Wavefunction} objects read the
following input:

\begin{description}
  \item[\keywd{basis}] Specifies a \clsnmref{GaussianBasisSet} object.
    There is no default.

  \item[\keywd{integral}] Specifies an \clsnmref{Integral} object that
    computes the two electron integrals.  The default is a
    \clsnmref{IntegralV3} object.

\end{description}

%%%%%%%%%%%%%%%%%%%%%%%%%%%%%%%%%%%%%%%%%%%%%%%%%%%%%%%%%%%%%%%%%%%%%%%%%%

\section{The \clsnm{OneBodyWavefunction} Class}
\label{OneBodyWavefunction}\index{OneBodyWavefunction}

The \clsnm{OneBodyWavefunction} abstract class derives from
\clsnmref{Wavefunction}.  \clsnm{OneBodyWavefunction} objects read the
following input:

\begin{description}
  \item[\keywd{eigenvector\_accuracy}] Gives the accuracy to which
    eigenvectors are initially computed.  The default 1.0e-7.  Accuracies
    are usually adjusted as needed anyway, so it should not be necessary to
    change this.

\end{description}


%%%%%%%%%%%%%%%%%%%%%%%%%%%%%%%%%%%%%%%%%%%%%%%%%%%%%%%%%%%%%%%%%%%%%%%%%%

\section{The \clsnm{GaussianBasisSet} Class}
\label{GaussianBasisSet}\index{GaussianBasisSet}

The \clsnm{GaussianBasisSet} class is used describe a
basis set atomic composed of atomic gaussian orbitals.

\begin{description}
  \item[\keywd{molecule}] The gives a \clsnmref{Molecule} object.
     The is no default.

  \item[\keywd{puream}] If this boolean parameter is true then
     5D, 7F, etc. will be used.  Otherwise all cartesian functions
     will be used.  The default depends on the particular basis set.

  \item[\keywd{name}] This is a string giving the name of the basis set.
     Table~\ref{basissets} gives some of the recognized basis set names.
     It may be necessary to put the name in double quotes. There is no
     default.

  \item[\keywd{basis}] This is a vector of basis set names that can give a
     different basis set to each atom in the molecule.  If the
     \keywd{element} vector is given, then it gives different basis sets
     to different elements.  The default is to give every atom the
     basis set specified in \keywd{name}.

  \item[\keywd{element}] This is a vector of elements.  If it is given
     then it must have the same number of entries as the \keywd{basis}
     vector.

  \item[\keywd{basisdir}] A string giving a directory where basis
     set data files are to be sought.  See the text below for a
     complete description of what directors are consulted.

  \item[\keywd{basisfiles}] Each keyword in this vector of files is
     appended to the directory specified with \keywd{basisdir} and basis
     set data is read from them.

  \item[\keywd{matrixkit}] Specifies a \clsnmref{SCMatrixKit} object.
     It is usually not necessary to give this keyword, as the default
     action should get the correct \clsnmref{SCMatrixKit}.

\end{description}

Several files in various directories are checked for basis set data.
First, basis sets can be given by the user in the \keywd{basis} section at
the top level of the main input file.  Next, if a path is given with the
\keywd{basisdir} keyword, then all of the files given with the
\keywd{basisfiles} keyword are read in after appending their names to the
value of \keywd{basisdir}.  Basis sets can be given in these files in the
\keywd{basis} section at the top level as well.  If the named basis set
still cannot be found, then \clsnm{GaussianBasisSet} will try convert the
basis set name to a file name and check first in the directory given by
\keywd{basisdir}.  Next it checks for the environment variable
\verb|SCLIBDIR|.  If it is set it will look for the basis file in
\filnm{\$SCLIBDIR/basis}.  Otherwise it will look in the source code
distribution in the directory \filnm{SC/lib/basis}.  If the executable has
changed machines or the source code has be moved, then it may be necessary
to copy the library files to your machine and set the \verb|SCLIBDIR|
environmental variable.

The basis set itself is also given in the \clsnmref{ParsedKeyVal} format.
It is a \htmlref{vector}{pkvarray} of shells with the keyword
\keywd{:basis:} followed by the lowercase atomic name followed by \keywd{:}
followed by the basis set name (which may need to be placed inside double
quotes).  Each shell reads the following keywords:
\begin{description}
  \item[\keywd{type}] This is a \htmlref{vector}{pkvarray} that describes
    each component of this shell.  For each element the following two
    keywords are read:

  \begin{description}

    \item[\keywd{am}] The angular momentum of the component.  This can be
       given as the letter designation, \keywd{s}, \keywd{p}, \keywd{d},
       etc.  There is no default.

    \item[\keywd{puream}] If this boolean parameter is true then 5D, 7F,
       etc. shells are used.  The default is false.  This parameter can be
       overridden in the \clsnmref{GaussianBasisSet} specification.

  \end{description}

  \item[\keywd{exp}]  This is a \htmlref{vector}{pkvarray} giving the
     exponents of the primitive Gaussian functions.

  \item[\keywd{coef}] This is a \htmlref{matrix}{pkvarray} giving the
     coeffients of the primitive Gaussian functions.  The first index gives
     the component number of the shell and the second gives the primitive
     number.

\end{description}

An example might be easier to understand.  This is a basis set specificition
for STO-2G carbon:
\begin{alltt}
basis: (
 carbon: "STO-2G": [
  (type: [(am = s)]
   \{      exp      coef:0\} = \{
      27.38503303 0.43012850
       4.87452205 0.67891353
   \})
  (type: [(am = p) (am = s)]
   \{     exp      coef:1     coef:0\} = \{
       1.13674819 0.04947177 0.51154071
       0.28830936 0.96378241 0.61281990
   \})
 ]
)
\end{alltt}

\begin{table}
\caption{Available Basis Sets}
\begin{center}
\begin{tabular}{l}
  \multicolumn{1}{c}{Basis Set} \\
 \keywd{STO-2G} \\
 \keywd{STO-3G} \\
 \keywd{STO-6G} \\
 \keywd{3-21G} \\
 \keywd{3-21G*} \\
 \keywd{3-21++G} \\
 \keywd{3-21++G*} \\
 \keywd{4-31G} \\
 \keywd{4-31G*} \\
 \keywd{4-31G**} \\
 \keywd{6-31G} \\
 \keywd{6-31G*} \\
 \keywd{6-31G**} \\
 \keywd{6-31+G*} \\
 \keywd{6-31++G} \\
 \keywd{6-31++G*} \\
 \keywd{6-311G} \\
 \keywd{6-311G*} \\
 \keywd{6-311G**} \\
 \keywd{6-311++G**} \\
 \keywd{cc-pVDZ} \\
 \keywd{cc-pVTZ} \\
 \keywd{cc-pVQZ} \\
 \keywd{aug-cc-pVDZ} \\
 \keywd{aug-cc-pVTZ} \\
 \keywd{aug-cc-pVQZ} \\
\end{tabular}
\end{center}
\label{basissets}
\end{table}

%%%%%%%%%%%%%%%%%%%%%%%%%%%%%%%%%%%%%%%%%%%%%%%%%%%%%%%%%%%%%%%%%%%%%%%%%%

\section{The \clsnm{Integral} Class}
\label{Integral}\index{Integral}

The \clsnm{Integral} abstract class acts as a factory to provide
objects that compute one and two electron integrals.


%%%%%%%%%%%%%%%%%%%%%%%%%%%%%%%%%%%%%%%%%%%%%%%%%%%%%%%%%%%%%%%%%%%%%%%%%%

\section{The \clsnm{IntegralV3} Class}
\label{IntegralV3}\index{IntegralV3}

The \clsnm{IntegralV3} class derives from \clsnmref{Integral}.
It is currently the only specialization of \clsnmref{Integral}.

%%%%%%%%%%%%%%%%%%%%%%%%%%%%%%%%%%%%%%%%%%%%%%%%%%%%%%%%%%%%%%%%%%%%%%%%%%

\section{The \clsnm{SCF} Class}
\label{SCF}\index{SCF}

The \clsnm{SCF} abstract class derives from \clsnmref{OneBodyWavefunction}.
\clsnm{SCF} objects read the following input:

\begin{description}
  \item[\keywd{maxiter}]  This integer specifies the maximum number of
     SCF iterations.  The default is 40.

  \item[\keywd{density\_reset\_frequency}] This integer specifies how
     often, in term of SCF iterations, $\Delta D$ will be reset to $D$.
     The default is 10.

  \item[\keywd{reset\_occuptions}] Reassign the occupations after each
     iteration based on the eigenvalues.  This only has an effect for
     molecules with higher than $C_1$ symmetry.  The default is false.

  \item[\keywd{level\_shift}] The default is 0.

  \item[\keywd{extrap}] This specifies an \htmlref{object}{pkvobject} of
     type \clsnmref{SelfConsistentExtrapolation}.  The default is a
     \clsnmref{DIIS} object.

  \item[\keywd{memory}] The amount of memory that each processor may use.
     The default is 0 (minimal memory use).

  \item[\keywd{debug}] This integer can be used to produce output for
     debugging.  The default is 0.

  \item[\keywd{local\_density}] If this is true, a local copy of the
     density and $G$ matrix will be made on all nodes, even if a
     distributed matrix specialization is used.  The default is true.

  \item[\keywd{guess\_wavefunction}] This specifies the initial guess for
     the solution to the SCF equations.  This can be either a
     \clsnmref{OneBodyWavefunction} \htmlref{object}{pkvobject} or the name
     of file that contains the saved state of a
     \clsnmref{OneBodyWavefunction} object.  By default the one-electron
     hamiltonian will be diagonalized to obtain the initial guess.

\end{description}

%%%%%%%%%%%%%%%%%%%%%%%%%%%%%%%%%%%%%%%%%%%%%%%%%%%%%%%%%%%%%%%%%%%%%%%%%%

\section{The \clsnm{CLSCF} Class}
\label{CLSCF}\index{CLSCF}

The \clsnm{CLSCF} class derives from \clsnmref{SCF}.  It is used to compute
the Hartree-Fock energies of closed-shell molecules.  In addition to the
input for the \clsnmref{SCF} parent class, \clsnm{CLSCF} objects read the
following input:

\begin{description}
  \item[\keywd{total\_charge}] This integer gives the total charge, $c$, of
     the molecule.  The default is 0.

  \item[\keywd{docc}] This \htmlref{vector}{pkvarray} of integers gives the
     total number of doubly occupied orbitals of each irreducible
     representation.  By default, this will be chosen to make the molecule
     uncharged and the electrons will be distributed among the irreducible
     representations according to the orbital energies.

\end{description}

%%%%%%%%%%%%%%%%%%%%%%%%%%%%%%%%%%%%%%%%%%%%%%%%%%%%%%%%%%%%%%%%%%%%%%%%%%

\section{The \clsnm{HSOSSCF} Class}
\label{HSOSSCF}\index{HSOSSCF}

The \clsnm{HSOSSCF} class derives from \clsnmref{SCF}.  It is used to
compute the Hartree-Fock energies of high-spin open-shell molecules.  In
addition to the input for the \clsnmref{SCF} parent class, \clsnm{HSOSSCF}
objects read the following input:

\begin{description}
  \item[\keywd{total\_charge}] This integer gives the total charge, $c$, of
     the molecule.  The default is 0.

  \item[\keywd{nsocc}] This integer gives the total number of singly
     occupied orbitals, $n_\mathrm{socc}$.  If this is not given, then
     \keywd{multiplicity} will be read.

  \item[\keywd{multiplicity}] This integer gives the multiplicity, $m$, of
     the molecule.  The number of singly occupied orbitals is then
     $n_\mathrm{socc} = m - 1$.  If neither \keywd{nsocc} nor
     \keywd{multiplicity} is given, then if, in consideration of
     \clsnmref{total\_charge}, the number of electrons is even, the default
     $n_\mathrm{socc}$ is 2.  Otherwise, it is 1.

  \item[\keywd{ndocc}] This integer gives the total number of doubly
     occupied orbitals $n_\mathrm{docc}$.  The default $n_\mathrm{docc} =
     (c - n_\mathrm{socc})/2$.

  \item[\keywd{socc}] This \htmlref{vector}{pkvarray} of integers gives the
     total number of singly occupied orbitals of each irreducible
     representation.  By default, the $n_\mathrm{socc}$ singly occupied
     orbitals will be distributed according to orbital eigenvalues.  If
     \keywd{socc} is given, then \keywd{docc} must be given and they
     override \keywd{nsocc}, \keywd{multiplicity}, \keywd{ndocc}, and
     \keywd{total\_charge}.

  \item[\keywd{docc}] This \htmlref{vector}{pkvarray} of integers gives the
     total number of doubly occupied orbitals of each irreducible
     representation.  By default, the $n_\mathrm{docc}$ singly occupied
     orbitals will be distributed according to orbital eigenvalues.  If
     \keywd{docc} is given, then \keywd{socc} must be given and they
     override \keywd{nsocc}, \keywd{multiplicity}, \keywd{ndocc}, and
     \keywd{total\_charge}.

  \item[\keywd{maxiter}] This has the same meaning as in the parent
     class, \clsnmref{SCF}; however, the default value is 100.

  \item[\keywd{level\_shift}] This has the same meaning as in the parent
     class, \clsnmref{SCF}; however, the default value is 1.0.

\end{description}


%%%%%%%%%%%%%%%%%%%%%%%%%%%%%%%%%%%%%%%%%%%%%%%%%%%%%%%%%%%%%%%%%%%%%%%%%%

\section{The \clsnm{MBPT2} Class}
\label{MBPT2}\index{MBPT2}

The \clsnm{MBPT2} class derives from \clsnmref{Wavefunction}.
\clsnm{MBPT2} objects read the following input:

\begin{description}
  \item[\keywd{reference}] This gives the reference wavefunction.  It must
     be an \htmlref{object}{pkvobject} of type \clsnmref{CLSCF} for
     closed-shell molecules and \clsnmref{HSOSSCF} for open-shell
     molecules.  The is no default.

  \item[\keywd{nfzc}] The number of frozen core orbitals.  The default
     is 0.  If no atoms have an atomic number greater than 30, then
     the number of orbitals to be frozen can be automatically determined
     by specifying \keywd{nfzc = auto}.

  \item[\keywd{nfzv}] The number of frozen virtual orbitals.  The default
     is 0.

  \item[\keywd{memory}] The amount of memory, in bytes, that each processor
     may use.

  \item[\keywd{method}] This gives a string that must take on one of the
     values below.  The default is \keywd{mp} for closed-shell systems
     and \keywd{zapt} for open-shell systems.

      \begin{description}

        \item[\keywd{mp}] Use M\o{}ller-Plesset perturbation theory.  This
           is only valid for closed-shell systems.  Energies and
           gradients can be computed with this method.

        \item[\keywd{opt1}] Use the OPT1 variant of open-shell perturbation
           theory.  Only energies can be computed for open-shell systems.

        \item[\keywd{opt2}] Use the OPT2 variant of open-shell perturbation
           theory.  Only energies can be computed for open-shell systems.

        \item[\keywd{zapt}] Use the ZAPT variant of open-shell perturbation
           theory.  Only energies can be computed for open-shell systems.

     \end{description}

  \item[\keywd{algorithm}] This gives a string that must take on one of the
     values given below.  The default is \keywd{memgrp} for closed-shell
     systems.  For open-shell systems \keywd{v1} is used for a small number
     of processors and \keywd{v2} is used otherwise.

     \begin{description}

        \item[\keywd{memgrp}] Use the distributed shared memory algorithm
          (which uses a \clsnmref{MemoryGrp} object).  This is only valid
          for MP2 energies and gradients.

        \item[\keywd{v1}] Use algorithm V1.  Only energies can be
          computed.

        \item[\keywd{v2}] Use algorithm V2.  Only energies can be
          computed.

        \item[\keywd{v2lb}] Use a modified V2 algorithm that may compute a
           few more two electron integrals, but may get better load balance
           on the $O(n_\mathrm{basis}^5)$ part of the calculation.  Only
           energies can be computed.

     \end{description}

  \item[\keywd{memorygrp}] A \clsnmref{MemoryGrp}
     \htmlref{object}{pkvobject} that is used by the \type{memgrp}
     algorithm.  If this is not given the program will try to find an
     appropriate default.

  \item[\keywd{debug}] If this is nonzero, extra information is written to
     the output.  The default is 0.

\end{description}



% this is too messy for LaTeX
\begin{htmlonly}

\chapter{MPQC Class Hierarchies}

The input for some objects can require other objects.  Usually, it doesn't
matter precisely which object is given, only that it be one of the
derivatives of a particular class.  Below is summarized all of the
relationships between the classes.  Classes that are itemized below another
class derive from that class.  So, for example, if an object of
\clsnm{Function} is needed, then any concrete class that derives from
\clsnm{Function} will do.  Thus, \clsnm{MPBT2} and \clsnm{HSOSSCF} will
work.  However, \clsnm{SCF} will not because it is abstract and
\clsnm{IntSetCoor} will not because it is not a derivative.

\begin{itemize}
 \item[\clsnmref{Debugger}]
 \item[\clsnmref{Convergence}] Concrete
   \begin{itemize}
     \item[\clsnmref{MolEnergyConvergence}] Concrete
   \end{itemize}
 \item[\clsnmref{Function}]
   \begin{itemize}
     \item[\clsnmref{MolecularEnergy}]
     \begin{itemize}
       \item[\clsnmref{Wavefunction}]
       \begin{itemize}
         \item[\clsnmref{MBPT2}] Concrete
         \item[\clsnmref{OneBodyWavefunction}]
         \begin{itemize}
           \item[\clsnmref{SCF}]
         \end{itemize}
           \begin{itemize}
             \item[\clsnmref{CLSCF}] Concrete
             \item[\clsnmref{HSOSSCF}] Concrete
           \end{itemize}
       \end{itemize}
     \end{itemize}
   \end{itemize}
 \item[\clsnmref{GaussianBasisSet}] Concrete
 \item[\clsnmref{HessianUpdate}]
   \begin{itemize}
     \item[\clsnmref{DFPUpdate}] Concrete
       \begin{itemize}
         \item[\clsnmref{BFGSUpdate}] Concrete
       \end{itemize}
     \item[\clsnmref{PowellUpdate}] Concrete
   \end{itemize}
 \item[\clsnmref{IntCoor}]
   \begin{itemize}
     \item[\clsnmref{SimpleCo}]
       \begin{itemize}
         \item[\clsnmref{BendSimpleCo}] Concrete
         \item[\clsnmref{LinIPSimpleCo}] Concrete
         \item[\clsnmref{LinOPSimpleCo}] Concrete
         \item[\clsnmref{OutSimpleCo}] Concrete
         \item[\clsnmref{ScaledTorsSimpleCo}] Concrete
         \item[\clsnmref{StreSimpleCo}] Concrete
         \item[\clsnmref{TorsSimpleCo}] Concrete
       \end{itemize}
     \item[\clsnmref{SumIntCoor}] Concrete
   \end{itemize}
 \item[\clsnmref{IntCoorGen}] Concrete
 \item[\clsnmref{Integral}]
   \begin{itemize}
     \item[\clsnmref{IntegralV3}] Concrete
   \end{itemize}
 \item[\clsnmref{MemoryGrp}]
 \item[\clsnmref{MessageGrp}]
   \begin{itemize}
     \item[\clsnmref{ProcMessageGrp}] Concrete
     \item[\clsnmref{ShmMessageGrp}] Concrete
   \end{itemize}
 \item[\clsnmref{MolecularFrequencies}] Concrete
 \item[\clsnmref{MolecularCoor}]
   \begin{itemize}
     \item[\clsnmref{IntMolecularCoor}]
       \begin{itemize}
         \item[\clsnmref{RedundMolecularCoor}] Concrete
         \item[\clsnmref{SymmMolecularCoor}] Concrete
       \end{itemize}
   \end{itemize}
 \item[\clsnmref{Molecule}] Concrete
 \item[\clsnmref{Optimize}]
   \begin{itemize}
     \item[\clsnmref{EFCOpt}] Concrete
     \item[\clsnmref{LineOpt}]
     \item[\clsnmref{QNewtonOpt}] Concrete
   \end{itemize}
 \item[\clsnmref{PointGroup}] Concrete
 \item[\clsnmref{SCBlockInfo}] Concrete
 \item[\clsnmref{SCDimension}] Concrete
 \item[\clsnmref{SCMatrixKit}]
   \begin{itemize}
     \item[\clsnmref{DistSCMatrixKit}] Concrete
     \item[\clsnmref{LocalSCMatrixKit}] Concrete
     \item[\clsnmref{ReplSCMatrixKit}] Concrete
   \end{itemize}
 \item[\clsnmref{SelfConsistentExtrapolation}]
   \begin{itemize}
     \item[\clsnmref{DIIS}] Concrete
   \end{itemize}
 \item[\clsnmref{SetIntCoor}]
\end{itemize}

\end{htmlonly}

\chapter*{Glossary}
\label{glossary}

\begin{itemize}
\item[{\bfseries abstract class}]
          An abstract class cannot be created.  It is used
          as a parent class of another class.
\index{abstract}
\item[{\bfseries concrete class}]
          A class that is fully described and can be created.
\index{concrete}
\item[{\bfseries class}]
          A class data type.  Data with class types are objects.
\index{class}
\item[{\bfseries derivation}]
          The process of making a new class based on an already
          existing class.  The only way to use an abstract class
          is to make a derivative concrete class.
\index{derivation}
\item[{\bfseries instantiation}]
          The process of creating an object of a given class.
\index{instantiation}
\item[{\bfseries object}]
          A piece of data with associated functions.  The data
          contained and manipulations that can be done on an object
          are described in a class.
\index{object}
\end{itemize}

\printindex

\end{document}
