
\documentclass{report}

\usepackage{html}
\usepackage{verbatim}
\usepackage{alltt}
\usepackage{makeidx}

% Putting the item in a parbox prevent long member names from
% overrunning the line.
\newcommand*\classmember[1]{\parbox{4.5in}{\normalfont\bfseries #1}}
\newenvironment{classinterface}
               {\list{}{\setlength{\labelwidth}{\z@ \itemindent-\leftmargin}
                        \setlength{\itemsep}{0pt}
                        \setlength{\parsep}{0pt}
                        \setlength{\listparindent}{2em}
                        \renewcommand{\makelabel}[1]{\classmember{##1}}
                        }}
               {\endlist}

% classname
\newcommand{\clsnm}[1]{{\sffamily #1}}
\newcommand{\clsnmref}[1]{\htmlref{{\sffamily #1}}{#1}}

% variable
\newcommand{\vrbl}[1]{{\itshape #1}}

% source code
\newcommand{\srccd}[1]{{\ttfamily #1}}

% file names
\newcommand{\filnm}[1]{{\ttfamily #1}}

% command names
\newcommand{\exenm}[1]{{\ttfamily #1}}

% text to be typed by user
\newcommand{\type}[1]{{\ttfamily #1}}

% keyword
\newcommand{\keywd}[1]{{\ttfamily #1}}

\newcommand{\doheading}[2]{
  \ifnum #1=0
  \part{#2}
  \else\ifnum #1=1
    \chapter{#2}
    \else\ifnum #1=2
      \section{#2}
      \else\ifnum #1=3
        \subsection{#2}
        \else\ifnum #1=4
          \subsubsection{#2}
          \fi
        \fi
      \fi
    \fi
  \fi
}

\newcounter{headingnum}
\newcounter{subheadingnum}
\newcounter{subsubheadingnum}
\newcounter{clsheadingnum}
\newcounter{clssubheadingnum}
\setcounter{headingnum}{1}
\setcounter{subheadingnum}{2}
\setcounter{subsubheadingnum}{3}
\setcounter{clsheadingnum}{2}
\setcounter{clssubheadingnum}{3}

\newcommand{\scchapter}[1]{\doheading{\theheadingnum}{#1}}
\newcommand{\scsection}[1]{\doheading{\thesubheadingnum}{#1}}
\newcommand{\scsubsection}[1]{\doheading{\thesubsubheadingnum}{#1}}
\newcommand{\scclssection}[1]{\doheading{\theclsheadingnum}{#1}}
\newcommand{\scclssubsection}[1]{\doheading{\theclssubheadingnum}{#1}}

\newcommand{\clssection}[1]{\scclssection{The \clsnm{#1} Class}\label{#1}\index{#1}}
\newcommand{\tclssection}[2]{\scclssection{The \clsnm{#1}$<$#2$>$ Template Class}\label{#1}\index{#1}}
\newcommand{\clssubsection}[1]{\scclssubsection{#1}}


\setcounter{tocdepth}{1}

\makeindex

\begin{document}

\title{The MPQC User Manual}

\author{Curtis L. Janssen \and Edward T. Seidl \and Ida M. B. Nielsen
        \and Michael E. Colvin}

\date{\today}

\maketitle

\begin{abstract}
The Massively Parallel Quantum Chemistry program (MPQC) computes
the properties of molecules, {\it ab initio}, on a wide variety
of computer architectures.

MPQC can compute closed shell and general restricted open shell
Hartree-Fock energies and gradients, second order open shell
perturbation theory (OPT2[2]) and Z-averaged perturbation theory
(ZAPT2) energies, and second order closed shell M\o{}ller-Plesset
perturbation theory energies and gradients.  It also includes a
robust internal coordinate geometry optimizer that efficiently
optimizes molecules with many degrees of freedom.

MPQC runs on Unix compatible workstations (Intel/Linux, R8000,
RS/6000), symmetric multi-processors (SGI Power Challenge), and
massively parallel computers (IBM SP2, Intel Paragon).
\end{abstract}

\tableofcontents

\chapter{Introduction}
The Massively Parallel Quantum Chemistry program (MPQC) computes
the properties of molecules, {\it ab initio}, on a wide variety
of computer architectures.

MPQC can compute closed shell and general restricted open shell
Hartree-Fock energies and gradients, second order open shell
perturbation theory (OPT2[2]) and Z-averaged perturbation theory
(ZAPT2) energies, and second order closed shell M\o{}ller-Plesset
perturbation theory energies and gradients.  It also includes a
robust internal coordinate geometry optimizer that efficiently
optimizes molecules with many degrees of freedom.

MPQC runs on Unix compatible workstations (Intel/Linux, SGI R8000,
IBM RS/6000), symmetric multi-processors (SGI Power Challenge),
and massively parallel computers (IBM SP2, Intel Paragon).

MPQC is designed using object-orient programming techniques and
implemented in the C++ programming language.  This design has
propagated even to the input of MPQC which is object-oriented
as well.  The user specifies a group of objects that the program
creates when it starts.  MPQC manipulates the objects to produce
energies, gradients, geometries, and properties.

The object-oriented nature of the input makes this user manual
read like a bit like a programmers manual.  Object-orient oriented
terminology is thus a natural way to describe the input and a
short \hyperref{glossary}{glossary (Chapter~}{)}{glossary}
is provided for those in need.

The bulk of the manual is divided into two major parts, the
\hyperref{abstract classes}{abstract classes (Chapter~}{)}{absclass}
and
\hyperref{concrete classes}{concrete classes (Chapter~}{)}{conclass}.


\chapter{MPQC Input}
\index{input}

MPQC starts off by creating a
\hyperref{\clsnm{ParsedKeyVal}}
         {\clsnm{ParsedKeyVal} (see Section~}
         {)}
         {ParsedKeyVal}
object that parses the input file specified on the command line.
Since MPQC is still written in a hybrid of C and C++ the format
is a mixture of keyword value pairs and object specifications.

The following keywords are recognized at the top level of the
input:
\begin{itemize}
\item[{\ttfamily message}]
        This optional keyword gives a \clsnmref{MessageGrp} object
        that sets up MPQC for parallel processing.  The default
        is a \clsnmref{ProcMessageGrp} that runs MPQC on a single
        processor.  If another \clsnmref{MessageGrp} is used because
        of command line options or due to environment in which MPQC
        is run, then this keyword may be ignored.
\item[{\ttfamily debug}]
        This optional keyword gives a \clsnmref{Debugger} object
        which is used to help find the problem if MPQC encounters
        a catastrophic error.
\end{itemize}

Several other sections are required \ldots documentation is on the way.


\section{The \clsnm{ParsedKeyVal} Input Format}
\label{ParsedKeyVal}
\label{KeyVal}
\index{ParsedKeyVal}
\index{KeyVal}

The \clsnm{KeyVal} class provides a means for users to associate keywords
with values.  \clsnm{ParsedKeyVal} is a specialization of \clsnm{KeyVal}
that permits keyword/value associations in text such as an input file or a
command line string.

The package is flexible enough to allow complex structures and arrays as
well as objects to be read from an input file.

\subsection{Assignment}

As an example of the use of \clsnmref{ParsedKeyVal}, consider the following
input:
\begin{verbatim}
x_coordinate = 1.0
y_coordinate = 2.0
x_coordinate = 3.0
\end{verbatim}
Two assignements will be made.  The keyword \verb|x_coordinate| will be
associated with the value \verb|1.0| and the keyword \verb|y_coordinate|
will be assigned to \verb|2.0|.  The third line in the above input
will have no effect since \verb|x_coordinate| was assigned previously.

\subsection{Keyword Grouping}

Lets imagine that we have a program which needs to read in the
characteristics of animals.  There are lots of animals so it might be
nice to catagorize them by their family.  Here is a sample format for
such an input file:
\begin{verbatim}
reptile: (
  alligator: (
    legs = 4
    extinct = no
    )
  python: (
    legs = 0
    extinct = no
    )
  )
bird: (
  owl: (
    flys = yes
    extinct = no
    )
  )
\end{verbatim}

This sample illustrates the use of \vrbl{keyword} \verb|=| \vrbl{value}
assignments and the keyword grouping operators \verb|(| and \verb|)|.
The keywords in this example are
\begin{verbatim}
reptile:alligator:legs
reptile:alligator:extinct
reptile:alligator:legs
reptile:python:size
reptile:python:extinct
bird:owl:flys
bird:owl:extinct
\end{verbatim}

The \verb|:|'s occuring in these keywords break the keywords into
smaller logical units called keyword segments.  The sole purpose of this
is to allow persons writing input files to group the input into easy to
read sections.  In the above example there are two main sections, the
reptile section and the bird section.  The reptile section takes the
form \verb|reptile| \verb|:| \verb|(| \vrbl{keyword} \verb|=| \vrbl{value}
assignments \verb|)|.  Each of the keywords found between the
parentheses has the \verb|reptile:| prefix attached to it.  Within each
of these sections further keyword groupings can be used, as many and as
deeply nested as the user wants.

Keyword grouping is also useful when you need many different programs to
read from the same input file.  Each program can be assigned its own
unique section.

\subsection{Array Construction}

A method of specifying arrays of data would be useful.  One way to
do this would be as follows:
\begin{verbatim}
array: (
  0 = 5.4
  1 = 8.9
  2 = 3.7
  )
\end{verbatim}
The numbers \verb|0|, \verb|1|, and \verb|2| in this example are keyword
segments which serve as indices of \verb|array|.  However, this syntax
is somewhat awkward and array construction operators have been provided
to simplify the input for this case.  The following input is equivalent
to the above input:
\begin{verbatim}
array = [ 5.4 8.9 3.7 ]
\end{verbatim}

More complex arrays than this can be imagined.  Suppose an array of
complex numbers is needed.  For example the input
\begin{verbatim}
carray: (
  0: ( r = 1.0  i = 0.0 )
  1: ( r = 0.0  i = 1.0 )
  )
\end{verbatim}
could be written as
\begin{verbatim}
carray: [
  (r = 1.0 i = 0.0)
  (r = 0.0 i = 1.0)
  ]
\end{verbatim}
which looks a bit nicer than the example without array construction
operators.

Furthermore, the array construction operators can be nested in about
every imaginable way.  This allows multidimensional arrays of
complicated data to be represented.

It would be nice to just extend an array that is already defined.  This
feature has not been implemented; however, user demand for it might
result in its implementation.  The operators reserved for this purpose
are \verb|+[| and \verb|]|.

\subsection{Table Construction}

Although the array contstruction operators will suit most requirements
for enumerated lists of data, in some cases the input can still look
ugly.  This can, in some cases, be fixed with the table construction
operators, \verb|{| and \verb|}|.

Suppose a few long vectors of the same length are needed and the data in
the \verb|i|th element of each array is related or somehow belong
together.  If the arrays are so long that the width of a page is
exceeded, then data that should be seen next to each other are no longer
adjacent.  The way this problem can be fixed is to arrange the data
vertically side by side rather than horizontally.  The table
construction operators allows the user to achieve this in a very simple
manner.
\begin{verbatim}
balls: (
  color    = [  red      blue     red   ]
  diameter = [   12       14       11   ]
  material = [  rubber  vinyl   plastic ]
  bouces   = [  yes      no       no    ]
  coordinate = [[ 0.0  0.0  0.0]
                [ 1.0  2.0 -1.0]
                [ 1.0 -1.0  1.0]]
  )
\end{verbatim}
can be written
\begin{verbatim}
balls: (
  { color diameter material bounces     coordinate}
                  
  {  red     12    rubber    yes     [ 0.0  0.0  0.0]
     blue    14    vinyl     no      [ 1.0  2.0 -1.0]
     red     11    plastic   no      [ 1.0 -1.0  1.0] }
  )
\end{verbatim}
The length and width of the table can be anything the user desires.

\subsection{Value Substitution}
\label{ParsedKeyVal:valsub}

Occasionally, a user may need to repeat some value several times in an
input file.  If the value must be changed, it would be nice to only
change the value in one place.  The value substitution feature of
\clsnmref{ParsedKeyVal} always the user to do this.  Any place a value can
occur the user can place a \verb|$|.  Following this a keyword must be
given.  This keyword must have been assigned before the attempt is made
to use its value in a value substitution.

Here is an example illustrating most of the variable substition
features:
\begin{verbatim}
default:linewidth = 130
testsub: (
  ke: (
    ke_1 = 1
    ke_2 = 2
    ke_3: (
      ke_31 = 31
      ke_32 = 32
      )
    )
  kx = $ke
  r1 = 3.0
  r2 = $r1
  linewidth = $:default:linewidth
  )
\end{verbatim}
is the same as specifying
\begin{verbatim}
testsub: (
  ke: (
    ke_1 = 1
    ke_3: (
      ke_31 = 31
      ke_32 = 32
      )
    ke_2 = 2
    )
  linewidth = 130
  r2 = 3.0
  r1 = 3.0
  kx: (
    ke_1 = 1
    ke_2 = 2
    ke_3: (
      ke_31 = 31
      ke_32 = 32
      )
    )
  )
\end{verbatim}
It can be seen from this that value substitution can result in entire
keyword segment hierarchies being copied, as well as simple
substitutions.


\subsection{Expression Evaluation}

It is nice to be able to multiply numbers together directly in the input
file.  For example, you may need to change the units on a number and
want to see where exactly the number is coming from.  This motivated the
development of an expression evaluation facility in the
\clsnmref{ParsedKeyVal} package.  This facility is still under development,
but some primitive expressions can be evaluated in the current release.

Suppose your program requires several parameters \verb|x1|, \verb|x2|,
and \verb|x3|.  Furthermore, suppose that their ratios remain fixed for
all the runs of the program that you desire.  It would be best to
specify some scale factor in the input that would be the only thing that
has to be changed from run to run.  If you don't want to or cannot
modify the program, then this can be done directly with
\clsnmref{ParsedKeyVal} as follows
\begin{verbatim}
scale = 1.234
x1 = ( $:scale *  1.2 )
x2 = ( $:scale *  9.2 )
x3 = ( $:scale * -2.0 )
\end{verbatim}
So we see that to the right of the ``\verb|=|'' the characters
``\verb|(|'' and ``\verb|)|'' are the expression construction operators.
This is in contrast to their function when they are to the left of the
``\verb|=|'', where they are the keyword grouping operators.

Currently, the expression must be binary and the data is all converted
to double.  If you use the expression construction operators to produce
data that the program expects to be integer, you will certainly get the
wrong answers (unless the desired value happens to be zero).

\subsection{Objects}

An instance of an object can be can be specified by surrounding it's
classname with the ``\verb|<|'' and ``\verb|>|'' operators immediately
after the keyword naming the data.

A pointer to a single object can be associated with multiple keywords by
using
\hyperref{value substitution}
         {value substitution (see Section~}
         {)}
         {ParsedKeyVal:valsub}.
This is accomplished by holding references to all objects once they are
read in.

Consider the following input for a linked list, where \clsnm{A}
\begin{verbatim}
a1<A>: (
    next<A>: (
        next<B>: (
            bdata = 4
            next<A>:()
            )
        )
    )
a2 = $:a
\end{verbatim}

The \verb|a1| list would contain two \verb|A| objects followed by a
\verb|B| object followed by another \verb|A| object.  The \verb|a2| list
refers to exactly the same object as \verb|a1| (not a copy of
\verb|a1|).



\section{Sample Input}

The following input computes the energy of the water molecule
using four processors on a shared memory machine:
\begin{alltt}
message<\clsnmref{ShmMessageGrp}>: n = 4

mpqc: (
  molecule<\clsnmref{Molecule}>:(
    symmetry=c2v
    angstrom=yes
    { atoms geometry } = {
       O [0.0 0.0  0.0]
       H [0.8 0.0 -0.6]
     }
    )

  basis<\clsnmref{GaussianBasisSet}>: (
    molecule = $..:molecule
    name = "STO-3G"
    )
  )
\end{alltt}



\chapter{Running MPQC}
\index{running MPQC}

MPQC runs on Linux workstations, Silicon Graphics computers,
and more.

\section{Command Line Options}

\section{Environmental Variables}

By default, MPQC tries to find basis set library files in
the source code distribution.  If the executable or source
code is moved, MPQC can be notified with the environmental
variable \verb|SCLIBDIR|.  In the source code distribution
this directory is \verb|SC/lib|.

For example if you need to run MPQC on a machine that doesn't
have the source code distribution in the same place as the
machine on which MPQC is compiled you must do the following
on the machine with the source code:

\begin{alltt}
cd SC/lib
tar cvf ../sclib.tar .
\end{alltt}

Then transfer \verb|sclib.tar| to the machine that you want to run
MPQC on and do something like

\begin{alltt}
mkdir ~/sclib
cd ~/sclib
tar xvf ../sclib.tar
setenv SCLIBDIR ~/sclib
\end{alltt}

The setenv command is specific to the C-shell.  You will need to
do what is appropriate for your shell.

Now MPQC should be ready to run.

\section{Running MPQC on a Shared Memory Multiprocessor}
\index{computers, shared memory}
\index{shared memory}

By default, MPQC will run on only one CPU.  To specify more, associate
a \clsnmref{ShmMessageGrp} object with the \verb|message|
keyword.  For example, putting the following at the top of your input file
would run MPQC on four processors:

\begin{alltt}
message<\clsnmref{ShmMessageGrp}>: n = 4
\end{alltt}


To run MPQC, type:
\begin{alltt}
{\itshape mpqc_executable} {\itshape input_file}
\end{alltt}

If MPQC should unexpectedly die, shared memory segments and
semaphores will be left on the machine.  These should be promptly
cleaned up or other jobs may be prevented from running.  To
see if you have any of these resources allocated, use the
\verb|ipcs| command.  The output will look something
like:

\begin{alltt}
IPC status from /dev/kmem as of Wed Mar 13 14:42:18 1996
T     ID     KEY        MODE       OWNER    GROUP
Message Queues:
Shared Memory:
m 288800 0x00000000 --rw-------  cljanss     user
Semaphores:
s    390 0x00000000 --ra-------  cljanss     user
s    391 0x00000000 --ra-------  cljanss     user
\end{alltt}

To remove the IPC resources used by \verb|cljanss| in
the above example, type:

\begin{alltt}
ipcrm -m 288800
ipcrm -s 390
ipcrm -s 391
\end{alltt}

\section{Running MPQC on the Intel Paragon Running OSF}
\index{computers, Intel Paragon}
\index{Paragon}

To run interactively type:
\begin{alltt}
{\itshape mpqc_executable} -sz {\itshape n_node} {\itshape input_file}
\end{alltt}

\section{Running MPQC on the IBM SP2}
\index{computers, IBM SP2}
\index{SP2}

The code is currently working on the
\htmladdnormallink{Maui SP2}{http://www.mhpcc.edu/}.
They have \htmladdnormallink{extensive information}{http://www.mhpcc.edu/}
on how to run on their SP2.

\subsection{Running Interactively}

The job is controlled by a group of environment variables.
Here are my current settings:
\begin{verbatim}
MP_CSS_INTERRUPT=YES
MP_HOSTFILE=NULL
MP_EUILIB=us
MP_EUIDEVICE=css0
MP_RMPOOL=0
MP_RESD=YES
MP_LABELIO=yes
MP_INFOLEVEL=1
MP_PGMMODEL=spmd
MP_PROCS=4
\end{verbatim}
The last variable, \verb|MP_NPROCS|, sets the number
of processors that will be used.

The program is run with the command:
\begin{alltt}
poe {\itshape executable} {\itshape inputfile} -messagegrp MPIMessageGrp
\end{alltt}

\subsection{Using Load Leveler}

A load leveler script is needed to submit a command to the batch queue.
This specifies the queue (called class) and constraints on memory the
number of processors, etc.  The following template is a good start for
MPQC:

\begin{alltt}
#!/bin/sh
#@ environment = MessageGrp=MPIMessageGrp;MP_LANG=En_US;MP_LABELIO=YES;MP_INFOLEVEL=1;MP_PGMMODEL=spmd;MP_RESD=YES;MP_CSS_INTERRUPT=YES;MP_EUILIB=us
#@ input = /dev/null
#@ output = mpqc.out.$(Cluster).$(Process)
#@ error = mpqc.err.$(Cluster).$(Process)
#@ class = {\itshape CLASS}
#@ job_type = parallel
#@ min_processors = {\itshape MINPROC}
#@ max_processors = {\itshape MAXPROC}
#@ requirements =  (Adapter == "hps_user" && Memory >= {\itshape MEMORY})
#@ notification = complete
#@ notify_user = {\itshape EMAIL}
#@ shell = /bin/sh
#@ cpu_limit = {\itshape CPULIMIT}
#@ queue
/usr/lpp/poe/bin/poe {\itshape EXECUTABLE} {\itshape INPUTFILE}
\end{alltt}

To make the above useful, the italicized variables must be replaced.
Table~\ref{running:spvariables} explains the meaning of the variables:

\begin{table}
\caption{Variables Meanings for Sample IBM SP2 Loadleveler Script.}
\begin{center}
\begin{tabular}{lp{2.5in}c}
  \multicolumn{1}{c}{Variable}
     & \multicolumn{1}{c}{Meaning}
     & \multicolumn{1}{c}{Example} \\
  CLASS & queue name & small\_short \\
  MINPROC & minimum number of nodes & 4 \\
  MAXPROC & maximum number of nodes & 8 \\
  EMAIL & your email address & cljanss@ca.sandia.gov \\
  CPULIMIT & maximum CPU time (hour:min:sec) & 1:0:0 \\
  MEMORY & maximum amount of memory (MBytes) & 64 \\
  EXECUTABLE & path to the executable & mpqcic \\
  INPUTFILE & the input file name & mpqc.in \\
\end{tabular}
\end{center}
\label{running:spvariables}
\end{table}

The modified script is submitted with the following command:

\begin{alltt}
llsubmit {\itshape scriptname}
\end{alltt}



\chapter{MPQC Concrete Classes}
\label{conclass}
\index{concrete}

The concrete classes are classes that can be directly instantiated
by the user in the input file.


\clssection{Molecule}

The \clsnm{Molecule} class contains information about molecules.  It has a
\clsnmref{KeyVal} constructor that can create a new molecule from either a
PDB file or from a list of Cartesian coordinates.

The following \clsnmref{ParsedKeyVal} input reads from the PDB
file \verb|h2o.pdb|:
\begin{alltt}
molecule<\clsnmref{Molecule}>: (
   pdb_file = "h2o.pdb"
 )
\end{alltt}

The following input explicitly gives the atom coordinates:
\begin{alltt}
molecule<\clsnmref{Molecule}>: (
    angstrom=yes
    \{ atom_labels atoms           geometry            \} = \{
          O1         O   [ 0.000000000 0  0.369372944 ]
          H1         H   [ 0.783975899 0 -0.184686472 ]
          H2         H   [-0.783975899 0 -0.184686472 ]
     \}
    )
  )
\end{alltt}
The default units are Bohr with can be overridden with
\verb|angstrom=yes|.  The \verb|atom_labels| array can be
omitted.  The \verb|atoms| and \verb|geometry| arrays
are required.

The \clsnmref{Molecule} class has a \clsnmref{PointGroup}
member object, which also has a \clsnmref{KeyVal} constructor
that is called when a \clsnmref{Molecule} is made.  The
following example constructs a molecule with $C_{2v}$ symmetry:
\begin{alltt}
molecule<\clsnmref{Molecule}>: (
    symmetry=c2v
    angstrom=yes
    \{ atoms         geometry            \} = \{
        O   [0.000000000 0  0.369372944 ]
        H   [0.783975899 0 -0.184686472 ]
     \}
    )
  )
\end{alltt}
Only the symmetry unique atoms need can be specified.  Nonunique
atoms can be given too, however, numerical errors in the
geometry specification can result in the generation of extra
atoms so be careful.


\clssection{PointGroup}

The \clsnm{PointGroup} \clsnmref{KeyVal} constructor looks
for three keywords: \srccd{symmetry},
\srccd{symmetry\_frame}, and
\srccd{origin}. \srccd{symmetry} is a string containing the
Schoenflies symbol of the point group.  \srccd{origin} is an
array of doubles which gives the x, y, and z coordinates of
the origin of the symmetry frame.  \srccd{symmetry\_frame}
is a 3 by 3 array of arrays of doubles which specify the
principal axes for the transformation matrices as a unitary
rotation.

For example, a simple input which will use the default
\srccd{origin} and \srccd{symmetry\_frame} ((0,0,0) and the
unit matrix, respectively), might look like this:

\begin{alltt}
pointgrp<\clsnmref{PointGroup}>: (
  symmetry = "c2v"
)
\end{alltt}

By default, the principal rotation axis is taken to be the z
axis.  If you already have a set of coordinates which assume
that the rotation axis is the x axis, then you'll have to
rotate your frame of reference with \srccd{symmetry\_frame}:

\begin{alltt}
pointgrp<\clsnmref{PointGroup}>: (
  symmetry = "c2v"
  symmetry_frame = [
    [ 0 0 1 ]
    [ 0 1 0 ]
    [ 1 0 0 ]
  ]
)
\end{alltt}


\clssection{ShmMessageGrp}
\index{ShmMessageGrp}
\index{MessageGrp!ShmMessageGrp}

The \clsnm{ShmMessageGrp} class is an implementation of
\clsnmref{MessageGrp} that allows multiple process to be
started that communication with shared memory.  This
only provides improved performance if you have multiple
CPU's in a symmetric multiprocessor configuration.  Nonetheless,
it is quite useful on a single CPU for tracking down bugs.

The \clsnm{ShmMessageGrp} \clsnmref{KeyVal} constructor takes
a single argument that specifies the number of processors.
Here is an example of a \clsnmref{ParsedKeyVal} input that
creates a \clsnm{ShmMessageGrp} that runs on four processors.
\begin{alltt}
message<\clsnm{ShmMessageGrp}>: n = 4
\end{alltt}


\clssection{ProcMessageGrp}
\index{ProcMessageGrp}
\index{MessageGrp!ProcMessageGrp}

The \clsnm{ProcMessageGrp} class provides
a simple single processor implementation of
the \clsnmref{MessageGrp} class.


\clssection{Debugger}
\index{Debugger}

The \clsnm{Debugger} class describes what should be done when
a catastrophic error causes unexpected program termination.
It can try things such as run start a debugger running where
the program died or it can attempt to produce a stack traceback
showing roughly where the program died.  These attempts could
just complicate matters more if the program has gotten into
a peculiar enough state.

Table~\ref{debugger:keyval} shows the \clsnm{KeyVal} input parameters
for \clsnm{Debugger}.

\begin{table}
\caption{\clsnmref{KeyVal} Input for \clsnm{Debugger}}
\begin{center}
\begin{tabular}{lp{2.5in}p{1in}}
  \multicolumn{1}{c}{Keyword}
     & \multicolumn{1}{c}{Meaning}
     & \multicolumn{1}{c}{Default} \\
 \verb|debug|
        & Try to start a debugger when an error occurs.  Doesn't
        work on all machines.
        & \verb|yes|, if possible \\
 \verb|traceback|
        & Try to print out a traceback extracting return addresses
        from the call stack.  Doesn't work on most machines.
        & \verb|yes|, if possible \\
 \verb|exit|
        & Exit on errors.
        & \verb|yes| \\
 \verb|wait_for_debugger|
        & When starting a debugger go into an infinite loop
        to give the debugger a change to attach to the process.
        & \verb|yes| \\
 \verb|handle_defaults|
        & Handle a standard set of signals such as SIGBUS, SIGSEGV, etc.
        & \verb|yes| \\
 \verb|prefix|
        & Gives a string that is printed before each line
        that is printed by \clsnm{Debugger}.
        & empty \\
 \verb|cmd|
        & Gives a command to be executed to start the debugger.
        & varies with machine
\end{tabular}
\end{center}
\label{debugger:keyval}
\end{table}


\chapter{MPQC Abstract Classes}
\label{absclass}
\index{abstract}

The abstract classes cannot be instantiated in the MPQC input.
However, they are used to implement the
\htmlref{concrete classes}{conclass} which can be instantiated.


\clssection{MolecularEnergy}

The \clsnm{MolecularEnergy} class contains information about molecules.

The \clsnm{MolecularEnergy} \clsnm{KeyVal} constructor computes the
energy of molecules.  It requires as input a
\htmladdnormallink{\clsnm{Molecule}}{}.


\clssection{MessageGrp}
\index{MessageGrp}

The \clsnm{MessageGrp} abstract class provides
a mechanism for moving data and objects between
nodes in a parallel machine.


\chapter{Glossary}
\label{glossary}

\begin{itemize}
\item[{\bfseries abstract class}]
          An abstract class cannot be created.  It is used
          as a parent class of another class.
\index{abstract}
\item[{\bfseries concrete class}]
          A class that is fully described and can be created.
\index{concrete}
\item[{\bfseries class}]
          A class data type.  Data with class types are objects.
\index{class}
\item[{\bfseries instantiation}]
          The process of creating an object of a given class.
\index{instantiation}
\item[{\bfseries object}]
          A piece of data with associated functions.  The data
          contained and manipulations that can be done on an object
          are described in a class.
\index{object}
\end{itemize}

\printindex

\end{document}
