
\clssection{Molecule}

The \clsnm{Molecule} class contains information about molecules.  It has a
\clsnmref{KeyVal} constructor that can create a new molecule from either a
PDB file or from a list of Cartesian coordinates.

The following \clsnmref{ParsedKeyVal} input reads from the PDB
file \verb|h2o.pdb|:
\begin{alltt}
molecule<\clsnmref{Molecule}>: (
   pdb_file = "h2o.pdb"
 )
\end{alltt}

The following input explicitly gives the atom coordinates:
\begin{alltt}
molecule<\clsnmref{Molecule}>: (
    angstrom=yes
    \{ atom_labels atoms           geometry            \} = \{
          O1         O   [ 0.000000000 0  0.369372944 ]
          H1         H   [ 0.783975899 0 -0.184686472 ]
          H2         H   [-0.783975899 0 -0.184686472 ]
     \}
    )
  )
\end{alltt}
The default units are Bohr with can be overridden with
\verb|angstrom=yes|.  The \verb|atom_labels| array can be
omitted.  The \verb|atoms| and \verb|geometry| arrays
are required.

The \clsnmref{Molecule} class has a \clsnmref{PointGroup}
member object, which also has a \clsnmref{KeyVal} constructor
that is called when a \clsnmref{Molecule} is made.  The
following example constructs a molecule with $C_{2v}$ symmetry:
\begin{alltt}
molecule<\clsnmref{Molecule}>: (
    symmetry=c2v
    angstrom=yes
    \{ atoms         geometry            \} = \{
        O   [0.000000000 0  0.369372944 ]
        H   [0.783975899 0 -0.184686472 ]
     \}
    )
  )
\end{alltt}
Only the symmetry unique atoms need can be specified.  Nonunique
atoms can be given too, however, numerical errors in the
geometry specification can result in the generation of extra
atoms so be careful.
