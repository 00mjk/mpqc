
%%%%%%%%%%%%%%%%%%%%%%%%%%%%%%%%%%%%%%%%%%%%%%%%%%%%%%%%%%%%%%%%%%%%%%%%%%

\section{The \clsnm{MBPT2} Class}
\label{MBPT2}\index{MBPT2}

The \clsnm{MBPT2} class derives from \clsnmref{Wavefunction}.
\clsnm{MBPT2} objects read the following input:

\begin{description}
  \item[\keywd{reference}] This gives the reference wavefunction.  It must
     be an object of type \clsnmref{CLSCF} for closed-shell molecules and
     \clsnmref{HSOSSCF} for open-shell molecules.  The is no default.

  \item[\keywd{nfzc}] The number of frozen core orbitals.  The default
     is 0.

  \item[\keywd{nfzv}] The number of frozen virtual orbitals.  The default
     is 0.

  \item[\keywd{memory}] The amount of memory that each processor may use.

  \item[\keywd{method}] This gives a string that must take on one of the
     values below.  The default is \keywd{mp} for closed-shell systems
     and \keywd{zapt} for open-shell systems.

      \begin{description}

        \item[\keywd{mp}] Use M\o{}ller-Plesset perturbation theory.  This
           is only valid for closed-shell systems.  Energies and
           gradients can be computed with this method.

        \item[\keywd{opt1}] Use the OPT1 variant of open-shell perturbation
           theory.  Only energies can be computed for open-shell systems.

        \item[\keywd{opt2}] Use the OPT2 variant of open-shell perturbation
           theory.  Only energies can be computed for open-shell systems.

        \item[\keywd{zapt}] Use the ZAPT variant of open-shell perturbation
           theory.  Only energies can be computed for open-shell systems.

     \end{description}

  \item[\keywd{algorithm}] This gives a string that must take on one of the
     values given below.  The default is \keywd{memgrp} for closed-shell
     systems.  For open-shell systems \keywd{v1} is used for a small number
     of processors and \keywd{v2} is used otherwise.

     \begin{description}

        \item[\keywd{memgrp}] Use the distributed shared memory algorithm
          (which uses a \clsnmref{MemoryGrp} object).  This is only valid
          for MP2 energies and gradients.

        \item[\keywd{v1}] Use algorithm V1.

        \item[\keywd{v2}] Use algorithm V2.

        \item[\keywd{v2lb}] Use a modified V2 algorithm that may compute a
           few more two electron integrals, but may get better load balance
           on the $O(n_\mathrm{basis}^5)$ part of the calculation.

     \end{description}

  \item[\keywd{memorygrp}] A \clsnmref{MemoryGrp} object that is used by
     the \type{memgrp} algorithm.  If this is not given the program will
     try to find an appropriate default.

  \item[\keywd{debug}] If this is nonzero, extra information is written to
     the output.  The default is 0.

\end{description}

