
%%%%%%%%%%%%%%%%%%%%%%%%%%%%%%%%%%%%%%%%%%%%%%%%%%%%%%%%%%%%%%%%%%%%%%%%%%

\section{The \clsnm{Wavefunction} Class}
\label{Wavefunction}\index{Wavefunction}

The \clsnm{Wavefunction} abstract class derives from
\clsnmref{MolecularEnergy}.  \clsnm{Wavefunction} objects read the
following input:

\begin{description}
  \item[\keywd{basis}] Specifies a \clsnmref{GaussianBasisSet} object.
    There is no default.

  \item[\keywd{integral}] Specifies an \clsnmref{Integral} object that
    computes the two electron integrals.  The default is a
    \clsnmref{IntegralV3} object.

  \item[\keywd{print\_nao}] This specifies a boolean value.  If true
    the natural atomic orbitals will be printed.  Not all wavefunction
    will be able to do this.  The default is false.

  \item[\keywd{print\_npa}] This specifies a boolean value.  If true the
    natural population analysis will be printed.  Not all wavefunction will
    be able to do this.  The default is true if \keywd{print\_nao} is
    true, otherwise it is false.

\end{description}

%%%%%%%%%%%%%%%%%%%%%%%%%%%%%%%%%%%%%%%%%%%%%%%%%%%%%%%%%%%%%%%%%%%%%%%%%%

\section{The \clsnm{OneBodyWavefunction} Class}
\label{OneBodyWavefunction}\index{OneBodyWavefunction}

The \clsnm{OneBodyWavefunction} abstract class derives from
\clsnmref{Wavefunction}.  \clsnm{OneBodyWavefunction} objects read the
following input:

\begin{description}
  \item[\keywd{eigenvector\_accuracy}] Gives the accuracy to which
    eigenvectors are initially computed.  The default 1.0e-7.  Accuracies
    are usually adjusted as needed anyway, so it should not be necessary to
    change this.

\end{description}
