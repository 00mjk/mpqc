
%%%%%%%%%%%%%%%%%%%%%%%%%%%%%%%%%%%%%%%%%%%%%%%%%%%%%%%%%%%%%%%%%%%%%%%%%%

\section{The \clsnm{SCF} Class}
\label{SCF}\index{SCF}

The \clsnm{SCF} abstract class derives from \clsnmref{OneBodyWavefunction}.
\clsnm{SCF} objects read the following input:

\begin{description}
  \item[\keywd{maxiter}]  This integer specifies the maximum number of
     SCF iterations.  The default is 40.

  \item[\keywd{density\_reset\_frequency}] This integer specifies how
     often, in term of SCF iterations, $\Delta D$ will be reset to $D$.
     The default is 10.

  \item[\keywd{reset\_occuptions}] Reassign the occupations after each
     iteration based on the eigenvalues.  This only has an effect for
     molecules with higher than $C_1$ symmetry.  The default is false.

  \item[\keywd{level\_shift}] The default is 0.

  \item[\keywd{extrap}] This specifies an object of type
     \clsnmref{SelfConsistentExtrapolation}.  The default is a
     \clsnmref{DIIS} object.

  \item[\keywd{memory}] The amount of memory that each processor may use.
     The default is 0 (minimal memory use).

  \item[\keywd{debug}] This integer can be used to produce output for
     debugging.  The default is 0.

  \item[\keywd{local\_density}] If this is true, a local copy of the
     density and $G$ matrix will be made on all nodes, even if a
     distributed matrix specialization is used.  The default is true.

  \item[\keywd{guess\_wavefunction}] This specifies the initial guess for
     the solution to the SCF equations.  This can be either a
     \clsnmref{OneBodyWavefunction} object or the name of file that
     contains the saved state of a \clsnmref{OneBodyWavefunction} object.
     By default the one-electron hamiltonian will be diagonalized to obtain
     the initial guess.

\end{description}

%%%%%%%%%%%%%%%%%%%%%%%%%%%%%%%%%%%%%%%%%%%%%%%%%%%%%%%%%%%%%%%%%%%%%%%%%%

\section{The \clsnm{CLSCF} Class}
\label{CLSCF}\index{CLSCF}

The \clsnm{CLSCF} class derives from \clsnmref{SCF}.  It is used to compute
the Hartree-Fock energies of closed-shell molecules.  In addition to the
input for the \clsnmref{SCF} parent class, \clsnm{CLSCF} objects read the
following input:

\begin{description}
  \item[\keywd{docc}] This vector of integers gives the total number of
     doubly occupied orbitals of each irreducible representation.  By
     default, this will be chosen to make the molecule uncharged and the
     electrons will be distributed among the irreducible representations
     according to the orbital energies.

\end{description}

%%%%%%%%%%%%%%%%%%%%%%%%%%%%%%%%%%%%%%%%%%%%%%%%%%%%%%%%%%%%%%%%%%%%%%%%%%

\section{The \clsnm{HSOSSCF} Class}
\label{HSOSSCF}\index{HSOSSCF}

The \clsnm{HSOSSCF} class derives from \clsnmref{SCF}.  It is used to
compute the Hartree-Fock energies of high-spin open-shell molecules.  In
addition to the input for the \clsnmref{SCF} parent class, \clsnm{HSOSSCF}
objects read the following input:

\begin{description}
  \item[\keywd{total\_charge}] This integer gives the total charge, $c$, of
     the molecule.  The default is 0.

  \item[\keywd{nsocc}] This integer gives the total number of singly
     occupied orbitals, $n_\mathrm{socc}$.  If this is not given, then
     \keywd{multiplicity} will be read.

  \item[\keywd{multiplicity}] This integer gives the multiplicity, $m$, of
     the molecule.  The number of singly occupied orbitals is then
     $n_\mathrm{socc} = m - 1$.  If neither \keywd{nsocc} nor
     \keywd{multiplicity} is given, then if, in consideration of
     \clsnmref{total\_charge}, the number of electrons is even, the default
     $n_\mathrm{socc}$ is 2.  Otherwise, it is 1.

  \item[\keywd{ndocc}] This integer gives the total number of doubly
     occupied orbitals $n_\mathrm{docc}$.  The default $n_\mathrm{docc} =
     (c - n_\mathrm{socc})/2$.

  \item[\keywd{socc}] This vector of integers gives the total number of
     singly occupied orbitals of each irreducible representation.  By
     default, the $n_\mathrm{socc}$ singly occupied orbitals will be
     distributed according to orbital eigenvalues.  If \keywd{socc} is
     given, then \keywd{docc} must be given and they override
     \keywd{nsocc}, \keywd{multiplicity}, \keywd{ndocc}, and
     \keywd{total\_charge}.

  \item[\keywd{docc}] This vector of integers gives the total number of
     doubly occupied orbitals of each irreducible representation.  By
     default, the $n_\mathrm{docc}$ singly occupied orbitals will be
     distributed according to orbital eigenvalues.  If \keywd{docc} is
     given, then \keywd{socc} must be given and they override
     \keywd{nsocc}, \keywd{multiplicity}, \keywd{ndocc}, and
     \keywd{total\_charge}.

  \item[\keywd{maxiter}] This has the same meaning as in the parent
     class, \clsnmref{SCF}; however, the default value is 100.

  \item[\keywd{level\_shift}] This has the same meaning as in the parent
     class, \clsnmref{SCF}; however, the default value is 1.0.

\end{description}
