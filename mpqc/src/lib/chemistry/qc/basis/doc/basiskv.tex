
%%%%%%%%%%%%%%%%%%%%%%%%%%%%%%%%%%%%%%%%%%%%%%%%%%%%%%%%%%%%%%%%%%%%%%%%%%

\section{The \clsnm{GaussianBasisSet} Class}
\label{GaussianBasisSet}\index{GaussianBasisSet}

The \clsnm{GaussianBasisSet} class is used describe a
basis set atomic composed of atomic gaussian orbitals.

\begin{description}
  \item[\keywd{molecule}] The gives a \clsnmref{Molecule} object.
     The is no default.

  \item[\keywd{puream}] If this boolean parameter is true then
     5D, 7F, etc. will be used.  Otherwise all cartesian functions
     will be used.  The default depends on the particular basis set.

  \item[\keywd{name}] This is a string giving the name of the basis set.
     Table~\ref{basissets} gives some of the recognized basis set names.
     It may be necessary to put the name in double quotes. There is no
     default.

  \item[\keywd{basis}] This is a vector of basis set names that can give a
     different basis set to each atom in the molecule.  If the
     \keywd{element} vector is given, then it gives different basis sets
     to different elements.  The default is to give every atom the
     basis set specified in \keywd{name}.

  \item[\keywd{element}] This is a vector of elements.  If it is given
     then it must have the same number of entries as the \keywd{basis}
     vector.

  \item[\keywd{basisdir}] A string giving a directory where basis
     set data files are to be sought.  See the text below for a
     complete description of what directors are consulted.

  \item[\keywd{basisfiles}] Each keyword in this vector of files is
     appended to the directory specified with \keywd{basisdir} and basis
     set data is read from them.

  \item[\keywd{matrixkit}] Specifies a \clsnmref{SCMatrixKit} object.
     It is usually not necessary to give this keyword, as the default
     action should get the correct \clsnmref{SCMatrixKit}.

\end{description}

Several files in various directories are checked for basis set data.
First, basis sets can be given by the user in the \keywd{basis} section at
the top level of the main input file.  Next, if a path is given with the
\keywd{basisdir} keyword, then all of the files given with the
\keywd{basisfiles} keyword are read in after appending their names to the
value of \keywd{basisdir}.  Basis sets can be given in these files in the
\keywd{basis} section at the top level as well.  If the named basis set
still cannot be found, then \clsnm{GaussianBasisSet} will try convert the
basis set name to a file name and check first in the directory given by
\keywd{basisdir}.  Next it checks for the environment variable
\verb|SCLIBDIR|.  If it is set it will look for the basis file in
\filnm{\$SCLIBDIR/basis}.  Otherwise it will look in the source code
distribution in the directory \filnm{SC/lib/basis}.  If the executable has
changed machines or the source code has be moved, then it may be necessary
to copy the library files to your machine and set the \verb|SCLIBDIR|
environmental variable.

The basis set itself is also given in the \clsnmref{ParsedKeyVal} format.
It is a \htmlref{vector}{pkvarray} of shells with the keyword
\keywd{:basis:} followed by the lowercase atomic name followed by \keywd{:}
followed by the basis set name (which may need to be placed inside double
quotes).  Each shell reads the following keywords:
\begin{description}
  \item[\keywd{type}] This is a \htmlref{vector}{pkvarray} that describes
    each component of this shell.  For each element the following two
    keywords are read:

  \begin{description}

    \item[\keywd{am}] The angular momentum of the component.  This can be
       given as the letter designation, \keywd{s}, \keywd{p}, \keywd{d},
       etc.  There is no default.

    \item[\keywd{puream}] If this boolean parameter is true then 5D, 7F,
       etc. shells are used.  The default is false.  This parameter can be
       overridden in the \clsnmref{GaussianBasisSet} specification.

  \end{description}

  \item[\keywd{exp}]  This is a \htmlref{vector}{pkvarray} giving the
     exponents of the primitive Gaussian functions.

  \item[\keywd{coef}] This is a \htmlref{matrix}{pkvarray} giving the
     coeffients of the primitive Gaussian functions.  The first index gives
     the component number of the shell and the second gives the primitive
     number.

\end{description}

An example might be easier to understand.  This is a basis set specificition
for STO-2G carbon:
\begin{alltt}
basis: (
 carbon: "STO-2G": [
  (type: [(am = s)]
   \{      exp      coef:0\} = \{
      27.38503303 0.43012850
       4.87452205 0.67891353
   \})
  (type: [(am = p) (am = s)]
   \{     exp      coef:1     coef:0\} = \{
       1.13674819 0.04947177 0.51154071
       0.28830936 0.96378241 0.61281990
   \})
 ]
)
\end{alltt}

\begin{table}
\caption{Available Basis Sets}
\begin{center}
\begin{tabular}{l}
  \multicolumn{1}{c}{Basis Set} \\
 \keywd{STO-2G} \\
 \keywd{STO-3G} \\
 \keywd{STO-6G} \\
 \keywd{3-21G} \\
 \keywd{3-21G*} \\
 \keywd{3-21++G} \\
 \keywd{3-21++G*} \\
 \keywd{4-31G} \\
 \keywd{4-31G*} \\
 \keywd{4-31G**} \\
 \keywd{6-31G} \\
 \keywd{6-31G*} \\
 \keywd{6-31G**} \\
 \keywd{6-31+G*} \\
 \keywd{6-31++G} \\
 \keywd{6-31++G*} \\
 \keywd{6-311G} \\
 \keywd{6-311G*} \\
 \keywd{6-311G**} \\
 \keywd{6-311++G**} \\
 \keywd{cc-pVDZ} \\
 \keywd{cc-pVTZ} \\
 \keywd{cc-pVQZ} \\
 \keywd{aug-cc-pVDZ} \\
 \keywd{aug-cc-pVTZ} \\
 \keywd{aug-cc-pVQZ} \\
\end{tabular}
\end{center}
\label{basissets}
\end{table}

%%%%%%%%%%%%%%%%%%%%%%%%%%%%%%%%%%%%%%%%%%%%%%%%%%%%%%%%%%%%%%%%%%%%%%%%%%

\section{The \clsnm{Integral} Class}
\label{Integral}\index{Integral}

The \clsnm{Integral} abstract class acts as a factory to provide
objects that compute one and two electron integrals.
