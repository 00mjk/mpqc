
%%%%%%%%%%%%%%%%%%%%%%%%%%%%%%%%%%%%%%%%%%%%%%%%%%%%%%%%%%%%%%%%%%%%%%%%%%

\section{The \clsnm{GaussianBasisSet} Class}
\label{GaussianBasisSet}\index{GaussianBasisSet}

The \clsnm{GaussianBasisSet} class is used describe a
basis set atomic composed of atomic gaussian orbitals.

\begin{description}
  \item[\keywd{molecule}] The gives a \clsnmref{Molecule} object.
     The is no default.

  \item[\keywd{puream}] If this boolean parameter is true then
     5D, 7F, etc. will be used.  Otherwise all cartesian functions
     will be used.  The default depends on the particular basis set.

  \item[\keywd{name}] This is a string giving the name of the basis set.
     Table~\ref{basissets} gives some of the recognized basis set names.
     It may be necessary to put the name in double quotes. There is no
     default.

  \item[\keywd{basis}] This is a vector of basis set names that can give a
     different basis set to each atom in the molecule.  If the
     \keywd{element} vector is given, then it gives different basis sets
     to different elements.  The default is to give every atom the
     basis set specified in \keywd{name}.

  \item[\keywd{element}] This is a vector of elements.  If it is given
     then it must have the same number of entries as the \keywd{basis}
     vector.

  \item[\keywd{basisdir}] A string giving a directory where basis
     set data files are to be sought.  See the text below for a
     complete description of what directors are consulted.

  \item[\keywd{basisfiles}] Each keyword in this vector of files is
     appended to the directory specified with \keywd{basisdir} and basis
     set data is read from them.

  \item[\keywd{matrixkit}] Specifies a \clsnmref{SCMatrixKit} object.
     It is usually not necessary to give this keyword, as the default
     action should get the correct \clsnmref{SCMatrixKit}.

\end{description}

Several files in various directories are checked for basis set data.
First, basis sets can be given by the user in the \keywd{basis} section at
the top level of the main input file.  Next, if a path is given with the
\keywd{basisdir} keyword, then all of the files given with the
\keywd{basisfiles} keyword are read in after appending their names to the
value of \keywd{basisdir}.  Basis sets can be given in these files in the
\keywd{basis} section at the top level as well.  If the named basis set
still cannot be found, then \clsnm{GaussianBasisSet} will try convert the
basis set name to a file name and check first in the directory given by
\keywd{basisdir}.  Next it checks for the environment variable
\verb|SCLIBDIR|.  If it is set it will look for the basis file in
\filnm{\$SCLIBDIR/basis}.  Otherwise it will look in the source code
distribution in the directory \filnm{SC/lib/basis}.  If the executable has
changed machines or the source code has be moved, then it may be necessary
to copy the library files to your machine and set the \verb|SCLIBDIR|
environmental variable.

The basis set itself is also given in the \clsnmref{ParsedKeyVal} format.
It is a \htmlref{vector}{pkvarray} of shells with the keyword
\keywd{:basis:} followed by the lowercase atomic name followed by \keywd{:}
followed by the basis set name (which may need to be placed inside double
quotes).  Each shell reads the following keywords:
\begin{description}
  \item[\keywd{type}] This is a \htmlref{vector}{pkvarray} that describes
    each component of this shell.  For each element the following two
    keywords are read:

  \begin{description}

    \item[\keywd{am}] The angular momentum of the component.  This can be
       given as the letter designation, \keywd{s}, \keywd{p}, \keywd{d},
       etc.  There is no default.

    \item[\keywd{puream}] If this boolean parameter is true then 5D, 7F,
       etc. shells are used.  The default is false.  This parameter can be
       overridden in the \clsnmref{GaussianBasisSet} specification.

  \end{description}

  \item[\keywd{exp}]  This is a \htmlref{vector}{pkvarray} giving the
     exponents of the primitive Gaussian functions.

  \item[\keywd{coef}] This is a \htmlref{matrix}{pkvarray} giving the
     coeffients of the primitive Gaussian functions.  The first index gives
     the component number of the shell and the second gives the primitive
     number.

\end{description}

An example might be easier to understand.  This is a basis set specificition
for STO-2G carbon:
\begin{alltt}
basis: (
 carbon: "STO-2G": [
  (type: [(am = s)]
   \{      exp      coef:0\} = \{
      27.38503303 0.43012850
       4.87452205 0.67891353
   \})
  (type: [(am = p) (am = s)]
   \{     exp      coef:1     coef:0\} = \{
       1.13674819 0.04947177 0.51154071
       0.28830936 0.96378241 0.61281990
   \})
 ]
)
\end{alltt}

\begin{htmlonly}
\begin{table}
\caption{Available Basis Sets with the supported elements for each basis
and the number of basis functions for H, $n_0$, first row, $n_1$, and
second row, $n_2$, atoms.}
\begin{tabular}{lcrrr}
  \multicolumn{1}{c}{Basis Set}&
   \multicolumn{1}{c}{Elements}&
   \multicolumn{1}{c}{$n_0$}&
   \multicolumn{1}{c}{$n_1$}&
   \multicolumn{1}{c}{$n_2$} \\
\verb*|STO-2G| & H-Ca & 1 & 5 & 9 \\
\verb*|STO-3G| & H-Kr & 1 & 5 & 9 \\
\verb*|STO-3G*| & H-Ar & 1 & 5 & 15 \\
\verb*|STO-6G| & H-Kr & 1 & 5 & 9 \\
\verb*|MINI (Huzinaga)| & H-Ca & 1 & 5 & 9 \\
\verb*|MINI (Scaled)| & H-Ca & 1 & 5 & 9 \\
\verb*|MIDI (Huzinaga)| & H-Na & 2 & 9 &  \\
\verb*|DZ (Dunning)| & H, Li, B-Ne, Al-Cl & 2 & 10 & 18 \\
\verb*|DZP (Dunning)| & H, Li, B-Ne, Al-Cl & 5 & 16 & 24 \\
\verb*|DZP + Diffuse (Dunning)| & H, B-Ne & 6 & 19 &  \\
\verb*|3-21G| & H-Kr & 2 & 9 & 13 \\
\verb*|3-21G*| & H-Ar & 2 & 9 & 19 \\
\verb*|3-21++G| & H-Ar & 3 & 13 & 17 \\
\verb*|3-21++G*| & H-Ar & 3 & 13 & 23 \\
\verb*|4-31G| & H-Ne, P-Cl & 2 & 9 & 13 \\
\verb*|4-31G*| & H-Ne, P-Cl & 2 & 15 & 19 \\
\verb*|4-31G**| & H-Ne, P-Cl & 5 & 15 & 19 \\
\verb*|6-31G| & H-Ar & 2 & 9 & 13 \\
\verb*|6-31G*| & H-Ar & 2 & 15 & 19 \\
\verb*|6-31G**| & H-Ar & 5 & 15 & 19 \\
\verb*|6-31+G*| & H-Ar & 2 & 19 & 23 \\
\verb*|6-31++G| & H-Ar & 3 & 13 & 17 \\
\verb*|6-31++G*| & H-Ar & 3 & 19 & 23 \\
\verb*|6-31++G**| & H-Ar & 6 & 19 & 23 \\
\verb*|6-311G| & H-Ar, Ga-Kr & 3 & 13 & 21 \\
\verb*|6-311G*| & H-Ar, Ga-Kr & 3 & 19 & 27 \\
\verb*|6-311G**| & H-Ar, Ga-Kr & 6 & 19 & 27 \\
\verb*|6-311G(2df,2pd)| & H-Ne & 15 & 35 &  \\
\verb*|6-311++G**| & H-Ne & 7 & 23 &  \\
\verb*|6-311++G(2d,2p)| & H-Ne & 10 & 29 &  \\
\verb*|6-311++G(3df,3pd)| & H-Ar & 19 & 45 & 53 \\
\verb*|cc-pVDZ| & H, He, B-Ne, Al-Ar & 5 & 14 & 18 \\
\verb*|cc-pVTZ| & H, He, B-Ne, Al-Ar & 14 & 30 & 34 \\
\verb*|cc-pVQZ| & H, He, B-Ne, Al-Ar & 30 & 55 & 59 \\
\verb*|cc-pV5Z| & H-Ne, Al-Ar & 55 & 91 & 95 \\
\verb*|aug-cc-pVDZ| & H, He, B-Ne, Al-Ar & 9 & 23 & 27 \\
\verb*|aug-cc-pVTZ| & H, He, B-Ne, Al-Ar & 23 & 46 & 50 \\
\verb*|aug-cc-pVQZ| & H, He, B-Ne, Al-Ar & 46 & 80 & 84 \\
\verb*|aug-cc-pV5Z| & H, He, B-Ne, Al-Ar & 80 & 127 & 131 \\
\verb*|cc-pCVDZ| & B-Ne &  & 18 &  \\
\verb*|cc-pCVTZ| & B-Ne &  & 43 &  \\
\verb*|cc-pCVQZ| & B-Ne &  & 84 &  \\
\verb*|cc-pCV5Z| & B-Ne &  & 145 &  \\
\verb*|aug-cc-pCVDZ| & B-F &  & 27 &  \\
\verb*|aug-cc-pCVTZ| & B-Ne &  & 59 &  \\
\verb*|aug-cc-pCVQZ| & B-Ne &  & 109 &  \\
\verb*|aug-cc-pCV5Z| & B-F &  & 181 &  \\
\verb*|NASA Ames ANO| & H, B-Ne, Al, P, Ti, Fe, Ni & 30 & 55 & 59 \\
\end{tabular}
\end{center}
\label{basissets}
\end{table}
\end{htmlonly}

\begin{latexonly}
\begin{longtable}{lcrrr}
\caption{Available Basis Sets with the supported elements for each basis
and the number of basis functions for H, $n_0$, first row, $n_1$, and
second row, $n_2$, atoms.} \\
\hline\hline
  \multicolumn{1}{c}{Basis Set}&
   \multicolumn{1}{c}{Elements}&
   \multicolumn{1}{c}{$n_0$}&
   \multicolumn{1}{c}{$n_1$}&
   \multicolumn{1}{c}{$n_2$} \\
\hline
\endfirsthead
\multicolumn{5}{l}{\small\sl continued from previous page} \\ \hline
\endhead
\hline \multicolumn{5}{r}{\small\sl continued on next page} \\
\endfoot
\hline\hline
\endlastfoot
\verb*|STO-2G| & H-Ca & 1 & 5 & 9 \\
\verb*|STO-3G| & H-Kr & 1 & 5 & 9 \\
\verb*|STO-3G*| & H-Ar & 1 & 5 & 15 \\
\verb*|STO-6G| & H-Kr & 1 & 5 & 9 \\
\verb*|MINI (Huzinaga)| & H-Ca & 1 & 5 & 9 \\
\verb*|MINI (Scaled)| & H-Ca & 1 & 5 & 9 \\
\verb*|MIDI (Huzinaga)| & H-Na & 2 & 9 &  \\
\verb*|DZ (Dunning)| & H, Li, B-Ne, Al-Cl & 2 & 10 & 18 \\
\verb*|DZP (Dunning)| & H, Li, B-Ne, Al-Cl & 5 & 16 & 24 \\
\verb*|DZP + Diffuse (Dunning)| & H, B-Ne & 6 & 19 &  \\
\verb*|3-21G| & H-Kr & 2 & 9 & 13 \\
\verb*|3-21G*| & H-Ar & 2 & 9 & 19 \\
\verb*|3-21++G| & H-Ar & 3 & 13 & 17 \\
\verb*|3-21++G*| & H-Ar & 3 & 13 & 23 \\
\verb*|4-31G| & H-Ne, P-Cl & 2 & 9 & 13 \\
\verb*|4-31G*| & H-Ne, P-Cl & 2 & 15 & 19 \\
\verb*|4-31G**| & H-Ne, P-Cl & 5 & 15 & 19 \\
\verb*|6-31G| & H-Ar & 2 & 9 & 13 \\
\verb*|6-31G*| & H-Ar & 2 & 15 & 19 \\
\verb*|6-31G**| & H-Ar & 5 & 15 & 19 \\
\verb*|6-31+G*| & H-Ar & 2 & 19 & 23 \\
\verb*|6-31++G| & H-Ar & 3 & 13 & 17 \\
\verb*|6-31++G*| & H-Ar & 3 & 19 & 23 \\
\verb*|6-31++G**| & H-Ar & 6 & 19 & 23 \\
\verb*|6-311G| & H-Ar, Ga-Kr & 3 & 13 & 21 \\
\verb*|6-311G*| & H-Ar, Ga-Kr & 3 & 19 & 27 \\
\verb*|6-311G**| & H-Ar, Ga-Kr & 6 & 19 & 27 \\
\verb*|6-311G(2df,2pd)| & H-Ne & 15 & 35 &  \\
\verb*|6-311++G**| & H-Ne & 7 & 23 &  \\
\verb*|6-311++G(2d,2p)| & H-Ne & 10 & 29 &  \\
\verb*|6-311++G(3df,3pd)| & H-Ar & 19 & 45 & 53 \\
\verb*|cc-pVDZ| & H, He, B-Ne, Al-Ar & 5 & 14 & 18 \\
\verb*|cc-pVTZ| & H, He, B-Ne, Al-Ar & 14 & 30 & 34 \\
\verb*|cc-pVQZ| & H, He, B-Ne, Al-Ar & 30 & 55 & 59 \\
\verb*|cc-pV5Z| & H-Ne, Al-Ar & 55 & 91 & 95 \\
\verb*|aug-cc-pVDZ| & H, He, B-Ne, Al-Ar & 9 & 23 & 27 \\
\verb*|aug-cc-pVTZ| & H, He, B-Ne, Al-Ar & 23 & 46 & 50 \\
\verb*|aug-cc-pVQZ| & H, He, B-Ne, Al-Ar & 46 & 80 & 84 \\
\verb*|aug-cc-pV5Z| & H, He, B-Ne, Al-Ar & 80 & 127 & 131 \\
\verb*|cc-pCVDZ| & B-Ne &  & 18 &  \\
\verb*|cc-pCVTZ| & B-Ne &  & 43 &  \\
\verb*|cc-pCVQZ| & B-Ne &  & 84 &  \\
\verb*|cc-pCV5Z| & B-Ne &  & 145 &  \\
\verb*|aug-cc-pCVDZ| & B-F &  & 27 &  \\
\verb*|aug-cc-pCVTZ| & B-Ne &  & 59 &  \\
\verb*|aug-cc-pCVQZ| & B-Ne &  & 109 &  \\
\verb*|aug-cc-pCV5Z| & B-F &  & 181 &  \\
\verb*|NASA Ames ANO| & H, B-Ne, Al, P, Ti, Fe, Ni & 30 & 55 & 59 \\
\label{basissets}
\end{longtable}
\end{latexonly}

Nearly all the basis sets provided with the distribution were obtained from
the the Pacific Northwest Laboratory basis set database.  They were then
translated to the \clsnmref{ParsedKeyVal} format.  If you find an error
please report it to the mpqc development team.  Here is the citation for
the database:
\begin{quote}
Basis sets were obtained from the Extensible Computational Chemistry
Environment Basis Set Database, Version 1.0, as developed and
distributed by the Molecular Science Computing Facility, Environmental
and Molecular Sciences Laboratory which is part of the Pacific
Northwest Laboratory, P.O. Box 999, Richland, Washington 99352, USA,
and funded by the U.S. Department of Energy. The Pacific Northwest
Laboratory is a multi-program laboratory operated by Battelle Memorial
Institue for the U.S. Department of Energy under contract
DE-AC06-76RLO 1830. Contact David Feller, Karen Schuchardt, or Don
Jones for further information.
\end{quote}
The original scientific citations for each basis set are at the top of its
data file.

%%%%%%%%%%%%%%%%%%%%%%%%%%%%%%%%%%%%%%%%%%%%%%%%%%%%%%%%%%%%%%%%%%%%%%%%%%

\section{The \clsnm{Integral} Class}
\label{Integral}\index{Integral}

The \clsnm{Integral} abstract class acts as a factory to provide
objects that compute one and two electron integrals.
