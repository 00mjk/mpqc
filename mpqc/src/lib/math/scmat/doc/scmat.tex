
\addtocounter{clsheadingnum}{1}
\addtocounter{clssubheadingnum}{1}

\chapter{The Matrix Library}
%\doheading{2}{Introduction}

The scientific computing matrix library (SCMAT) is designed around a set
of matrix abstractions that permit very general matrix implementations.
This flexibility is needed to support diverse computing environments.
For example, this library must support, at a minimum: simple matrices
that provide efficient matrix computations in a uniprocessor
environment, clusters of processors with enough memory to store all
matrices connected by a relatively slow network (workstations on an
LAN), clusters of processors with enough memory to store all matrices
and a fast interconnect network (a massively parallel machine such as
the Intel Paragon), and clusters of machines that don't have enough
memory to hold entire matrices.

The design of SCMAT differs from other object-oriented matrix packages
in two important ways.  First, the matrix classes are abstract base
classes.  No storage layout is defined and virtual function calls must
be used to access individual matrix elements.  This would have a
negative performance impact if users needed to frequently access matrix
elements.  The interface to the matrix classes is hopefully rich enough
to avoid individual matrix element access for any computationally
significant task.  The second major difference is that symmetric
matrices do not inherit from matrices, etc.  The SCMAT user must know
whether a matrix is symmetric at all places it is used if any
performance gain, by virtue of symmetry, is expected.

Dimension information is contained objects of the \clsnmref{SCDimension}
type.  In addition to the simple integer dimension, application specific
blocking information can be provided.  For example, in a quantum chemistry
application, the dimension corresponding to the atomic orbital basis set
will have block sizes that correspond to the shells.  Dimensions are used
to create new matrix or vector objects.

The primary abstract classes are \clsnmref{SCMatrix},
\clsnmref{SymmSCMatrix}, \clsnmref{DiagSCMatrix}, and \clsnmref{SCVector}.
These represent matrices, symmetric matrices, diagonal matrices, and
vectors, respectively.  These abstract classes are specialized into groups
of classes.  For example, the locally stored matrix implementation
specializes the abstract classes to \clsnmref{LocalSCMatrix},
\clsnmref{LocalSymmSCMatrix}, \clsnmref{LocalDiagSCMatrix},
\clsnmref{LocalSCVector}, \clsnmref{LocalSCDimension}, and
\clsnmref{LocalSCMatrixKit}.  These specializations are all designed to
work with each other.  However, a given specialization is incompatible with
other matrix specializations.  An attempt to multiply a local matrix by a
distributed matrix would generate an error at runtime.

Since the different groups of classes do not interoperate, some mechanism
of creating consistent specializations is needed.  This is done with
\clsnmref{SCMatrixKit} objects.  \clsnmref{SCMatrixKit} is an abstract base
type which has specializations that correspond to each group of the matrix
specializations.  It is used to create matrices and vectors from that
group.  For example, the \clsnmref{DistSCMatrixKit} is used to create
objects of type \clsnmref{DistSCMatrix}, \clsnmref{DistSymmSCMatrix},
\clsnmref{DistDiagSCMatrix}, and \clsnmref{DistSCVector}.

The abstract matrix classes and their derivations are usually not directly
used by SCMAT users.  The most convenient classes to use are the smart
pointer classes \clsnmref{RefSCMatrix}, \clsnmref{RefSymmSCMatrix},
\clsnmref{RefDiagSCMatrix}, \clsnmref{RefSCDimension}, and
\clsnmref{RefSCMatrixKit}.  These automatically delete matrix objects when
they are no longer needed.  This is through a reference count mechanism
that is supported by the \clsnmref{VRefCount} base class from which the
abstract matrix classes derive.  The smart pointer classes also have matrix
operations such as \srccd{operator *()}, \srccd{operator -()}, and
\srccd{operator +()} defined as members for convenience.  These forward the
operations to the contained matrix object.  The smart pointer classes also
simplify creation of matrices by providing constructors that take as
arguments one or more \clsnmref{RefSCDimension}'s and a
\clsnmref{RefSCMatrixKit}.  These initialize the smart pointer to contain a
new matrix with a specialization corresponding to that of the
\clsnmref{RefSCMatrixKit}.  Matrix operations not provided by the smart
pointer classes but present as member in the abstract classes can be
accessed with \srccd{operator->()}.

If a needed matrix operation is missing, mechanisms exist to add more
general operations.  Operations which only depend on individual elements of
matrices can be provided by specializations of the \clsnmref{SCElementOp}
class.  Sometimes we need operations on matrices with identical dimensions
that examine each element in one matrix along with the corresponding
element from the other matrix.  This is accomplished with
\clsnmref{SCElementOp2} for two matrices and with \clsnmref{SCElementOp3}
for three.

Other features of SCMAT include run-time type facilities and persistence.
Castdown operations (type conversions from less to more derived objects)
and other run-time type information are provided by the
\clsnmref{DescribedClass} base class.  Persistence is not provided by
inheriting from \clsnmref{SavableState} base clase as is the case with many
other classes in the SC class hierarchies, because it is necessary to save
objects in an implementation independent manner.  If a calculation
checkpoints a matrix on a single processor machine and later is restarted
on a multiprocessor machine the matrix would need to be restored as a
different matrix specialization.  This is handled by saving and restoring
matrices' and vectors' data without reference to the specialization.

The following include files are provided by the matrix library:

\begin{description}
\item[\filnm{matrix.h}]
Usually, this is the only include file needed by users of matrices.  It
declares reference counting pointers to abstract matrices.

If kit for a matrix must be created, or a member specific to an
implementation is needed, then that implementation's header file must be
included.

\item[\filnm{elemop.h}]
This is the next most useful include file.  It defines useful
\clsnmref{SCElementOp}, \clsnmref{SCElementOp2}, and \clsnmref{SCElementOp3}
specializations.

\item[\filnm{abstract.h}]
This include file contains the declarations for abstract classes that
users do not usually need to see.  These include \clsnmref{SCDimension},
\clsnmref{SCMatrix}, \clsnmref{SymmSCMatrix}, \clsnmref{DiagSCMatrix},
\clsnmref{SCMatrixKit}.  This file is currently included by
\filnm{matrix.h}.

\item[\filnm{block.h}]
This file declares \clsnmref{SCMatrixBlock} and specializations.  It
only need be include by users implementing new \clsnmref{SCElementOp}
specializations.

\item[\filnm{blkiter.h}]
This include file declares the implementations of
\clsnmref{SCMatrixBlockIter}.  It only need be include by users implementing
new \clsnmref{SCElementOp} specializations.

\item[\filnm{vector3.h}]
This declares \clsnmref{SCVector3}, a lightweight vector of length three.

\item[\filnm{matrix3.h}]
This declares \clsnmref{SCMatrix3}, a lightweight matrix of dimension three by
three.  It includes \filnm{vector3.h}.

\item[\filnm{local.h}]
This include file is the matrix implementation for locally stored
matrices.  These are suitable for use in a uniprocessor environment.  The
\clsnmref{LocalSCMatrixKit} is the default matrix implementation returned
by the static member \clsnmref{SCMatrixKit}\srccd{::default\_matrixkit}.
This file usually doesn't need to be included.

\item[\filnm{dist.h}]
This include file is the matrix implementation for distributed matrices.
These are suitable for use in a distributed memory multiprocessor which
does not have enough memory to hold all of the matrix elements on each
processor.  This file usually doesn't need to be included.

\item[\filnm{repl.h}]
This include file is the matrix implementation for replicated matrices.
These are suitable for use in a distributed memory multiprocessor which
does have enough memory to hold all of the matrix elements on each
processor.  This file usually doesn't need to be included.

\item[\filnm{blocked.h}]
This include file is the matrix implementation for blocked matrices.
Blocked matrices store a matrix as subblocks that are matrices from another
matrix specialization.  These are used to save storage and computation time
in quantum chemistry applications for molecules with other than $C_1$ point
group symmetry.

\end{description}

\section{Matrix Dimensions}
%\doheading{2}{Matrix Dimensions}

In addition to the simple integer dimension, objects of the
\clsnmref{SCDimension} class contain application specific blocking
information.  This information is held in an object of class
\clsnmref{SCBlockInfo}.

\input{math/scmat/doc/SCDimension.cls.tex}
\input{math/scmat/doc/SCBlockInfo.cls.tex}

\section{Matrix Reference Classes}\label{matrixrefclass}
%\doheading{2}{Matrix Reference Classes}\label{matrixrefclass}

The easiest way to use SCMAT is through the smart pointer classes
\clsnmref{RefSCMatrix}, \clsnmref{RefSymmSCMatrix},
\clsnmref{RefDiagSCMatrix}, \clsnmref{RefSCVector},
\clsnmref{RefSCDimension}, and \clsnmref{RefSCMatrixKit}.  These are based
on the \clsnmref{Ref} reference counting package and automatically delete
matrix objects when they are no longer needed.  These reference classes
also have common operations defined as members for convenience.  This makes
it unnecessary to also use the sometimes awkward syntax of
\srccd{operator->()} to manipulate the contained objects.

\input{math/scmat/doc/RefSCDimension.cls.tex}
\input{math/scmat/doc/RefSCVector.cls.tex}
\input{math/scmat/doc/RefSCMatrix.cls.tex}
\input{math/scmat/doc/RefSymmSCMatrix.cls.tex}
\input{math/scmat/doc/RefDiagSCMatrix.cls.tex}

\section{Abstract Matrix Classes}
%\doheading{2}{Abstract Matrix Classes}

This section documents the primary abstract classes: \clsnmref{SCMatrix},
\clsnmref{SymmSCMatrix}, \clsnmref{DiagSCMatrix}, and \clsnmref{SCVector},
as well as the \clsnmref{SCMatrixKit} class which allows the programmer to
generate consistent specializations of matrices.  These represent matrices,
symmetric matrices, diagonal matrices, and vectors, respectively.

This section is primarily for implementers of new specializations
of matrices.  Users of existing matrices will be most interested
in \htmlref{the matrix reference classes}{matrixrefclass}.

\input{math/scmat/doc/SCMatrixKit.cls.tex}
\input{math/scmat/doc/SCVector.cls.tex}
\input{math/scmat/doc/SCMatrix.cls.tex}
\input{math/scmat/doc/SymmSCMatrix.cls.tex}
\input{math/scmat/doc/DiagSCMatrix.cls.tex}

\section{Matrix Storage}
%\doheading{2}{Matrix Storage}

All elements of matrices and vectors are kept in blocks.  The
choice of blocks and where they are keep is left up to each
matrix specialization.

\input{math/scmat/doc/SCMatrixBlock.cls.tex}
\input{math/scmat/doc/SCMatrixRectBlock.cls.tex}
\input{math/scmat/doc/SCMatrixRectSubBlock.cls.tex}
\input{math/scmat/doc/SCMatrixLTriBlock.cls.tex}
\input{math/scmat/doc/SCMatrixLTriSubBlock.cls.tex}
\input{math/scmat/doc/SCMatrixDiagBlock.cls.tex}
\input{math/scmat/doc/SCMatrixDiagSubBlock.cls.tex}
\input{math/scmat/doc/SCVectorSimpleBlock.cls.tex}
\input{math/scmat/doc/SCVectorSimpleSubBlock.cls.tex}

\section{Manipulating Matrix Elements with Element Operations}
%\doheading{2}{Manipulating Matrix Elements with Element Operations}

\input{math/scmat/doc/SCElementOp.cls.tex}
\input{math/scmat/doc/SCElementOp2.cls.tex}
\input{math/scmat/doc/SCElementOp3.cls.tex}
\input{math/scmat/doc/SCMatrixBlockIter.cls.tex}

\section{\clsnm{SCElementOp} Specializations}
%\doheading{2}{\clsnm{SCElementOp} Specializations}

Several commonly needed element operations are already coded
up and available by including \filnm{math/scmat/elemop.h}.
Below are descriptions of these classes:

\begin{description}
\item[\clsnm{SCElementScalarProduct}] This \clsnmref{SCElementOp2} computes
the scalar product of two matrices or vectors.  The result is available
after the operation from the return value of the \srccd{result()} member.
\item[\clsnm{SCDestructiveElementProduct}] This \clsnmref{SCElementOp2}
replaces the elements of the matrix or vector whose \srccd{element\_op}
member is called.  The resulting values are the element by element products
of the two matrices or vectors.
\item[\clsnm{SCElementScale}] This scales each element by an amount given
in the constructor.
\item[\clsnm{SCElementRandomize}] This generates random elements.
\item[\clsnm{SCElementAssign}] Assign to each element the value passed to
the constructor.
\item[\clsnm{SCElementSquareRoot}] Replace each element with its square
root.
\item[\clsnm{SCElementInvert}] Replace each element by its reciprocal.
\item[\clsnm{SCElementScaleDiagonal}] Scales the diagonal elements of a
matrix by the argument passed to the constructor.  Use of this on a vector
is undefined.
\item[\clsnm{SCElementShiftDiagonal}] Add the value passed to the
constructor to the diagonal elements of the matrix.  Use of this on a
vector is undefined.
\item[\clsnm{SCElementMaxAbs}] Find the maximum absolute value element in a
matrix or vector.  The result is available as the return value of the
\srccd{result()} member.
\item[\clsnm{SCElementDot}] The constructor for this class takes three
arguments: \linebreak \srccd{SCElementDot(double**\vrbl{a},
double**\vrbl{b}, int \vrbl{length})}.  The length of each vector given by
\vrbl{a} and \vrbl{b} is given by \vrbl{length}.  The number of vectors in
\vrbl{a} is the number of rows in the matrix and the number in \vrbl{b} is
the number of columns.  To each element in the matrix $m_{ij}$ the dot
product of the $a_i$ and $b_j$ is added.
\item[\clsnm{SCElementAccumulateSCMatrix}]  This is obsolete---do not use it.
\item[\clsnm{SCElementAccumulateSymmSCMatrix}] This is obsolete---do not
use it.
\item[\clsnm{SCElementAccumulateDiagSCMatrix}] This is obsolete---do not
use it.
\item[\clsnm{SCElementAccumulateSCVector}] This is obsolete---do not use
it.
\end{description}

\section{Manipulating Matrix Elements with Block Iterators}
%\doheading{2}{Manipulating Matrix Elements with Block Iterators}

\input{math/scmat/doc/SCMatrixSubblockIter.cls.tex}

\subsection{Local Matrices}
%\doheading{3}{Local Matrices}

Local matrices do no communication.  All elements reside on each node
and all computations are duplicated on each node.

\subsection{Replicated Matrices}
%\doheading{3}{Replicated Matrices}

Replicated matrices hold all of the elements on each node, however
do some communications in order to reduce computation time.

\subsection{Distributed Matrices}
%\doheading{3}{Distributed Matrices}

Distributed matrices spread the elements across all the nodes and
thus require less storage than local matrices however these use
more communications that replicated matrices.

\subsection{Blocked Matrices}
%\doheading{3}{Blocked Matrices}

Blocked matrices are used to implement point group symmetry.  Another
matrix specialization is used to hold the diagonal subblocks of a
matrix.  The offdiagonal subblocks are known to be zero and not stored.
This results in considerable savings in storage and computation for
those cases where it applies.

\addtocounter{clsheadingnum}{-1}
\addtocounter{clssubheadingnum}{-1}
