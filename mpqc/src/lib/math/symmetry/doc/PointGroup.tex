
\clssection{PointGroup}

The \clsnm{PointGroup} \clsnmref{KeyVal} constructor looks
for three keywords: \srccd{symmetry},
\srccd{symmetry\_frame}, and
\srccd{origin}. \srccd{symmetry} is a string containing the
Schoenflies symbol of the point group.  \srccd{origin} is an
array of doubles which gives the x, y, and z coordinates of
the origin of the symmetry frame.  \srccd{symmetry\_frame}
is a 3 by 3 array of arrays of doubles which specify the
principal axes for the transformation matrices as a unitary
rotation.

For example, a simple input which will use the default
\srccd{origin} and \srccd{symmetry\_frame} ((0,0,0) and the
unit matrix, respectively), might look like this:

\begin{alltt}
pointgrp<\clsnmref{PointGroup}>: (
  symmetry = "c2v"
)
\end{alltt}

By default, the principal rotation axis is taken to be the z
axis.  If you already have a set of coordinates which assume
that the rotation axis is the x axis, then you'll have to
rotate your frame of reference with \srccd{symmetry\_frame}:

\begin{alltt}
pointgrp<\clsnmref{PointGroup}>: (
  symmetry = "c2v"
  symmetry_frame = [
    [ 0 0 1 ]
    [ 0 1 0 ]
    [ 1 0 0 ]
  ]
)
\end{alltt}
