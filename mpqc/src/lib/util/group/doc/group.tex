
\chapter{The Group Library}

The group library provides two important abstract base
classes for communications.  The \clsnmref{MessageGrp} class
is the for standard message passing with explicit sends and
receives on all involved nodes.  A generic technique for
doing global operations is provided.  Common global
reductions, such as a global sum, are directly provided for
convenience.  The \clsnmref{MessageGrp} class has
specializations that permits its use on a wide variety of
hardware:

\begin{itemize}
\item \clsnm{PVMMessageGrp} uses the PVM package
\item \clsnm{MPIMessageGrp} uses the MPI library
\item \clsnm{ParagonMessageGrp} uses the Intel Paragon's NX library
\item \clsnm{ShmMessageGrp} uses System V IPC (for
      symmetric multiprocessors)
\end{itemize}

The other abstract communication class of importance,
\clsnmref{MemoryGrp}, provides the appearance of global
shared memory.  Only where communications with System V IPC
is available to all of the processes can real shared memory
be used.  Otherwise active messages or fake active messages
must be used to request pieces of memory that do not lie on
the local node.  True active messages are available when
using the OSF operating system on the Intel Paragon.
Otherwise the active messages are `faked' by checking for
outstanding memory requests whenever a communications class
member is called.  Faking active messages is not desirable
and will be replaced by true active messages as more
libraries are able to support them.

