
\section{The \clsnm{Debugger} Class}
\label{Debugger}\index{Debugger}

The \clsnm{Debugger} class describes what should be done when
a catastrophic error causes unexpected program termination.
It can try things such as start a debugger running where
the program died or it can attempt to produce a stack traceback
showing roughly where the program died.  These attempts will not
always succeed.

The \clsnm{KeyVal} input parameters for \clsnm{Debugger} are given
below.

\begin{description}
 \item[\keywd{debug}] Try to start a debugger when an error occurs.
        Doesn't work on all machines. The default is true, if possible.

 \item[\keywd{traceback}] Try to print out a traceback extracting return
        addresses from the call stack.  Doesn't work on most machines.  The
        default is true, if possible.

 \item[\keywd{exit}] Exit on errors.  The default is true.

 \item[\keywd{wait\_for\_debugger}] When starting a debugger go into an
        infinite loop to give the debugger a chance to attach to the
        process.  The default is true.

 \item[\keywd{sleep}] When starting a debugger wait this many seconds to
        give the debugger a chance to attach to the process.  The default
        is 0.

 \item[\keywd{handle\_defaults}] Handle a standard set of signals such as
        SIGBUS, SIGSEGV, etc.  The default is true.

 \item[\keywd{prefix}] Gives a string that is printed before each line that
        is printed by \clsnm{Debugger}. The default is nothing.

 \item[\keywd{cmd}] Gives a command to be executed to start the debugger.
        The default varies with machine.

\end{description}
