
\clssection{Debugger}
\index{Debugger}

The \clsnm{Debugger} class describes what should be done when
a catastrophic error causes unexpected program termination.
It can try things such as run start a debugger running where
the program died or it can attempt to produce a stack traceback
showing roughly where the program died.  These attempts could
just complicate matters more if the program has gotten into
a peculiar enough state.

Table~\ref{debugger:keyval} shows the \clsnm{KeyVal} input parameters
for \clsnm{Debugger}.

\begin{table}
\caption{\clsnmref{KeyVal} Input for \clsnm{Debugger}}
\begin{center}
\begin{tabular}{lp{2.5in}p{1in}}
  \multicolumn{1}{c}{Keyword}
     & \multicolumn{1}{c}{Meaning}
     & \multicolumn{1}{c}{Default} \\
 \verb|debug|
        & Try to start a debugger when an error occurs.  Doesn't
        work on all machines.
        & \verb|yes|, if possible \\
 \verb|traceback|
        & Try to print out a traceback extracting return addresses
        from the call stack.  Doesn't work on most machines.
        & \verb|yes|, if possible \\
 \verb|exit|
        & Exit on errors.
        & \verb|yes| \\
 \verb|wait_for_debugger|
        & When starting a debugger go into an infinite loop
        to give the debugger a chance to attach to the process.
        & \verb|yes| \\
 \verb|sleep|
        & When starting a debugger wait this many seconds
        to give the debugger a chance to attach to the process.
        & \verb|0| \\
 \verb|handle_defaults|
        & Handle a standard set of signals such as SIGBUS, SIGSEGV, etc.
        & \verb|yes| \\
 \verb|prefix|
        & Gives a string that is printed before each line
        that is printed by \clsnm{Debugger}.
        & empty \\
 \verb|cmd|
        & Gives a command to be executed to start the debugger.
        & varies with machine
\end{tabular}
\end{center}
\label{debugger:keyval}
\end{table}
